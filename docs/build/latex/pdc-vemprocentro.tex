%% Generated by Sphinx.
\def\sphinxdocclass{report}
\documentclass[letterpaper,10pt,brazil]{sphinxmanual}
\ifdefined\pdfpxdimen
   \let\sphinxpxdimen\pdfpxdimen\else\newdimen\sphinxpxdimen
\fi \sphinxpxdimen=.75bp\relax
\ifdefined\pdfimageresolution
    \pdfimageresolution= \numexpr \dimexpr1in\relax/\sphinxpxdimen\relax
\fi
%% let collapsible pdf bookmarks panel have high depth per default
\PassOptionsToPackage{bookmarksdepth=5}{hyperref}


\PassOptionsToPackage{warn}{textcomp}
\usepackage[utf8]{inputenc}
\ifdefined\DeclareUnicodeCharacter
% support both utf8 and utf8x syntaxes
  \ifdefined\DeclareUnicodeCharacterAsOptional
    \def\sphinxDUC#1{\DeclareUnicodeCharacter{"#1}}
  \else
    \let\sphinxDUC\DeclareUnicodeCharacter
  \fi
  \sphinxDUC{00A0}{\nobreakspace}
  \sphinxDUC{2500}{\sphinxunichar{2500}}
  \sphinxDUC{2502}{\sphinxunichar{2502}}
  \sphinxDUC{2514}{\sphinxunichar{2514}}
  \sphinxDUC{251C}{\sphinxunichar{251C}}
  \sphinxDUC{2572}{\textbackslash}
\fi
\usepackage{cmap}
\usepackage[T1]{fontenc}
\usepackage{amsmath,amssymb,amstext}
\usepackage{babel}



\usepackage{tgtermes}
\usepackage{tgheros}
\renewcommand{\ttdefault}{txtt}



\usepackage[Sonny]{fncychap}
\ChNameVar{\Large\normalfont\sffamily}
\ChTitleVar{\Large\normalfont\sffamily}
\usepackage{sphinx}

\fvset{fontsize=auto}
\usepackage{geometry}


% Include hyperref last.
\usepackage{hyperref}
% Fix anchor placement for figures with captions.
\usepackage{hypcap}% it must be loaded after hyperref.
% Set up styles of URL: it should be placed after hyperref.
\urlstyle{same}

\addto\captionsbrazil{\renewcommand{\contentsname}{Conteúdo:}}

\usepackage{sphinxmessages}
\setcounter{tocdepth}{2}


% Jupyter Notebook code cell colors
\definecolor{nbsphinxin}{HTML}{307FC1}
\definecolor{nbsphinxout}{HTML}{BF5B3D}
\definecolor{nbsphinx-code-bg}{HTML}{F5F5F5}
\definecolor{nbsphinx-code-border}{HTML}{E0E0E0}
\definecolor{nbsphinx-stderr}{HTML}{FFDDDD}
% ANSI colors for output streams and traceback highlighting
\definecolor{ansi-black}{HTML}{3E424D}
\definecolor{ansi-black-intense}{HTML}{282C36}
\definecolor{ansi-red}{HTML}{E75C58}
\definecolor{ansi-red-intense}{HTML}{B22B31}
\definecolor{ansi-green}{HTML}{00A250}
\definecolor{ansi-green-intense}{HTML}{007427}
\definecolor{ansi-yellow}{HTML}{DDB62B}
\definecolor{ansi-yellow-intense}{HTML}{B27D12}
\definecolor{ansi-blue}{HTML}{208FFB}
\definecolor{ansi-blue-intense}{HTML}{0065CA}
\definecolor{ansi-magenta}{HTML}{D160C4}
\definecolor{ansi-magenta-intense}{HTML}{A03196}
\definecolor{ansi-cyan}{HTML}{60C6C8}
\definecolor{ansi-cyan-intense}{HTML}{258F8F}
\definecolor{ansi-white}{HTML}{C5C1B4}
\definecolor{ansi-white-intense}{HTML}{A1A6B2}
\definecolor{ansi-default-inverse-fg}{HTML}{FFFFFF}
\definecolor{ansi-default-inverse-bg}{HTML}{000000}

% Define an environment for non-plain-text code cell outputs (e.g. images)
\makeatletter
\newenvironment{nbsphinxfancyoutput}{%
    % Avoid fatal error with framed.sty if graphics too long to fit on one page
    \let\sphinxincludegraphics\nbsphinxincludegraphics
    \nbsphinx@image@maxheight\textheight
    \advance\nbsphinx@image@maxheight -2\fboxsep   % default \fboxsep 3pt
    \advance\nbsphinx@image@maxheight -2\fboxrule  % default \fboxrule 0.4pt
    \advance\nbsphinx@image@maxheight -\baselineskip
\def\nbsphinxfcolorbox{\spx@fcolorbox{nbsphinx-code-border}{white}}%
\def\FrameCommand{\nbsphinxfcolorbox\nbsphinxfancyaddprompt\@empty}%
\def\FirstFrameCommand{\nbsphinxfcolorbox\nbsphinxfancyaddprompt\sphinxVerbatim@Continues}%
\def\MidFrameCommand{\nbsphinxfcolorbox\sphinxVerbatim@Continued\sphinxVerbatim@Continues}%
\def\LastFrameCommand{\nbsphinxfcolorbox\sphinxVerbatim@Continued\@empty}%
\MakeFramed{\advance\hsize-\width\@totalleftmargin\z@\linewidth\hsize\@setminipage}%
\lineskip=1ex\lineskiplimit=1ex\raggedright%
}{\par\unskip\@minipagefalse\endMakeFramed}
\makeatother
\newbox\nbsphinxpromptbox
\def\nbsphinxfancyaddprompt{\ifvoid\nbsphinxpromptbox\else
    \kern\fboxrule\kern\fboxsep
    \copy\nbsphinxpromptbox
    \kern-\ht\nbsphinxpromptbox\kern-\dp\nbsphinxpromptbox
    \kern-\fboxsep\kern-\fboxrule\nointerlineskip
    \fi}
\newlength\nbsphinxcodecellspacing
\setlength{\nbsphinxcodecellspacing}{0pt}

% Define support macros for attaching opening and closing lines to notebooks
\newsavebox\nbsphinxbox
\makeatletter
\newcommand{\nbsphinxstartnotebook}[1]{%
    \par
    % measure needed space
    \setbox\nbsphinxbox\vtop{{#1\par}}
    % reserve some space at bottom of page, else start new page
    \needspace{\dimexpr2.5\baselineskip+\ht\nbsphinxbox+\dp\nbsphinxbox}
    % mimic vertical spacing from \section command
      \addpenalty\@secpenalty
      \@tempskipa 3.5ex \@plus 1ex \@minus .2ex\relax
      \addvspace\@tempskipa
      {\Large\@tempskipa\baselineskip
             \advance\@tempskipa-\prevdepth
             \advance\@tempskipa-\ht\nbsphinxbox
             \ifdim\@tempskipa>\z@
               \vskip \@tempskipa
             \fi}
    \unvbox\nbsphinxbox
    % if notebook starts with a \section, prevent it from adding extra space
    \@nobreaktrue\everypar{\@nobreakfalse\everypar{}}%
    % compensate the parskip which will get inserted by next paragraph
    \nobreak\vskip-\parskip
    % do not break here
    \nobreak
}% end of \nbsphinxstartnotebook

\newcommand{\nbsphinxstopnotebook}[1]{%
    \par
    % measure needed space
    \setbox\nbsphinxbox\vbox{{#1\par}}
    \nobreak % it updates page totals
    \dimen@\pagegoal
    \advance\dimen@-\pagetotal \advance\dimen@-\pagedepth
    \advance\dimen@-\ht\nbsphinxbox \advance\dimen@-\dp\nbsphinxbox
    \ifdim\dimen@<\z@
      % little space left
      \unvbox\nbsphinxbox
      \kern-.8\baselineskip
      \nobreak\vskip\z@\@plus1fil
      \penalty100
      \vskip\z@\@plus-1fil
      \kern.8\baselineskip
    \else
      \unvbox\nbsphinxbox
    \fi
}% end of \nbsphinxstopnotebook

% Ensure height of an included graphics fits in nbsphinxfancyoutput frame
\newdimen\nbsphinx@image@maxheight % set in nbsphinxfancyoutput environment
\newcommand*{\nbsphinxincludegraphics}[2][]{%
    \gdef\spx@includegraphics@options{#1}%
    \setbox\spx@image@box\hbox{\includegraphics[#1,draft]{#2}}%
    \in@false
    \ifdim \wd\spx@image@box>\linewidth
      \g@addto@macro\spx@includegraphics@options{,width=\linewidth}%
      \in@true
    \fi
    % no rotation, no need to worry about depth
    \ifdim \ht\spx@image@box>\nbsphinx@image@maxheight
      \g@addto@macro\spx@includegraphics@options{,height=\nbsphinx@image@maxheight}%
      \in@true
    \fi
    \ifin@
      \g@addto@macro\spx@includegraphics@options{,keepaspectratio}%
    \fi
    \setbox\spx@image@box\box\voidb@x % clear memory
    \expandafter\includegraphics\expandafter[\spx@includegraphics@options]{#2}%
}% end of "\MakeFrame"-safe variant of \sphinxincludegraphics
\makeatother

\makeatletter
\renewcommand*\sphinx@verbatim@nolig@list{\do\'\do\`}
\begingroup
\catcode`'=\active
\let\nbsphinx@noligs\@noligs
\g@addto@macro\nbsphinx@noligs{\let'\PYGZsq}
\endgroup
\makeatother
\renewcommand*\sphinxbreaksbeforeactivelist{\do\<\do\"\do\'}
\renewcommand*\sphinxbreaksafteractivelist{\do\.\do\,\do\:\do\;\do\?\do\!\do\/\do\>\do\-}
\makeatletter
\fvset{codes*=\sphinxbreaksattexescapedchars\do\^\^\let\@noligs\nbsphinx@noligs}
\makeatother



\title{PDC\sphinxhyphen{}Vem Pro Centro}
\date{Oct 22, 2022}
\release{0.1}
\author{Geraldo Braz Júnior e Anselmo Cardoso de Paiva}
\newcommand{\sphinxlogo}{\vbox{}}
\renewcommand{\releasename}{Release}
\makeindex
\begin{document}

\ifdefined\shorthandoff
  \ifnum\catcode`\=\string=\active\shorthandoff{=}\fi
  \ifnum\catcode`\"=\active\shorthandoff{"}\fi
\fi

\pagestyle{empty}
\sphinxmaketitle
\pagestyle{plain}
\sphinxtableofcontents
\pagestyle{normal}
\phantomsection\label{\detokenize{index::doc}}


\sphinxstepscope


\chapter{Sobre}
\label{\detokenize{sobre:sobre}}\label{\detokenize{sobre::doc}}
\sphinxAtStartPar
O Banco Interamericano de Desenvolvimento (BID) investe continuamente na construção de cidades sustentáveis e inteligentes através da Divisão de Desenvolvimento e Habitação (HUD). As iniciativas incluem a utilização de tecnologias para análise de dados e assim proporcionar uma gestão inteligente da cidade, dentro do contexto de cidades inteligentes. Exemplos desses esforços são as Cooperação Técnicas (CT) regionais: (i) RG\sphinxhyphen{}T3083 “Metodologia para avaliação, identificação e implantação de projetos de Cidades Inteligentes para a América Latina e Caribe (ALC)” e a RG\sphinxhyphen{}T3095 “Big Data para o desenvolvimento urbano sustentável”. A primeira gerou como conclusão de que os dados coletados pelas administrações municipais ainda não estão sendo utilizados com todo o seu potencial para gerar soluções para os desafios urbanos. A segunda gerou uma série de resultados como mapeamento dos principais desafios urbanos nos municípios de análise (São Paulo (Brasil), Miraflores (Peru), Montevidéu (Uruguai), Quito (Equador) e Xalapa (México)), fortalecimento na qualificação profissional das equipes na análise de dados e finalmente o desenvolvimento e implementação de um plano de ação de uso dos dados para potencializar a análise de dados.

\sphinxAtStartPar
Com base nos resultados promissores obtidos nas cooperações citadas, o BID, com intuito de contribuir com as cidades brasileiras de Recife, São Luís e Vitória, decide apoiar ações de potencialização do uso de solução de Big Data na forma da Cooperação Técnica BR\sphinxhyphen{}T1496 “Potencializando o uso de Soluções de Big Data para Cidades Inteligentes”. Essa cooperação é composta por 2 consultorias interligadas. A primeira com o objetivo de Desenvolvimento de Diagnóstico dos Dados Municipais e a segunda, assunto deste plano de trabalho, consiste no Desenvolvimento de Prova de Conceito e Plano de Ação de Gestão de Dados.

\sphinxAtStartPar
A cidade de São Luís, com 1,1 milhão de habitantes, é a capital do estado do Maranhão. Está considerando uma segunda operação urbana com o Banco para migrar São Luís para um modelo de Cidade Inteligente, aproveitando soluções de dados para apoiar o planejamento urbano, projetos especiais e inovação da cidade.

\sphinxAtStartPar
As consultorias são totalmente interligadas uma vez que a primeira promove a coleta, análise e diagnóstico de dados abertos e restritos e a segunda consultoria deve orientar o processo de diagnóstico que deve incluir mecanismos de curadoria, qualidade, pré\sphinxhyphen{}processamento e limpeza de dados, para que estes possam ser adequadamente em provas de conceito a serem construídas como um dos objetivos da segunda consultoria.

\sphinxAtStartPar
A prefeitura de São Luís escolheu trabalhar as análises desta consultoria voltadas para dados da área de Saúde, com foco em epidemiologia, acidentes e distribuição de atendimento. Outro requisito importante a ser considerado foi que as provas de conceitos a serem desenvolvidas devem levar em consideração o fator de localização espacial dos dados.

\sphinxAtStartPar
Como forma de subsidiar o processo de diagnóstico dos dados fornecidos pela prefeitura e coletados em base de dados abertos, esta prova de conceito tem como finalidade apresentar uma solução para auxílio neste processo, especificamente no cenário do Programa Vem pro Centro.


\section{Programa Vem pro Centro}
\label{\detokenize{sobre:programa-vem-pro-centro}}
\sphinxAtStartPar
O Programa Vem pro Centro busca, por meio de ações de reabilitação, restauro e construção em imóveis ociosos, produzir unidades habitacionais, equipamentos públicos e pequenos comércios, que garantam o adensamento populacional, a conservação integrada, a sustentabilidade e a inclusão socioeconômica da população e dos usuários da região central de São Luís.

\sphinxstepscope


\chapter{Provas de Conceito}
\label{\detokenize{pdc:provas-de-conceito}}\label{\detokenize{pdc::doc}}
\sphinxAtStartPar
A solução está organizada em dois grandes módulos funcionais:
\begin{enumerate}
\sphinxsetlistlabels{\arabic}{enumi}{enumii}{}{.}%
\item {} 
\sphinxAtStartPar
Gerenciamento de dados: responsável pela funcionalidade de curadoria de dados

\item {} 
\sphinxAtStartPar
Visualização e Análise: responsável pela obtenção, processamento e visualização das análises sobre os dados

\end{enumerate}

\sphinxAtStartPar
A solução ainda está organizada de maneira a permitir interações tanto via interface web. Essa prerrogativa permite a fácil customização de serviços e ainda privilegia usuários avançados na análise de dados.

\sphinxAtStartPar
Os usuários conceituais definidos para as provas de conceito estão organizados em níveis de acesso. A autenticação e autorização deve ser realizada via banco de dados:
\begin{enumerate}
\sphinxsetlistlabels{\arabic}{enumi}{enumii}{}{.}%
\item {} 
\sphinxAtStartPar
Curador de dados: responsável pela manutenção de informações. Cada secretaria terá seu próprio responsável ou equipe para esta finalidade.

\item {} 
\sphinxAtStartPar
Analistas: usuários das diversas secretarias que desejam realizar análises sobre os dados.

\end{enumerate}

\sphinxAtStartPar
As funcionalidades previstas são apresentadas abaixo.
Em verde as funcionalidades obtidas diretamente. Em Azuis funcionalidades gerais de dados. Em amarelo funcionalidades para mapas.
Em vermelho funcionalidade para gráficos e em cinza, funcionalidade pública geral:

\noindent\sphinxincludegraphics[width=500\sphinxpxdimen]{{casos}.png}

\sphinxAtStartPar
Este diagrama apresenta que um Analista também é um usuário da prova de conceito.
Ele visualizará 2 funções principais no ambiente (em verde):
\begin{enumerate}
\sphinxsetlistlabels{\arabic}{enumi}{enumii}{}{.}%
\item {} 
\sphinxAtStartPar
Visualizar dados: engloba todos os mecanismos de visualização a serem construídos, começando pela Seleção de Fonte de Dados. Com os dados selecionados, este poderá aplicar filtragens, via Filtragem de Dados. Esses dados podem ser apresentados na forma de texto, gráficos (via Geração de Gráficos) e ainda mapas. Os mapas ainda poderão ser configurados ao Adicionar suporte espacial e Adicionar novas Camadas.

\item {} 
\sphinxAtStartPar
Obter Fontes de Dados: consiste em permitir que as fontes de dados sejam recuperadas

\end{enumerate}

\sphinxAtStartPar
O curador de dados ainda possui como responsabilidade a manutenção das informações atualizadas. Este acumula todas as funcionalidades apresentadas anteriormente.
Isso permite que ele teste as informações inseridas na base. As funcionalidades são:
\begin{itemize}
\item {} 
\sphinxAtStartPar
Manter Dados: inserir, remover, atualizar fontes de informações na base. As informações de suporte espacial serão inseridas diretamente via API disponibilizada. As informações de métricas ou indicadores serão inseridas via funcionalidade específica.

\end{itemize}

\sphinxAtStartPar
A partir das funcionalidades descritas, podemos organizar os módulos das PDCs no diagrama de componentes apresentado abaixo.
Além do módulos propostos, ainda são apresentados o Postgres como centralizador de dados e toda implementação sobre ambiente python.

\noindent\sphinxincludegraphics[width=500\sphinxpxdimen]{{componentes}.png}

\sphinxAtStartPar
O fluxo básico de funcionamento da solução é representado abaixo. Entende\sphinxhyphen{}se que pela sua simplificação, a prova de conceito
deseja apresentar a viabilidade de construção de soluções sobre análise de dados.
Ou seja, a construção da PDC foi focada na simplicidade e capacidade de implantação no ambiente real da prefeitura de São Luís,
cabendo a esta uma série de melhorias que poderiam acontecer de acordo com o requisito necessário.

\noindent\sphinxincludegraphics[width=500\sphinxpxdimen]{{func}.png}

\sphinxstepscope


\chapter{Como Usar}
\label{\detokenize{usage:como-usar}}\label{\detokenize{usage::doc}}

\section{Instalação}
\label{\detokenize{usage:instalacao}}
\sphinxAtStartPar
Para a instalação do módulo, recomenda\sphinxhyphen{}se o uso de Python 3.8 ou superior.
Recomenda\sphinxhyphen{}se o uso de um gerenciador de envs.

\begin{sphinxVerbatim}[commandchars=\\\{\}]
\PYG{g+gp}{\PYGZdl{} }pip install virtualenv
\PYG{g+gp}{\PYGZdl{} }virtualenv ENV\PYGZus{}NOME
\PYG{g+gp+gpVirtualEnv}{(.ENV\PYGZus{}NOME)} \PYG{g+gp}{\PYGZdl{} }\PYG{n+nb}{source} bin/activate
\end{sphinxVerbatim}

\sphinxAtStartPar
Em seguida, clonar o projeto no Github:

\begin{sphinxVerbatim}[commandchars=\\\{\}]
\PYG{g+gp+gpVirtualEnv}{(.ENV\PYGZus{}NOME)} \PYG{g+gp}{\PYGZdl{} }git clone https://github.com/gebraz/pdcbid2.git
\PYG{g+gp+gpVirtualEnv}{(.ENV\PYGZus{}NOME)} \PYG{g+gp}{\PYGZdl{} }\PYG{n+nb}{cd} pdcbid2
\end{sphinxVerbatim}

\sphinxAtStartPar
O arquivo requirements.txt possui todos as dependências necessárias e que devem ser
instaladas para o funcionamento da PDC.

\begin{sphinxVerbatim}[commandchars=\\\{\}]
\PYG{g+gp+gpVirtualEnv}{(.ENV\PYGZus{}NOME)} \PYG{g+gp}{\PYGZdl{} }pip install \PYGZhy{}r requirements.txt
\end{sphinxVerbatim}


\section{Configurar Globals.py}
\label{\detokenize{usage:configurar-globals-py}}
\sphinxAtStartPar
Para o funcionamento de ambas PDCs, é necessário configurar os parâmetros de
conexão com o banco de dados.
As configurações estão no arquivo \sphinxcode{\sphinxupquote{globals.py}} na pasta \sphinxcode{\sphinxupquote{apiModulo}}.
As seguintes informações devem ser preenchidas:

\begin{sphinxVerbatim}[commandchars=\\\{\}]
\PYG{n}{HOST} \PYG{o}{=} \PYG{l+s+s1}{\PYGZsq{}}\PYG{l+s+s1}{Endereço.Ip.Do.Banco}\PYG{l+s+s1}{\PYGZsq{}}
\PYG{n}{USER} \PYG{o}{=} \PYG{l+s+s1}{\PYGZsq{}}\PYG{l+s+s1}{\PYGZlt{}nome do usuário\PYGZgt{}}\PYG{l+s+s1}{\PYGZsq{}}
\PYG{n}{PASS} \PYG{o}{=} \PYG{l+s+s1}{\PYGZsq{}}\PYG{l+s+s1}{\PYGZlt{}senha do usuário\PYGZgt{}}\PYG{l+s+s1}{\PYGZsq{}}
\PYG{n}{DATABASE} \PYG{o}{=} \PYG{l+s+s1}{\PYGZsq{}}\PYG{l+s+s1}{\PYGZlt{}nome do database\PYGZgt{}}\PYG{l+s+s1}{\PYGZsq{}}
\end{sphinxVerbatim}


\section{Como usar PDC\sphinxhyphen{}API}
\label{\detokenize{usage:como-usar-pdc-api}}
\sphinxAtStartPar
Este módulo pode ser usado diretamente no python ou no Jupyter Notebook.
Para tanto, recomenda\sphinxhyphen{}se adicionar o path do módulo ao ambiente de execução.

\begin{sphinxVerbatim}[commandchars=\\\{\}]
import os
import sys

sys.path.append(“/path/to/dir/apiModulo”)

...
\end{sphinxVerbatim}


\section{Como executar PDC\sphinxhyphen{}Vis}
\label{\detokenize{usage:como-executar-pdc-vis}}
\sphinxAtStartPar
Após a configuração do módulo (veja {\hyperref[\detokenize{pdcvis:pdc-visualizacao}]{\sphinxcrossref{\DUrole{std,std-ref}{PDC\sphinxhyphen{}Visualização}}}}), estando na pasta do projeto, executar:

\begin{sphinxVerbatim}[commandchars=\\\{\}]
\PYG{g+gp+gpVirtualEnv}{(.ENV\PYGZus{}NOME)} \PYG{g+gp}{\PYGZdl{} }streamlit run app.py
\end{sphinxVerbatim}

\sphinxstepscope


\chapter{PDC\sphinxhyphen{}API}
\label{\detokenize{api:pdc-api}}\label{\detokenize{api::doc}}
\sphinxAtStartPar
Esta seção apresenta as funcionalidades construídas para o módulo de API da PDC.
Essas são organizadas em Consulta, Inserção e Visualização.

\begin{sphinxadmonition}{note}{Note:}
\sphinxAtStartPar
Obs.: Para o correto funcionamento deste módulo, é necessário seguir as configurações
do \sphinxcode{\sphinxupquote{globals.py}} apresentadas em {\hyperref[\detokenize{usage:configurar-globals-py}]{\sphinxcrossref{\DUrole{std,std-ref}{Configurar Globals.py}}}}.
\end{sphinxadmonition}

\begin{sphinxadmonition}{note}{Note:}
\sphinxAtStartPar
Obs.: O módulo não está instalado no classpath do python. Para se certificar do bom
funcionamento do mesmo, siga as instruções apresentadas em {\hyperref[\detokenize{usage:como-usar-pdc-api}]{\sphinxcrossref{\DUrole{std,std-ref}{Como usar PDC\sphinxhyphen{}API}}}}.
\end{sphinxadmonition}

\sphinxstepscope


\section{ApiConsulta}
\label{\detokenize{api_gen/apiModulo.api_consulta:apiconsulta}}\label{\detokenize{api_gen/apiModulo.api_consulta::doc}}\index{ApiConsulta (class in apiModulo.api\_consulta)@\spxentry{ApiConsulta}\spxextra{class in apiModulo.api\_consulta}}

\begin{fulllineitems}
\phantomsection\label{\detokenize{api_gen/apiModulo.api_consulta:apiModulo.api_consulta.ApiConsulta}}
\pysigstartsignatures
\pysiglinewithargsret{\sphinxbfcode{\sphinxupquote{class\DUrole{w}{  }}}\sphinxcode{\sphinxupquote{apiModulo.api\_consulta.}}\sphinxbfcode{\sphinxupquote{ApiConsulta}}}{\emph{\DUrole{n}{host}}, \emph{\DUrole{n}{user}}, \emph{\DUrole{n}{database}}, \emph{\DUrole{n}{p}}}{}
\pysigstopsignatures
\sphinxAtStartPar
Classe com as especificações de acesso e manipulação dos dados
\subsubsection*{Membros}


\begin{savenotes}\sphinxattablestart
\sphinxthistablewithglobalstyle
\sphinxthistablewithnovlinesstyle
\centering
\begin{tabulary}{\linewidth}[t]{\X{1}{2}\X{1}{2}}
\sphinxtoprule
\sphinxtableatstartofbodyhook
\sphinxAtStartPar
{\hyperref[\detokenize{api_gen/apiModulo.api_consulta:apiModulo.api_consulta.ApiConsulta.limparDados}]{\sphinxcrossref{\sphinxcode{\sphinxupquote{ApiConsulta.limparDados}}}}}(dados)
&
\sphinxAtStartPar
Remover caracteres em dados que deveriam ser numéricos
\\
\sphinxhline
\sphinxAtStartPar
{\hyperref[\detokenize{api_gen/apiModulo.api_consulta:apiModulo.api_consulta.ApiConsulta.lstCamadas}]{\sphinxcrossref{\sphinxcode{\sphinxupquote{ApiConsulta.lstCamadas}}}}}({[}filtro{]})
&
\sphinxAtStartPar
Obtem lista das camadas presentes na Base
\\
\sphinxhline
\sphinxAtStartPar
{\hyperref[\detokenize{api_gen/apiModulo.api_consulta:apiModulo.api_consulta.ApiConsulta.lstIndicador}]{\sphinxcrossref{\sphinxcode{\sphinxupquote{ApiConsulta.lstIndicador}}}}}({[}filtro, tema, assunto{]})
&
\sphinxAtStartPar
Obtem os Indicadores presentes na base, filtrando por qualquer termo, tema ou assunto
\\
\sphinxhline
\sphinxAtStartPar
{\hyperref[\detokenize{api_gen/apiModulo.api_consulta:apiModulo.api_consulta.ApiConsulta.lstTema}]{\sphinxcrossref{\sphinxcode{\sphinxupquote{ApiConsulta.lstTema}}}}}({[}filtro{]})
&
\sphinxAtStartPar
Obtem lista de temas cadastrados em Indicadores
\\
\sphinxhline
\sphinxAtStartPar
{\hyperref[\detokenize{api_gen/apiModulo.api_consulta:apiModulo.api_consulta.ApiConsulta.obterCamada}]{\sphinxcrossref{\sphinxcode{\sphinxupquote{ApiConsulta.obterCamada}}}}}({[}nome\_tabela, ...{]})
&
\sphinxAtStartPar
Obtém todos os dados de uma tabela do tipo camada (tem suporte espacial).
\\
\sphinxhline
\sphinxAtStartPar
{\hyperref[\detokenize{api_gen/apiModulo.api_consulta:apiModulo.api_consulta.ApiConsulta.obterIndicador}]{\sphinxcrossref{\sphinxcode{\sphinxupquote{ApiConsulta.obterIndicador}}}}}({[}index, tabela, ...{]})
&
\sphinxAtStartPar
Obtém dados de um Indicador
\\
\sphinxhline
\sphinxAtStartPar
{\hyperref[\detokenize{api_gen/apiModulo.api_consulta:apiModulo.api_consulta.ApiConsulta.obterMetadados}]{\sphinxcrossref{\sphinxcode{\sphinxupquote{ApiConsulta.obterMetadados}}}}}({[}tema, indexes{]})
&
\sphinxAtStartPar
Obtem as descrições dos indicadores, contendo definição e camada
\\
\sphinxhline
\sphinxAtStartPar
{\hyperref[\detokenize{api_gen/apiModulo.api_consulta:apiModulo.api_consulta.ApiConsulta.obterTabela}]{\sphinxcrossref{\sphinxcode{\sphinxupquote{ApiConsulta.obterTabela}}}}}({[}nome\_tabela{]})
&
\sphinxAtStartPar
Obtém todos os dados de uma tabela.
\\
\sphinxhline
\sphinxAtStartPar
{\hyperref[\detokenize{api_gen/apiModulo.api_consulta:apiModulo.api_consulta.ApiConsulta.obterTema}]{\sphinxcrossref{\sphinxcode{\sphinxupquote{ApiConsulta.obterTema}}}}}({[}tema, indexes{]})
&
\sphinxAtStartPar
Obtem os dados na forma de um pandas/geopandas.
\\
\sphinxhline
\sphinxAtStartPar
{\hyperref[\detokenize{api_gen/apiModulo.api_consulta:apiModulo.api_consulta.ApiConsulta.obterTemaSoma}]{\sphinxcrossref{\sphinxcode{\sphinxupquote{ApiConsulta.obterTemaSoma}}}}}({[}tema, indexes{]})
&
\sphinxAtStartPar
Obtém os dados dos indicadores e soma seus resultados.
\\
\sphinxbottomrule
\end{tabulary}
\sphinxtableafterendhook\par
\sphinxattableend\end{savenotes}
\index{limparDados() (apiModulo.api\_consulta.ApiConsulta method)@\spxentry{limparDados()}\spxextra{apiModulo.api\_consulta.ApiConsulta method}}

\begin{fulllineitems}
\phantomsection\label{\detokenize{api_gen/apiModulo.api_consulta:apiModulo.api_consulta.ApiConsulta.limparDados}}
\pysigstartsignatures
\pysiglinewithargsret{\sphinxbfcode{\sphinxupquote{limparDados}}}{\emph{\DUrole{n}{dados}}}{}
\pysigstopsignatures
\sphinxAtStartPar
Remover caracteres em dados que deveriam ser numéricos
\begin{quote}\begin{description}
\sphinxlineitem{Parameters}
\sphinxAtStartPar
\sphinxstyleliteralstrong{\sphinxupquote{dados}} \textendash{} dataframe/geodataframe com os dados

\sphinxlineitem{Returns}
\sphinxAtStartPar
Dados pós\sphinxhyphen{}processados

\sphinxlineitem{Return type}
\sphinxAtStartPar
geopandas

\end{description}\end{quote}

\end{fulllineitems}

\index{lstCamadas() (apiModulo.api\_consulta.ApiConsulta method)@\spxentry{lstCamadas()}\spxextra{apiModulo.api\_consulta.ApiConsulta method}}

\begin{fulllineitems}
\phantomsection\label{\detokenize{api_gen/apiModulo.api_consulta:apiModulo.api_consulta.ApiConsulta.lstCamadas}}
\pysigstartsignatures
\pysiglinewithargsret{\sphinxbfcode{\sphinxupquote{lstCamadas}}}{\emph{\DUrole{n}{filtro}\DUrole{o}{=}\DUrole{default_value}{\textquotesingle{}\textquotesingle{}}}}{}
\pysigstopsignatures
\sphinxAtStartPar
Obtem lista das camadas presentes na Base
\begin{quote}\begin{description}
\sphinxlineitem{Parameters}
\sphinxAtStartPar
\sphinxstyleliteralstrong{\sphinxupquote{filtro}} \textendash{} filtro de nome

\sphinxlineitem{Returns}
\sphinxAtStartPar
Lista de camadas presentes na base

\sphinxlineitem{Return type}
\sphinxAtStartPar
dataframe \sphinxhyphen{} pandas

\end{description}\end{quote}

\end{fulllineitems}

\index{lstIndicador() (apiModulo.api\_consulta.ApiConsulta method)@\spxentry{lstIndicador()}\spxextra{apiModulo.api\_consulta.ApiConsulta method}}

\begin{fulllineitems}
\phantomsection\label{\detokenize{api_gen/apiModulo.api_consulta:apiModulo.api_consulta.ApiConsulta.lstIndicador}}
\pysigstartsignatures
\pysiglinewithargsret{\sphinxbfcode{\sphinxupquote{lstIndicador}}}{\emph{\DUrole{n}{filtro}\DUrole{o}{=}\DUrole{default_value}{\textquotesingle{}\textquotesingle{}}}, \emph{\DUrole{n}{tema}\DUrole{o}{=}\DUrole{default_value}{\textquotesingle{}\textquotesingle{}}}, \emph{\DUrole{n}{assunto}\DUrole{o}{=}\DUrole{default_value}{\textquotesingle{}\textquotesingle{}}}}{}
\pysigstopsignatures
\sphinxAtStartPar
Obtem os Indicadores presentes na base, filtrando por qualquer termo, tema ou assunto
\begin{quote}\begin{description}
\sphinxlineitem{Parameters}\begin{itemize}
\item {} 
\sphinxAtStartPar
\sphinxstyleliteralstrong{\sphinxupquote{filtro}} \textendash{} campo de texto que deve ser utilizado como filtro nos campos de tema, assunto e descrição

\item {} 
\sphinxAtStartPar
\sphinxstyleliteralstrong{\sphinxupquote{tema}} \textendash{} filtra especificamente por tema

\item {} 
\sphinxAtStartPar
\sphinxstyleliteralstrong{\sphinxupquote{assunto}} \textendash{} filtra especificamente por assunto

\end{itemize}

\sphinxlineitem{Returns}
\sphinxAtStartPar
Resultados organizados como tema, assunto, descrição, fonte, tabela, definição, tabela

\sphinxlineitem{Return type}
\sphinxAtStartPar
dataframe \sphinxhyphen{} pandas

\end{description}\end{quote}

\end{fulllineitems}

\index{lstTema() (apiModulo.api\_consulta.ApiConsulta method)@\spxentry{lstTema()}\spxextra{apiModulo.api\_consulta.ApiConsulta method}}

\begin{fulllineitems}
\phantomsection\label{\detokenize{api_gen/apiModulo.api_consulta:apiModulo.api_consulta.ApiConsulta.lstTema}}
\pysigstartsignatures
\pysiglinewithargsret{\sphinxbfcode{\sphinxupquote{lstTema}}}{\emph{\DUrole{n}{filtro}\DUrole{o}{=}\DUrole{default_value}{\textquotesingle{}\textquotesingle{}}}}{}
\pysigstopsignatures
\sphinxAtStartPar
Obtem lista de temas cadastrados em Indicadores
\begin{quote}\begin{description}
\sphinxlineitem{Parameters}
\sphinxAtStartPar
\sphinxstyleliteralstrong{\sphinxupquote{filtro}} \textendash{} filtro sobre o campo tema

\sphinxlineitem{Returns}
\sphinxAtStartPar
Lista de temas presentes na base

\sphinxlineitem{Return type}
\sphinxAtStartPar
dataframe \sphinxhyphen{} pandas

\end{description}\end{quote}

\end{fulllineitems}

\index{obterCamada() (apiModulo.api\_consulta.ApiConsulta method)@\spxentry{obterCamada()}\spxextra{apiModulo.api\_consulta.ApiConsulta method}}

\begin{fulllineitems}
\phantomsection\label{\detokenize{api_gen/apiModulo.api_consulta:apiModulo.api_consulta.ApiConsulta.obterCamada}}
\pysigstartsignatures
\pysiglinewithargsret{\sphinxbfcode{\sphinxupquote{obterCamada}}}{\emph{\DUrole{n}{nome\_tabela}\DUrole{o}{=}\DUrole{default_value}{\textquotesingle{}\textquotesingle{}}}, \emph{\DUrole{n}{simples}\DUrole{o}{=}\DUrole{default_value}{False}}, \emph{\DUrole{n}{campo}\DUrole{o}{=}\DUrole{default_value}{\textquotesingle{}\textquotesingle{}}}, \emph{\DUrole{n}{filtro}\DUrole{o}{=}\DUrole{default_value}{\textquotesingle{}\textquotesingle{}}}}{}
\pysigstopsignatures
\sphinxAtStartPar
Obtém todos os dados de uma tabela do tipo camada (tem suporte espacial). 
Não deve ser usado para dados de indicadores.
\begin{quote}\begin{description}
\sphinxlineitem{Parameters}\begin{itemize}
\item {} 
\sphinxAtStartPar
\sphinxstyleliteralstrong{\sphinxupquote{nome\_tabela}} \textendash{} nome da tabela no banco

\item {} 
\sphinxAtStartPar
\sphinxstyleliteralstrong{\sphinxupquote{simples}} \textendash{} quando True retorna apenas cod\_camada e geometria

\item {} 
\sphinxAtStartPar
\sphinxstyleliteralstrong{\sphinxupquote{campo}} \textendash{} quando filtro != ‘’, especifica qual campo adicional da tabela deve ser obtido

\item {} 
\sphinxAtStartPar
\sphinxstyleliteralstrong{\sphinxupquote{filtro}} \textendash{} usado na cláusula where

\end{itemize}

\sphinxlineitem{Returns}
\sphinxAtStartPar
Dados, na forma de geopandas

\sphinxlineitem{Return type}
\sphinxAtStartPar
geopandas

\end{description}\end{quote}

\end{fulllineitems}

\index{obterIndicador() (apiModulo.api\_consulta.ApiConsulta method)@\spxentry{obterIndicador()}\spxextra{apiModulo.api\_consulta.ApiConsulta method}}

\begin{fulllineitems}
\phantomsection\label{\detokenize{api_gen/apiModulo.api_consulta:apiModulo.api_consulta.ApiConsulta.obterIndicador}}
\pysigstartsignatures
\pysiglinewithargsret{\sphinxbfcode{\sphinxupquote{obterIndicador}}}{\emph{\DUrole{n}{index}\DUrole{o}{=}\DUrole{default_value}{\sphinxhyphen{}1}}, \emph{\DUrole{n}{tabela}\DUrole{o}{=}\DUrole{default_value}{\textquotesingle{}\textquotesingle{}}}, \emph{\DUrole{n}{definicao}\DUrole{o}{=}\DUrole{default_value}{\textquotesingle{}\textquotesingle{}}}}{}
\pysigstopsignatures
\sphinxAtStartPar
Obtém dados de um Indicador
\begin{quote}\begin{description}
\sphinxlineitem{Parameters}\begin{itemize}
\item {} 
\sphinxAtStartPar
\sphinxstyleliteralstrong{\sphinxupquote{index}} \textendash{} ID do indicador na base

\item {} 
\sphinxAtStartPar
\sphinxstyleliteralstrong{\sphinxupquote{tabela}} \textendash{} Nome da tabela com dados no banco (presente no Indicador)

\item {} 
\sphinxAtStartPar
\sphinxstyleliteralstrong{\sphinxupquote{definicao}} \textendash{} Nome da coluna com dados no banco (presente no Indicador)

\end{itemize}

\sphinxlineitem{Returns}
\sphinxAtStartPar
Dados

\sphinxlineitem{Return type}
\sphinxAtStartPar
pandas/geopandas

\end{description}\end{quote}

\end{fulllineitems}

\index{obterMetadados() (apiModulo.api\_consulta.ApiConsulta method)@\spxentry{obterMetadados()}\spxextra{apiModulo.api\_consulta.ApiConsulta method}}

\begin{fulllineitems}
\phantomsection\label{\detokenize{api_gen/apiModulo.api_consulta:apiModulo.api_consulta.ApiConsulta.obterMetadados}}
\pysigstartsignatures
\pysiglinewithargsret{\sphinxbfcode{\sphinxupquote{obterMetadados}}}{\emph{\DUrole{n}{tema}\DUrole{o}{=}\DUrole{default_value}{\textquotesingle{}\textquotesingle{}}}, \emph{\DUrole{n}{indexes}\DUrole{o}{=}\DUrole{default_value}{\textquotesingle{}\textquotesingle{}}}}{}
\pysigstopsignatures
\sphinxAtStartPar
Obtem as descrições dos indicadores, contendo definição e camada
\begin{quote}\begin{description}
\sphinxlineitem{Parameters}\begin{itemize}
\item {} 
\sphinxAtStartPar
\sphinxstyleliteralstrong{\sphinxupquote{filtro}} \textendash{} filtro de nome

\item {} 
\sphinxAtStartPar
\sphinxstyleliteralstrong{\sphinxupquote{indexes}} \textendash{} lista de ids de Indicadores no formato de string, Ex. {[}1, 2, 3, 4{]}

\end{itemize}

\sphinxlineitem{Returns}
\sphinxAtStartPar
Dicionário com o metadados da base

\sphinxlineitem{Return type}
\sphinxAtStartPar
dict

\end{description}\end{quote}

\end{fulllineitems}

\index{obterTabela() (apiModulo.api\_consulta.ApiConsulta method)@\spxentry{obterTabela()}\spxextra{apiModulo.api\_consulta.ApiConsulta method}}

\begin{fulllineitems}
\phantomsection\label{\detokenize{api_gen/apiModulo.api_consulta:apiModulo.api_consulta.ApiConsulta.obterTabela}}
\pysigstartsignatures
\pysiglinewithargsret{\sphinxbfcode{\sphinxupquote{obterTabela}}}{\emph{\DUrole{n}{nome\_tabela}\DUrole{o}{=}\DUrole{default_value}{\textquotesingle{}\textquotesingle{}}}}{}
\pysigstopsignatures
\sphinxAtStartPar
Obtém todos os dados de uma tabela. 
Não deve ser usado para camadas.
\begin{quote}\begin{description}
\sphinxlineitem{Parameters}
\sphinxAtStartPar
\sphinxstyleliteralstrong{\sphinxupquote{nome\_tabela}} \textendash{} nome da tabela no banco

\sphinxlineitem{Returns}
\sphinxAtStartPar
Dados, na forma de pandas

\sphinxlineitem{Return type}
\sphinxAtStartPar
pandas

\end{description}\end{quote}

\end{fulllineitems}

\index{obterTema() (apiModulo.api\_consulta.ApiConsulta method)@\spxentry{obterTema()}\spxextra{apiModulo.api\_consulta.ApiConsulta method}}

\begin{fulllineitems}
\phantomsection\label{\detokenize{api_gen/apiModulo.api_consulta:apiModulo.api_consulta.ApiConsulta.obterTema}}
\pysigstartsignatures
\pysiglinewithargsret{\sphinxbfcode{\sphinxupquote{obterTema}}}{\emph{\DUrole{n}{tema}\DUrole{o}{=}\DUrole{default_value}{\textquotesingle{}\textquotesingle{}}}, \emph{\DUrole{n}{indexes}\DUrole{o}{=}\DUrole{default_value}{\textquotesingle{}\textquotesingle{}}}}{}
\pysigstopsignatures
\sphinxAtStartPar
Obtem os dados na forma de um pandas/geopandas. Os Indicadores precisam compartilhar o mesmo suporte espacial
\begin{quote}\begin{description}
\sphinxlineitem{Parameters}\begin{itemize}
\item {} 
\sphinxAtStartPar
\sphinxstyleliteralstrong{\sphinxupquote{tema}} \textendash{} utilizado para filtrar os campos de tema, assunto e descrição

\item {} 
\sphinxAtStartPar
\sphinxstyleliteralstrong{\sphinxupquote{indexes}} \textendash{} lista de ids de Indicadores no formato de string, Ex. {[}1, 2, 3, 4{]}

\end{itemize}

\sphinxlineitem{Returns}
\sphinxAtStartPar
Dados obtidos no banco. Se houver campo espacial, geopandas. Se não, um pandas. Os metadados são também devolvidos.

\sphinxlineitem{Return type}
\sphinxAtStartPar
geopandas, dict

\end{description}\end{quote}

\end{fulllineitems}

\index{obterTemaSoma() (apiModulo.api\_consulta.ApiConsulta method)@\spxentry{obterTemaSoma()}\spxextra{apiModulo.api\_consulta.ApiConsulta method}}

\begin{fulllineitems}
\phantomsection\label{\detokenize{api_gen/apiModulo.api_consulta:apiModulo.api_consulta.ApiConsulta.obterTemaSoma}}
\pysigstartsignatures
\pysiglinewithargsret{\sphinxbfcode{\sphinxupquote{obterTemaSoma}}}{\emph{\DUrole{n}{tema}\DUrole{o}{=}\DUrole{default_value}{\textquotesingle{}\textquotesingle{}}}, \emph{\DUrole{n}{indexes}\DUrole{o}{=}\DUrole{default_value}{\textquotesingle{}\textquotesingle{}}}}{}
\pysigstopsignatures
\sphinxAtStartPar
Obtém os dados dos indicadores e soma seus resultados.
Garanta que todos os indicadores são do tipo numérico.
\begin{quote}\begin{description}
\sphinxlineitem{Parameters}\begin{itemize}
\item {} 
\sphinxAtStartPar
\sphinxstyleliteralstrong{\sphinxupquote{tema}} \textendash{} utilizado para filtrar os campos de tema, assunto e descrição

\item {} 
\sphinxAtStartPar
\sphinxstyleliteralstrong{\sphinxupquote{indexes}} \textendash{} lista de ids de Indicadores no formato de string, Ex. {[}1, 2, 3, 4{]}

\end{itemize}

\sphinxlineitem{Returns}
\sphinxAtStartPar
Dados obtidos no banco. Se houver campo espacial, geopandas. Se não, um pandas. Os metadados são também devolvidos.

\sphinxlineitem{Return type}
\sphinxAtStartPar
geopandas, dict

\end{description}\end{quote}

\end{fulllineitems}


\end{fulllineitems}


\sphinxstepscope


\section{ApiIns}
\label{\detokenize{api_gen/apiModulo.api_insercao:apiins}}\label{\detokenize{api_gen/apiModulo.api_insercao::doc}}\index{ApiIns (class in apiModulo.api\_insercao)@\spxentry{ApiIns}\spxextra{class in apiModulo.api\_insercao}}

\begin{fulllineitems}
\phantomsection\label{\detokenize{api_gen/apiModulo.api_insercao:apiModulo.api_insercao.ApiIns}}
\pysigstartsignatures
\pysiglinewithargsret{\sphinxbfcode{\sphinxupquote{class\DUrole{w}{  }}}\sphinxcode{\sphinxupquote{apiModulo.api\_insercao.}}\sphinxbfcode{\sphinxupquote{ApiIns}}}{\emph{\DUrole{n}{host}}, \emph{\DUrole{n}{user}}, \emph{\DUrole{n}{database}}, \emph{\DUrole{n}{p}}}{}
\pysigstopsignatures
\sphinxAtStartPar
Classe com as especificações para criação de indicadores e inserção de informação
\subsubsection*{Membros}


\begin{savenotes}\sphinxattablestart
\sphinxthistablewithglobalstyle
\sphinxthistablewithnovlinesstyle
\centering
\begin{tabulary}{\linewidth}[t]{\X{1}{2}\X{1}{2}}
\sphinxtoprule
\sphinxtableatstartofbodyhook
\sphinxAtStartPar
{\hyperref[\detokenize{api_gen/apiModulo.api_insercao:apiModulo.api_insercao.ApiIns.inserirDados}]{\sphinxcrossref{\sphinxcode{\sphinxupquote{ApiIns.inserirDados}}}}}(data, nome\_tabela{[}, ...{]})
&
\sphinxAtStartPar
Insere dados no banco de dados, baseado numa entrada e especificação de camada quando por o caso
\\
\sphinxhline
\sphinxAtStartPar
{\hyperref[\detokenize{api_gen/apiModulo.api_insercao:apiModulo.api_insercao.ApiIns.inserirCamada}]{\sphinxcrossref{\sphinxcode{\sphinxupquote{ApiIns.inserirCamada}}}}}(dado, tabela{[}, tipo, ...{]})
&
\sphinxAtStartPar
Insere dados geografico no banco de dados, geopandas ou string para o shapefile
\\
\sphinxhline
\sphinxAtStartPar
{\hyperref[\detokenize{api_gen/apiModulo.api_insercao:apiModulo.api_insercao.ApiIns.inserirIndicador}]{\sphinxcrossref{\sphinxcode{\sphinxupquote{ApiIns.inserirIndicador}}}}}(tema, assunto, ...)
&
\sphinxAtStartPar
Inserir indicador no banco de dados
\\
\sphinxhline
\sphinxAtStartPar
{\hyperref[\detokenize{api_gen/apiModulo.api_insercao:apiModulo.api_insercao.ApiIns.removerIndicador}]{\sphinxcrossref{\sphinxcode{\sphinxupquote{ApiIns.removerIndicador}}}}}(id)
&
\sphinxAtStartPar
Remove indicador no banco de dados
\\
\sphinxbottomrule
\end{tabulary}
\sphinxtableafterendhook\par
\sphinxattableend\end{savenotes}
\index{inserirCamada() (apiModulo.api\_insercao.ApiIns method)@\spxentry{inserirCamada()}\spxextra{apiModulo.api\_insercao.ApiIns method}}

\begin{fulllineitems}
\phantomsection\label{\detokenize{api_gen/apiModulo.api_insercao:apiModulo.api_insercao.ApiIns.inserirCamada}}
\pysigstartsignatures
\pysiglinewithargsret{\sphinxbfcode{\sphinxupquote{inserirCamada}}}{\emph{\DUrole{n}{dado}}, \emph{\DUrole{n}{tabela}}, \emph{\DUrole{n}{tipo}\DUrole{o}{=}\DUrole{default_value}{\textquotesingle{}MULTIPOLYGON\textquotesingle{}}}, \emph{\DUrole{n}{campo\_chave}\DUrole{o}{=}\DUrole{default_value}{None}}, \emph{\DUrole{n}{nome}\DUrole{o}{=}\DUrole{default_value}{\textquotesingle{}\textquotesingle{}}}, \emph{\DUrole{n}{descricao}\DUrole{o}{=}\DUrole{default_value}{\textquotesingle{}\textquotesingle{}}}}{}
\pysigstopsignatures
\sphinxAtStartPar
Insere dados geografico no banco de dados, geopandas ou string para o shapefile
\begin{quote}\begin{description}
\sphinxlineitem{Parameters}\begin{itemize}
\item {} 
\sphinxAtStartPar
\sphinxstyleliteralstrong{\sphinxupquote{dado}} \textendash{} Path do arquivo(csv) ou objeto geopandas

\item {} 
\sphinxAtStartPar
\sphinxstyleliteralstrong{\sphinxupquote{tabela}} \textendash{} Nome da tabela a ser criada na base de dados

\item {} 
\sphinxAtStartPar
\sphinxstyleliteralstrong{\sphinxupquote{tipo}} \textendash{} Tipo de dado geo, MULTIPOLYGON, POINT, LINE

\item {} 
\sphinxAtStartPar
\sphinxstyleliteralstrong{\sphinxupquote{campo\_chave}} \textendash{} nome da coluna no geopandas que representa a chave dos dados

\item {} 
\sphinxAtStartPar
\sphinxstyleliteralstrong{\sphinxupquote{nome}} \textendash{} nome dado a camada, para referência nos Indicadores

\item {} 
\sphinxAtStartPar
\sphinxstyleliteralstrong{\sphinxupquote{delimiter}} \textendash{} Delimitador de coluna se carregar de csv. Padrão ‘;’

\end{itemize}

\end{description}\end{quote}

\end{fulllineitems}

\index{inserirDados() (apiModulo.api\_insercao.ApiIns method)@\spxentry{inserirDados()}\spxextra{apiModulo.api\_insercao.ApiIns method}}

\begin{fulllineitems}
\phantomsection\label{\detokenize{api_gen/apiModulo.api_insercao:apiModulo.api_insercao.ApiIns.inserirDados}}
\pysigstartsignatures
\pysiglinewithargsret{\sphinxbfcode{\sphinxupquote{inserirDados}}}{\emph{\DUrole{n}{data}}, \emph{\DUrole{n}{nome\_tabela}}, \emph{\DUrole{n}{indice}\DUrole{o}{=}\DUrole{default_value}{\textquotesingle{}\textquotesingle{}}}, \emph{\DUrole{n}{camada}\DUrole{o}{=}\DUrole{default_value}{\textquotesingle{}\textquotesingle{}}}, \emph{\DUrole{n}{campo\_camada}\DUrole{o}{=}\DUrole{default_value}{\textquotesingle{}\textquotesingle{}}}, \emph{\DUrole{n}{delimiter}\DUrole{o}{=}\DUrole{default_value}{\textquotesingle{};\textquotesingle{}}}}{}
\pysigstopsignatures
\sphinxAtStartPar
Insere dados no banco de dados, baseado numa entrada e especificação de camada quando por o caso
\begin{quote}\begin{description}
\sphinxlineitem{Parameters}\begin{itemize}
\item {} 
\sphinxAtStartPar
\sphinxstyleliteralstrong{\sphinxupquote{data}} \textendash{} Path do arquivo(csv) ou objeto pandas

\item {} 
\sphinxAtStartPar
\sphinxstyleliteralstrong{\sphinxupquote{indice}} \textendash{} Coluna a ser usada como chave primária. Quando não informado usa o index sequencial do pandas

\item {} 
\sphinxAtStartPar
\sphinxstyleliteralstrong{\sphinxupquote{nome\_tabela}} \textendash{} Nome da tabela destino no banco de dados

\item {} 
\sphinxAtStartPar
\sphinxstyleliteralstrong{\sphinxupquote{delimiter}} \textendash{} Delimitador de coluna se carregar de csv. Padrão ‘;’

\end{itemize}

\end{description}\end{quote}

\end{fulllineitems}

\index{inserirIndicador() (apiModulo.api\_insercao.ApiIns method)@\spxentry{inserirIndicador()}\spxextra{apiModulo.api\_insercao.ApiIns method}}

\begin{fulllineitems}
\phantomsection\label{\detokenize{api_gen/apiModulo.api_insercao:apiModulo.api_insercao.ApiIns.inserirIndicador}}
\pysigstartsignatures
\pysiglinewithargsret{\sphinxbfcode{\sphinxupquote{inserirIndicador}}}{\emph{\DUrole{n}{tema}}, \emph{\DUrole{n}{assunto}}, \emph{\DUrole{n}{tabela}}, \emph{\DUrole{n}{definicao}}, \emph{\DUrole{n}{descricao}}, \emph{\DUrole{n}{fonte}}, \emph{\DUrole{n}{ano}}, \emph{\DUrole{n}{camada}\DUrole{o}{=}\DUrole{default_value}{1}}, \emph{\DUrole{n}{id\_indicador}\DUrole{o}{=}\DUrole{default_value}{None}}}{}
\pysigstopsignatures
\sphinxAtStartPar
Inserir indicador no banco de dados
\begin{quote}\begin{description}
\sphinxlineitem{Parameters}\begin{itemize}
\item {} 
\sphinxAtStartPar
\sphinxstyleliteralstrong{\sphinxupquote{tema}} \textendash{} Tema do indicador(ex: Responsável, Domicílio, Pessoa)

\item {} 
\sphinxAtStartPar
\sphinxstyleliteralstrong{\sphinxupquote{assunto}} \textendash{} Assuntos do indicador, separados por vírgula(ex:Cor ou Raça, idade e gênero)

\item {} 
\sphinxAtStartPar
\sphinxstyleliteralstrong{\sphinxupquote{tabela}} \textendash{} Tabela que o indicador está apontando

\item {} 
\sphinxAtStartPar
\sphinxstyleliteralstrong{\sphinxupquote{definicao}} \textendash{} Coluna da tabela que o indicador está apontando

\item {} 
\sphinxAtStartPar
\sphinxstyleliteralstrong{\sphinxupquote{descricao}} \textendash{} Descrição do indicador que aparece no mapa

\item {} 
\sphinxAtStartPar
\sphinxstyleliteralstrong{\sphinxupquote{fonte}} \textendash{} De onde o indicador foi retirado(ex: IBGE Censo Demográfico 2010)

\item {} 
\sphinxAtStartPar
\sphinxstyleliteralstrong{\sphinxupquote{ano}} \textendash{} Ano da fonte do indicador

\item {} 
\sphinxAtStartPar
\sphinxstyleliteralstrong{\sphinxupquote{camada}} \textendash{} Número da camada que o indicador está apontando. Camada 1 por padrão

\item {} 
\sphinxAtStartPar
\sphinxstyleliteralstrong{\sphinxupquote{id\_indicador}} \textendash{} Código do ID do indicador. Padrão None.

\end{itemize}

\end{description}\end{quote}

\end{fulllineitems}

\index{removerIndicador() (apiModulo.api\_insercao.ApiIns method)@\spxentry{removerIndicador()}\spxextra{apiModulo.api\_insercao.ApiIns method}}

\begin{fulllineitems}
\phantomsection\label{\detokenize{api_gen/apiModulo.api_insercao:apiModulo.api_insercao.ApiIns.removerIndicador}}
\pysigstartsignatures
\pysiglinewithargsret{\sphinxbfcode{\sphinxupquote{removerIndicador}}}{\emph{\DUrole{n}{id}}}{}
\pysigstopsignatures
\sphinxAtStartPar
Remove indicador no banco de dados
\begin{quote}\begin{description}
\sphinxlineitem{Parameters}
\sphinxAtStartPar
\sphinxstyleliteralstrong{\sphinxupquote{id}} \textendash{} Código do ID do indicador.

\end{description}\end{quote}

\end{fulllineitems}


\end{fulllineitems}


\sphinxstepscope


\section{ApiVis}
\label{\detokenize{api_gen/apiModulo.api_visualizacao:apivis}}\label{\detokenize{api_gen/apiModulo.api_visualizacao::doc}}\index{ApiVis (class in apiModulo.api\_visualizacao)@\spxentry{ApiVis}\spxextra{class in apiModulo.api\_visualizacao}}

\begin{fulllineitems}
\phantomsection\label{\detokenize{api_gen/apiModulo.api_visualizacao:apiModulo.api_visualizacao.ApiVis}}
\pysigstartsignatures
\pysiglinewithargsret{\sphinxbfcode{\sphinxupquote{class\DUrole{w}{  }}}\sphinxcode{\sphinxupquote{apiModulo.api\_visualizacao.}}\sphinxbfcode{\sphinxupquote{ApiVis}}}{\emph{\DUrole{n}{host}}, \emph{\DUrole{n}{user}}, \emph{\DUrole{n}{database}}, \emph{\DUrole{n}{p}}}{}
\pysigstopsignatures
\sphinxAtStartPar
Classe com métodos auxiliares para visualização no Jupyter
\subsubsection*{Membros}


\begin{savenotes}\sphinxattablestart
\sphinxthistablewithglobalstyle
\sphinxthistablewithnovlinesstyle
\centering
\begin{tabulary}{\linewidth}[t]{\X{1}{2}\X{1}{2}}
\sphinxtoprule
\sphinxtableatstartofbodyhook
\sphinxAtStartPar
{\hyperref[\detokenize{api_gen/apiModulo.api_visualizacao:apiModulo.api_visualizacao.ApiVis.visMapaIndicador}]{\sphinxcrossref{\sphinxcode{\sphinxupquote{ApiVis.visMapaIndicador}}}}}(indicador{[}, height, ...{]})
&
\sphinxAtStartPar
Constroi visualização em mapa de um indicador específico Obrigatório que o indicador tenha suporte espacial
\\
\sphinxhline
\sphinxAtStartPar
{\hyperref[\detokenize{api_gen/apiModulo.api_visualizacao:apiModulo.api_visualizacao.ApiVis.visMapaGJson}]{\sphinxcrossref{\sphinxcode{\sphinxupquote{ApiVis.visMapaGJson}}}}}(gdf, variavel, descricao)
&
\sphinxAtStartPar
Constroi visualização em mapa de um valor específico presente em um geopandas
\\
\sphinxbottomrule
\end{tabulary}
\sphinxtableafterendhook\par
\sphinxattableend\end{savenotes}
\index{visMapaDados() (apiModulo.api\_visualizacao.ApiVis method)@\spxentry{visMapaDados()}\spxextra{apiModulo.api\_visualizacao.ApiVis method}}

\begin{fulllineitems}
\phantomsection\label{\detokenize{api_gen/apiModulo.api_visualizacao:apiModulo.api_visualizacao.ApiVis.visMapaDados}}
\pysigstartsignatures
\pysiglinewithargsret{\sphinxbfcode{\sphinxupquote{visMapaDados}}}{\emph{\DUrole{n}{df}}, \emph{\DUrole{n}{metadados}}, \emph{\DUrole{n}{height}\DUrole{o}{=}\DUrole{default_value}{1200}}, \emph{\DUrole{n}{width}\DUrole{o}{=}\DUrole{default_value}{600}}, \emph{\DUrole{n}{MAPA\_ZOOM}\DUrole{o}{=}\DUrole{default_value}{14.5}}}{}
\pysigstopsignatures
\sphinxAtStartPar
Constroi visualização em mapa de um indicador específico
Obrigatório que o indicador tenha suporte espacial
\begin{quote}\begin{description}
\sphinxlineitem{Parameters}\begin{itemize}
\item {} 
\sphinxAtStartPar
\sphinxstyleliteralstrong{\sphinxupquote{df}} \textendash{} GeoDataframe carregado com os dados

\item {} 
\sphinxAtStartPar
\sphinxstyleliteralstrong{\sphinxupquote{meta}} \textendash{} metadados das variáveis presentes no geodataframe

\item {} 
\sphinxAtStartPar
\sphinxstyleliteralstrong{\sphinxupquote{height}} \textendash{} Altura de cada mapa. Por padrão 1200

\item {} 
\sphinxAtStartPar
\sphinxstyleliteralstrong{\sphinxupquote{width}} \textendash{} Largura de cada mapa. Por padrão 600

\item {} 
\sphinxAtStartPar
\sphinxstyleliteralstrong{\sphinxupquote{MAPA\_ZOOM}} \textendash{} Zoom inicial do mapa. Por padrão 14.5

\end{itemize}

\end{description}\end{quote}

\end{fulllineitems}

\index{visMapaGJson() (apiModulo.api\_visualizacao.ApiVis method)@\spxentry{visMapaGJson()}\spxextra{apiModulo.api\_visualizacao.ApiVis method}}

\begin{fulllineitems}
\phantomsection\label{\detokenize{api_gen/apiModulo.api_visualizacao:apiModulo.api_visualizacao.ApiVis.visMapaGJson}}
\pysigstartsignatures
\pysiglinewithargsret{\sphinxbfcode{\sphinxupquote{visMapaGJson}}}{\emph{\DUrole{n}{gdf}}, \emph{\DUrole{n}{variavel}}, \emph{\DUrole{n}{descricao}}, \emph{\DUrole{n}{height}\DUrole{o}{=}\DUrole{default_value}{1200}}, \emph{\DUrole{n}{width}\DUrole{o}{=}\DUrole{default_value}{600}}, \emph{\DUrole{n}{MAPA\_ZOOM}\DUrole{o}{=}\DUrole{default_value}{14.5}}}{}
\pysigstopsignatures
\sphinxAtStartPar
Constroi visualização em mapa de um valor específico presente em um geopandas
\begin{quote}\begin{description}
\sphinxlineitem{Parameters}\begin{itemize}
\item {} 
\sphinxAtStartPar
\sphinxstyleliteralstrong{\sphinxupquote{gdf}} \textendash{} geopandas com os dados

\item {} 
\sphinxAtStartPar
\sphinxstyleliteralstrong{\sphinxupquote{variavel}} \textendash{} coluna do geopandas a ser visualizado

\item {} 
\sphinxAtStartPar
\sphinxstyleliteralstrong{\sphinxupquote{descricao}} \textendash{} breve descrição dos dados, a servir como legenda

\item {} 
\sphinxAtStartPar
\sphinxstyleliteralstrong{\sphinxupquote{height}} \textendash{} Altura de cada mapa. Por padrão 1200

\item {} 
\sphinxAtStartPar
\sphinxstyleliteralstrong{\sphinxupquote{width}} \textendash{} Largura de cada mapa. Por padrão 600

\item {} 
\sphinxAtStartPar
\sphinxstyleliteralstrong{\sphinxupquote{MAPA\_ZOOM}} \textendash{} Zoom inicial do mapa. Por padrão 14.5

\end{itemize}

\end{description}\end{quote}

\end{fulllineitems}

\index{visMapaIndicador() (apiModulo.api\_visualizacao.ApiVis method)@\spxentry{visMapaIndicador()}\spxextra{apiModulo.api\_visualizacao.ApiVis method}}

\begin{fulllineitems}
\phantomsection\label{\detokenize{api_gen/apiModulo.api_visualizacao:apiModulo.api_visualizacao.ApiVis.visMapaIndicador}}
\pysigstartsignatures
\pysiglinewithargsret{\sphinxbfcode{\sphinxupquote{visMapaIndicador}}}{\emph{\DUrole{n}{indicador}}, \emph{\DUrole{n}{height}\DUrole{o}{=}\DUrole{default_value}{1200}}, \emph{\DUrole{n}{width}\DUrole{o}{=}\DUrole{default_value}{600}}, \emph{\DUrole{n}{MAPA\_ZOOM}\DUrole{o}{=}\DUrole{default_value}{14.5}}}{}
\pysigstopsignatures
\sphinxAtStartPar
Constroi visualização em mapa de um indicador específico
Obrigatório que o indicador tenha suporte espacial
\begin{quote}\begin{description}
\sphinxlineitem{Parameters}\begin{itemize}
\item {} 
\sphinxAtStartPar
\sphinxstyleliteralstrong{\sphinxupquote{indicador}} \textendash{} id do indicador no banco. Precisa consultar usando o método lstIndicador \sphinxhyphen{} lista separada por vírugula

\item {} 
\sphinxAtStartPar
\sphinxstyleliteralstrong{\sphinxupquote{height}} \textendash{} Altura de cada mapa. Por padrão 1200

\item {} 
\sphinxAtStartPar
\sphinxstyleliteralstrong{\sphinxupquote{width}} \textendash{} Largura de cada mapa. Por padrão 600

\item {} 
\sphinxAtStartPar
\sphinxstyleliteralstrong{\sphinxupquote{MAPA\_ZOOM}} \textendash{} Zoom inicial do mapa. Por padrão 14.5

\end{itemize}

\end{description}\end{quote}

\end{fulllineitems}

\index{visMapaSemIndicador() (apiModulo.api\_visualizacao.ApiVis method)@\spxentry{visMapaSemIndicador()}\spxextra{apiModulo.api\_visualizacao.ApiVis method}}

\begin{fulllineitems}
\phantomsection\label{\detokenize{api_gen/apiModulo.api_visualizacao:apiModulo.api_visualizacao.ApiVis.visMapaSemIndicador}}
\pysigstartsignatures
\pysiglinewithargsret{\sphinxbfcode{\sphinxupquote{visMapaSemIndicador}}}{\emph{\DUrole{n}{dados}\DUrole{o}{=}\DUrole{default_value}{None}}, \emph{\DUrole{n}{coluna\_geom}\DUrole{o}{=}\DUrole{default_value}{None}}}{}
\pysigstopsignatures
\sphinxAtStartPar
”
Constroi visualização em mapa sem nenhum indicador

\end{fulllineitems}

\index{visMultiMapa() (apiModulo.api\_visualizacao.ApiVis method)@\spxentry{visMultiMapa()}\spxextra{apiModulo.api\_visualizacao.ApiVis method}}

\begin{fulllineitems}
\phantomsection\label{\detokenize{api_gen/apiModulo.api_visualizacao:apiModulo.api_visualizacao.ApiVis.visMultiMapa}}
\pysigstartsignatures
\pysiglinewithargsret{\sphinxbfcode{\sphinxupquote{visMultiMapa}}}{\emph{\DUrole{n}{map}\DUrole{o}{=}\DUrole{default_value}{None}}, \emph{\DUrole{n}{tipo}\DUrole{o}{=}\DUrole{default_value}{None}}, \emph{\DUrole{n}{dado}\DUrole{o}{=}\DUrole{default_value}{None}}, \emph{\DUrole{n}{variavel}\DUrole{o}{=}\DUrole{default_value}{None}}, \emph{\DUrole{n}{alias}\DUrole{o}{=}\DUrole{default_value}{None}}, \emph{\DUrole{n}{height}\DUrole{o}{=}\DUrole{default_value}{1200}}, \emph{\DUrole{n}{width}\DUrole{o}{=}\DUrole{default_value}{600}}, \emph{\DUrole{n}{MAPA\_ZOOM}\DUrole{o}{=}\DUrole{default_value}{14.5}}}{}
\pysigstopsignatures
\end{fulllineitems}


\end{fulllineitems}


\sphinxstepscope


\chapter{PDC\sphinxhyphen{}Visualização}
\label{\detokenize{pdcvis:pdc-visualizacao}}\label{\detokenize{pdcvis::doc}}
\sphinxAtStartPar
Esta seção apresenta as funcionalidades construídas para o módulo de Visualização
construída nesse PDC. Esse módulo faz uso da API para a construção de visualizações de mapa e gráficos
via Streamlit.


\section{Definição do \sphinxstyleliteralintitle{\sphinxupquote{config.yaml}}}
\label{\detokenize{pdcvis:definicao-do-config-yaml}}
\sphinxAtStartPar
Para o funcionamento do módulo, é necessário configura o arquivo \sphinxcode{\sphinxupquote{config.yaml}}.
Esse arquivo é organizado em blocos explicados a seguir:

\begin{sphinxadmonition}{note}{Note:}
\sphinxAtStartPar
O arquivo \sphinxcode{\sphinxupquote{config.yaml}} se encontra na raiz do projeto
\end{sphinxadmonition}
\begin{itemize}
\item {} 
\sphinxAtStartPar
Bloco base, \sphinxcode{\sphinxupquote{app}} deve sempre estar no início do arquivo.

\item {} \begin{description}
\sphinxlineitem{Bloco \sphinxcode{\sphinxupquote{area\_shape\_basico}}: representa camadas do mapa que poderão ser selecionados para apresentação}\begin{itemize}
\item {} 
\sphinxAtStartPar
info: nome indicativo para a camada

\item {} 
\sphinxAtStartPar
tabela: origem dos dados na base de dados

\item {} 
\sphinxAtStartPar
tipo: tipo de geometria dos dados

\end{itemize}

\end{description}

\end{itemize}

\sphinxAtStartPar
Exemplo:

\begin{sphinxVerbatim}[commandchars=\\\{\}]
\PYG{p+pIndicator}{\PYGZhy{}}\PYG{+w}{ }\PYG{n+nt}{area\PYGZus{}shape\PYGZus{}basico}\PYG{p}{:}
\PYG{+w}{      }\PYG{n+nt}{info}\PYG{p}{:}\PYG{+w}{ }\PYG{l+s}{\PYGZsq{}}\PYG{l+s}{Limites}\PYG{n+nv}{ }\PYG{l+s}{São}\PYG{n+nv}{ }\PYG{l+s}{Luís}\PYG{l+s}{\PYGZsq{}}
\PYG{+w}{      }\PYG{n+nt}{tabela}\PYG{p}{:}\PYG{+w}{ }\PYG{l+lScalar+lScalarPlain}{limites\PYGZus{}sao\PYGZus{}luis}
\PYG{+w}{      }\PYG{n+nt}{tipo}\PYG{p}{:}\PYG{+w}{ }\PYG{l+lScalar+lScalarPlain}{poligono}
\end{sphinxVerbatim}
\begin{itemize}
\item {} \begin{description}
\sphinxlineitem{Bloco \sphinxcode{\sphinxupquote{camada\_padrao}}: representa informações a serem adicionadas como filtros nos mapas}\begin{itemize}
\item {} 
\sphinxAtStartPar
tipo: tipo de geometria dos dados

\item {} 
\sphinxAtStartPar
info: nome indicativo para a camada

\item {} 
\sphinxAtStartPar
tabela: origem dos dados na base de dados

\item {} 
\sphinxAtStartPar
campos: lista de campos separada por \sphinxcode{\sphinxupquote{\#}} que serão consumidos da tabela

\item {} 
\sphinxAtStartPar
descricao: nome formatado dos campos informados, separados por \sphinxcode{\sphinxupquote{\#}}

\item {} 
\sphinxAtStartPar
popup: campos a serem inseridos no popup, separados por \sphinxcode{\sphinxupquote{\#}}

\item {} 
\sphinxAtStartPar
desc\_popup: nome formatado dos campos informados no popup, separados por \sphinxcode{\sphinxupquote{\#}}

\end{itemize}

\end{description}

\end{itemize}

\begin{sphinxVerbatim}[commandchars=\\\{\}]
\PYG{p+pIndicator}{\PYGZhy{}}\PYG{+w}{ }\PYG{n+nt}{camada\PYGZus{}padrao}\PYG{p}{:}
\PYG{+w}{      }\PYG{n+nt}{tipo}\PYG{p}{:}\PYG{+w}{ }\PYG{l+lScalar+lScalarPlain}{ponto}
\PYG{+w}{      }\PYG{n+nt}{info}\PYG{p}{:}\PYG{+w}{ }\PYG{l+s}{\PYGZsq{}}\PYG{l+s}{Equipamentos}\PYG{n+nv}{ }\PYG{l+s}{Público}\PYG{l+s}{\PYGZsq{}}
\PYG{+w}{      }\PYG{n+nt}{tabela}\PYG{p}{:}\PYG{+w}{ }\PYG{l+lScalar+lScalarPlain}{equipamento\PYGZus{}publico}
\PYG{+w}{      }\PYG{n+nt}{campos}\PYG{p}{:}\PYG{+w}{ }\PYG{l+lScalar+lScalarPlain}{educacao\PYGZsh{}saude\PYGZsh{}seguranca\PYGZsh{}cultural\PYGZsh{}assistencia\PYGZus{}social}
\PYG{+w}{      }\PYG{n+nt}{descricao}\PYG{p}{:}\PYG{+w}{ }\PYG{l+s}{\PYGZsq{}}\PYG{l+s}{Educação\PYGZsh{}Saúde\PYGZsh{}Segurança\PYGZsh{}Cultural\PYGZsh{}Assistência}\PYG{n+nv}{ }\PYG{l+s}{Social}\PYG{l+s}{\PYGZsq{}}
\PYG{+w}{      }\PYG{n+nt}{popup}\PYG{p}{:}\PYG{+w}{ }\PYG{l+lScalar+lScalarPlain}{tipo\PYGZsh{}name}
\PYG{+w}{      }\PYG{n+nt}{desc\PYGZus{}popup}\PYG{p}{:}\PYG{+w}{ }\PYG{l+s}{\PYGZsq{}}\PYG{l+s}{Tipo\PYGZsh{}Nome}\PYG{l+s}{\PYGZsq{}}
\end{sphinxVerbatim}

\sphinxAtStartPar
Os conteúdos (ou páginas) estão organizados em \sphinxcode{\sphinxupquote{topico}}. Cada um pode possuir \sphinxcode{\sphinxupquote{mapa}} e \sphinxcode{\sphinxupquote{grafico}}
de acordo com a necessidade de visualização.

\sphinxAtStartPar
Um tópico é definido por um \sphinxcode{\sphinxupquote{título}} e uma \sphinxcode{\sphinxupquote{descrição}}. O conteúdo dentro é feito pelas
declarações de mapas e gráficos.

\begin{sphinxVerbatim}[commandchars=\\\{\}]
\PYG{p+pIndicator}{\PYGZhy{}}\PYG{+w}{ }\PYG{n+nt}{topico}\PYG{p}{:}
\PYG{+w}{      }\PYG{n+nt}{titulo}\PYG{p}{:}\PYG{+w}{ }\PYG{l+s}{\PYGZsq{}}\PYG{l+s}{Restauro}\PYG{n+nv}{ }\PYG{l+s}{para}\PYG{n+nv}{ }\PYG{l+s}{Habitação}\PYG{l+s}{\PYGZsq{}}
\PYG{+w}{      }\PYG{n+nt}{descricao}\PYG{p}{:}\PYG{+w}{ }\PYG{l+s}{\PYGZsq{}}\PYG{l+s}{\PYGZsq{}}

\PYG{+w}{      }\PYG{n+nt}{mapa1}\PYG{p}{:}
\PYG{+w}{      }\PYG{n+nt}{grafico1}\PYG{p}{:}
\PYG{+w}{      }\PYG{n+nt}{mapa2}\PYG{p}{:}
\PYG{+w}{      }\PYG{n+nt}{grafico2}\PYG{p}{:}
\end{sphinxVerbatim}

\begin{sphinxadmonition}{note}{Note:}
\sphinxAtStartPar
Você pode inserir quantos mapas forem necessários, mas precisam ser numerados
em seguência: 1, 2, 3
\end{sphinxadmonition}
\begin{description}
\sphinxlineitem{Um mapa é definido por um título, uma descrição, e:}\begin{itemize}
\item {} 
\sphinxAtStartPar
tabela: tabela com os dados

\item {} 
\sphinxAtStartPar
camada: referência espacial na base

\item {} 
\sphinxAtStartPar
variavel: colunas da tabela, separadas por \sphinxcode{\sphinxupquote{\#}}

\item {} 
\sphinxAtStartPar
alias: nome formatado das colunas, separado por \sphinxcode{\sphinxupquote{\#}}

\item {} 
\sphinxAtStartPar
camada\_extra: referência a uma camada padrão declarada anteriormente

\item {} 
\sphinxAtStartPar
camada\_interna: referência a uma camada padrão declarada anteriomente

\item {} 
\sphinxAtStartPar
camada\_base: referência a uma camada padrão declarada anteriomente

\end{itemize}

\end{description}

\begin{sphinxVerbatim}[commandchars=\\\{\}]
\PYG{n+nt}{mapa1}\PYG{p}{:}
\PYG{+w}{    }\PYG{n+nt}{titulo}\PYG{p}{:}\PYG{+w}{ }\PYG{l+s}{\PYGZsq{}}\PYG{l+s}{**Distribuição}\PYG{n+nv}{ }\PYG{l+s}{de}\PYG{n+nv}{ }\PYG{l+s}{Mulheres}\PYG{n+nv}{ }\PYG{l+s}{por}\PYG{n+nv}{ }\PYG{l+s}{Setor}\PYG{n+nv}{ }\PYG{l+s}{censitario**}\PYG{l+s}{\PYGZsq{}}
\PYG{+w}{    }\PYG{n+nt}{descricao}\PYG{p}{:}\PYG{+w}{ }\PYG{l+s}{\PYGZsq{}}\PYG{l+s}{\PYGZsq{}}
\PYG{+w}{    }\PYG{n+nt}{tabela}\PYG{p}{:}\PYG{+w}{ }\PYG{l+lScalar+lScalarPlain}{pdc\PYGZus{}bid\PYGZus{}ibge}
\PYG{+w}{    }\PYG{n+nt}{camada}\PYG{p}{:}\PYG{+w}{ }\PYG{l+lScalar+lScalarPlain}{setor\PYGZus{}censitario}
\PYG{+w}{    }\PYG{n+nt}{variavel}\PYG{p}{:}\PYG{+w}{ }\PYG{l+s}{\PYGZsq{}}\PYG{l+s}{mulheres\PYGZus{}14\PYGZus{}anos\PYGZus{}ou\PYGZus{}mais\PYGZsh{}mulheres\PYGZus{}resp\PYGZus{}pelo\PYGZus{}dom}\PYG{l+s}{\PYGZsq{}}
\PYG{+w}{    }\PYG{n+nt}{alias}\PYG{p}{:}\PYG{+w}{ }\PYG{l+s}{\PYGZsq{}}\PYG{l+s}{Mulheres}\PYG{n+nv}{ }\PYG{l+s}{(14}\PYG{n+nv}{ }\PYG{l+s}{anos}\PYG{n+nv}{ }\PYG{l+s}{ou}\PYG{n+nv}{ }\PYG{l+s}{mais):\PYGZsh{}Mulheres}\PYG{n+nv}{ }\PYG{l+s}{responsáveis}\PYG{n+nv}{ }\PYG{l+s}{pelo}\PYG{n+nv}{ }\PYG{l+s}{domicílio}\PYG{l+s}{\PYGZsq{}}
\PYG{+w}{    }\PYG{n+nt}{camada\PYGZus{}extra}\PYG{p}{:}\PYG{+w}{ }\PYG{l+s}{\PYGZsq{}}\PYG{l+s}{lotes\PYGZus{}por\PYGZus{}ponto}\PYG{l+s}{\PYGZsq{}}
\PYG{+w}{    }\PYG{n+nt}{camada\PYGZus{}interna}\PYG{p}{:}\PYG{+w}{ }\PYG{l+lScalar+lScalarPlain}{tombamento}
\PYG{+w}{    }\PYG{n+nt}{camada\PYGZus{}base}\PYG{p}{:}\PYG{+w}{ }\PYG{l+lScalar+lScalarPlain}{anel\PYGZus{}viario}
\end{sphinxVerbatim}
\begin{description}
\sphinxlineitem{Um mapa é definido por um título, uma descrição, e:}\begin{itemize}
\item {} 
\sphinxAtStartPar
tabela: tabela com os dados

\item {} 
\sphinxAtStartPar
x: lista de colunas utilizadas das tabelas

\item {} 
\sphinxAtStartPar
x\_alias: nomes formatados das colunas

\item {} 
\sphinxAtStartPar
y: referencial do valor apresentado no eixo y

\item {} 
\sphinxAtStartPar
tipo: barra\_vertical \sphinxhyphen{} barra\_horizontal \sphinxhyphen{} pizza \sphinxhyphen{} scatter \sphinxhyphen{} linha

\end{itemize}

\end{description}

\begin{sphinxVerbatim}[commandchars=\\\{\}]
\PYG{n+nt}{grafico1}\PYG{p}{:}
\PYG{+w}{    }\PYG{n+nt}{titulo}\PYG{p}{:}\PYG{+w}{ }\PYG{l+s}{\PYGZsq{}}\PYG{l+s}{**Distribuição}\PYG{n+nv}{ }\PYG{l+s}{da}\PYG{n+nv}{ }\PYG{l+s}{população**}\PYG{l+s}{\PYGZsq{}}
\PYG{+w}{    }\PYG{n+nt}{descricao}\PYG{p}{:}\PYG{+w}{ }\PYG{l+s}{\PYGZsq{}}\PYG{l+s}{\PYGZsq{}}
\PYG{+w}{    }\PYG{n+nt}{tabela}\PYG{p}{:}\PYG{+w}{ }\PYG{l+lScalar+lScalarPlain}{pdc\PYGZus{}bid\PYGZus{}ibge}
\PYG{+w}{    }\PYG{n+nt}{x}\PYG{p}{:}
\PYG{+w}{        }\PYG{p+pIndicator}{\PYGZhy{}}\PYG{+w}{ }\PYG{l+lScalar+lScalarPlain}{mulheres\PYGZus{}14\PYGZus{}anos\PYGZus{}ou\PYGZus{}mais}
\PYG{+w}{        }\PYG{p+pIndicator}{\PYGZhy{}}\PYG{+w}{ }\PYG{l+lScalar+lScalarPlain}{mulher\PYGZus{}responsavel}
\PYG{+w}{        }\PYG{p+pIndicator}{\PYGZhy{}}\PYG{+w}{ }\PYG{l+lScalar+lScalarPlain}{idosos\PYGZus{}65\PYGZus{}anos\PYGZus{}ou\PYGZus{}mais}
\PYG{+w}{        }\PYG{p+pIndicator}{\PYGZhy{}}\PYG{+w}{ }\PYG{l+lScalar+lScalarPlain}{crianca\PYGZus{}13\PYGZus{}ou\PYGZus{}menos}
\PYG{+w}{    }\PYG{n+nt}{x\PYGZus{}alias}\PYG{p}{:}
\PYG{+w}{        }\PYG{p+pIndicator}{\PYGZhy{}}\PYG{+w}{ }\PYG{l+lScalar+lScalarPlain}{Mulheres}\PYG{l+lScalar+lScalarPlain}{ }\PYG{l+lScalar+lScalarPlain}{de}\PYG{l+lScalar+lScalarPlain}{ }\PYG{l+lScalar+lScalarPlain}{14}\PYG{l+lScalar+lScalarPlain}{ }\PYG{l+lScalar+lScalarPlain}{anos}\PYG{l+lScalar+lScalarPlain}{ }\PYG{l+lScalar+lScalarPlain}{ou}\PYG{l+lScalar+lScalarPlain}{ }\PYG{l+lScalar+lScalarPlain}{mais}
\PYG{+w}{        }\PYG{p+pIndicator}{\PYGZhy{}}\PYG{+w}{ }\PYG{l+lScalar+lScalarPlain}{Mulher}\PYG{l+lScalar+lScalarPlain}{ }\PYG{l+lScalar+lScalarPlain}{Responsável}
\PYG{+w}{        }\PYG{p+pIndicator}{\PYGZhy{}}\PYG{+w}{ }\PYG{l+lScalar+lScalarPlain}{Idosos}\PYG{l+lScalar+lScalarPlain}{ }\PYG{l+lScalar+lScalarPlain}{de}\PYG{l+lScalar+lScalarPlain}{ }\PYG{l+lScalar+lScalarPlain}{65}\PYG{l+lScalar+lScalarPlain}{ }\PYG{l+lScalar+lScalarPlain}{anos}\PYG{l+lScalar+lScalarPlain}{ }\PYG{l+lScalar+lScalarPlain}{ou}\PYG{l+lScalar+lScalarPlain}{ }\PYG{l+lScalar+lScalarPlain}{mais}
\PYG{+w}{        }\PYG{p+pIndicator}{\PYGZhy{}}\PYG{+w}{ }\PYG{l+lScalar+lScalarPlain}{Crianças}\PYG{l+lScalar+lScalarPlain}{ }\PYG{l+lScalar+lScalarPlain}{de}\PYG{l+lScalar+lScalarPlain}{ }\PYG{l+lScalar+lScalarPlain}{13}\PYG{l+lScalar+lScalarPlain}{ }\PYG{l+lScalar+lScalarPlain}{anos}\PYG{l+lScalar+lScalarPlain}{ }\PYG{l+lScalar+lScalarPlain}{ou}\PYG{l+lScalar+lScalarPlain}{ }\PYG{l+lScalar+lScalarPlain}{menos}
\PYG{+w}{    }\PYG{n+nt}{y}\PYG{p}{:}
\PYG{+w}{        }\PYG{p+pIndicator}{\PYGZhy{}}\PYG{+w}{ }\PYG{l+lScalar+lScalarPlain}{Quantidade}
\PYG{+w}{    }\PYG{n+nt}{tipo}\PYG{p}{:}\PYG{+w}{ }\PYG{l+lScalar+lScalarPlain}{barra\PYGZus{}vertical}
\end{sphinxVerbatim}


\section{Exemplo do arquivo \sphinxstyleliteralintitle{\sphinxupquote{config.yaml}}}
\label{\detokenize{pdcvis:exemplo-do-arquivo-config-yaml}}
\begin{sphinxVerbatim}[commandchars=\\\{\}]
\PYG{n+nt}{app}\PYG{p}{:}
\PYG{+w}{ }\PYG{p+pIndicator}{\PYGZhy{}}\PYG{+w}{ }\PYG{n+nt}{area\PYGZus{}shape\PYGZus{}basico}\PYG{p}{:}
\PYG{+w}{      }\PYG{n+nt}{info}\PYG{p}{:}\PYG{+w}{ }\PYG{l+s}{\PYGZsq{}}\PYG{l+s}{Limites}\PYG{n+nv}{ }\PYG{l+s}{São}\PYG{n+nv}{ }\PYG{l+s}{Luís}\PYG{l+s}{\PYGZsq{}}
\PYG{+w}{      }\PYG{n+nt}{tabela}\PYG{p}{:}\PYG{+w}{ }\PYG{l+lScalar+lScalarPlain}{limites\PYGZus{}sao\PYGZus{}luis}
\PYG{+w}{      }\PYG{n+nt}{tipo}\PYG{p}{:}\PYG{+w}{ }\PYG{l+lScalar+lScalarPlain}{poligono}
\PYG{+w}{ }\PYG{p+pIndicator}{\PYGZhy{}}\PYG{+w}{ }\PYG{n+nt}{area\PYGZus{}shape\PYGZus{}basico}\PYG{p}{:}
\PYG{+w}{      }\PYG{n+nt}{info}\PYG{p}{:}\PYG{+w}{ }\PYG{l+s}{\PYGZsq{}}\PYG{l+s}{Anél}\PYG{n+nv}{ }\PYG{l+s}{Viário}\PYG{l+s}{\PYGZsq{}}
\PYG{+w}{      }\PYG{n+nt}{tabela}\PYG{p}{:}\PYG{+w}{ }\PYG{l+lScalar+lScalarPlain}{anel\PYGZus{}viario}
\PYG{+w}{      }\PYG{n+nt}{tipo}\PYG{p}{:}\PYG{+w}{ }\PYG{l+lScalar+lScalarPlain}{poligono}
\PYG{+w}{ }\PYG{p+pIndicator}{\PYGZhy{}}\PYG{+w}{ }\PYG{n+nt}{camada\PYGZus{}padrao}\PYG{p}{:}
\PYG{+w}{      }\PYG{n+nt}{tipo}\PYG{p}{:}\PYG{+w}{ }\PYG{l+lScalar+lScalarPlain}{ponto}
\PYG{+w}{      }\PYG{n+nt}{info}\PYG{p}{:}\PYG{+w}{ }\PYG{l+s}{\PYGZsq{}}\PYG{l+s}{Informações}\PYG{n+nv}{ }\PYG{l+s}{de}\PYG{n+nv}{ }\PYG{l+s}{lotes}\PYG{l+s}{\PYGZsq{}}
\PYG{+w}{      }\PYG{n+nt}{tabela}\PYG{p}{:}\PYG{+w}{ }\PYG{l+lScalar+lScalarPlain}{lotes\PYGZus{}por\PYGZus{}ponto}
\PYG{+w}{      }\PYG{n+nt}{campos}\PYG{p}{:}\PYG{+w}{ }\PYG{l+lScalar+lScalarPlain}{finais\PYGZus{}prior\PYGZsh{}finais\PYGZus{}divida\PYGZsh{}finais\PYGZus{}estado\PYGZus{}de\PYGZus{}}
\PYG{+w}{      }\PYG{n+nt}{descricao}\PYG{p}{:}\PYG{+w}{ }\PYG{l+s}{\PYGZsq{}}\PYG{l+s}{Prioridade\PYGZsh{}Divida\PYGZsh{}Estado}\PYG{n+nv}{ }\PYG{l+s}{de}\PYG{n+nv}{ }\PYG{l+s}{Conservação}\PYG{l+s}{\PYGZsq{}}
\PYG{+w}{      }\PYG{n+nt}{popup}\PYG{p}{:}\PYG{+w}{ }\PYG{l+lScalar+lScalarPlain}{finais\PYGZus{}de\PYGZus{}uso\PYGZsh{}finais\PYGZus{}prior\PYGZsh{}finais\PYGZus{}divida\PYGZsh{}finais\PYGZus{}estado\PYGZus{}de\PYGZus{}}
\PYG{+w}{      }\PYG{n+nt}{desc\PYGZus{}popup}\PYG{p}{:}\PYG{+w}{ }\PYG{l+s}{\PYGZsq{}}\PYG{l+s}{Uso\PYGZsh{}Prioridade\PYGZsh{}Divida\PYGZsh{}Estado}\PYG{n+nv}{ }\PYG{l+s}{de}\PYG{n+nv}{ }\PYG{l+s}{Conservação}\PYG{l+s}{\PYGZsq{}}
\PYG{+w}{ }\PYG{p+pIndicator}{\PYGZhy{}}\PYG{+w}{ }\PYG{n+nt}{camada\PYGZus{}padrao}\PYG{p}{:}
\PYG{+w}{      }\PYG{n+nt}{tipo}\PYG{p}{:}\PYG{+w}{ }\PYG{l+lScalar+lScalarPlain}{ponto}
\PYG{+w}{      }\PYG{n+nt}{info}\PYG{p}{:}\PYG{+w}{ }\PYG{l+s}{\PYGZsq{}}\PYG{l+s}{Equipamentos}\PYG{n+nv}{ }\PYG{l+s}{Público}\PYG{l+s}{\PYGZsq{}}
\PYG{+w}{      }\PYG{n+nt}{tabela}\PYG{p}{:}\PYG{+w}{ }\PYG{l+lScalar+lScalarPlain}{equipamento\PYGZus{}publico}
\PYG{+w}{      }\PYG{n+nt}{campos}\PYG{p}{:}\PYG{+w}{ }\PYG{l+lScalar+lScalarPlain}{educacao\PYGZsh{}saude\PYGZsh{}seguranca\PYGZsh{}cultural\PYGZsh{}assistencia\PYGZus{}social}
\PYG{+w}{      }\PYG{n+nt}{descricao}\PYG{p}{:}\PYG{+w}{ }\PYG{l+s}{\PYGZsq{}}\PYG{l+s}{Educação\PYGZsh{}Saúde\PYGZsh{}Segurança\PYGZsh{}Cultural\PYGZsh{}Assistência}\PYG{n+nv}{ }\PYG{l+s}{Social}\PYG{l+s}{\PYGZsq{}}
\PYG{+w}{      }\PYG{n+nt}{popup}\PYG{p}{:}\PYG{+w}{ }\PYG{l+lScalar+lScalarPlain}{tipo\PYGZsh{}name}
\PYG{+w}{      }\PYG{n+nt}{desc\PYGZus{}popup}\PYG{p}{:}\PYG{+w}{ }\PYG{l+s}{\PYGZsq{}}\PYG{l+s}{Tipo\PYGZsh{}Nome}\PYG{l+s}{\PYGZsq{}}
\PYG{+w}{ }\PYG{p+pIndicator}{\PYGZhy{}}\PYG{+w}{ }\PYG{n+nt}{camada\PYGZus{}padrao}\PYG{p}{:}
\PYG{+w}{      }\PYG{n+nt}{tipo}\PYG{p}{:}\PYG{+w}{ }\PYG{l+lScalar+lScalarPlain}{poligono}
\PYG{+w}{      }\PYG{n+nt}{info}\PYG{p}{:}\PYG{+w}{ }\PYG{l+s}{\PYGZsq{}}\PYG{l+s}{Tombamento}\PYG{l+s}{\PYGZsq{}}
\PYG{+w}{      }\PYG{n+nt}{tabela}\PYG{p}{:}\PYG{+w}{ }\PYG{l+lScalar+lScalarPlain}{tombamento}
\PYG{+w}{      }\PYG{n+nt}{campos}\PYG{p}{:}\PYG{+w}{ }\PYG{l+lScalar+lScalarPlain}{name}
\PYG{+w}{      }\PYG{n+nt}{descricao}\PYG{p}{:}\PYG{+w}{ }\PYG{l+s}{\PYGZsq{}}\PYG{l+s}{Nome}\PYG{l+s}{\PYGZsq{}}
\PYG{+w}{      }\PYG{n+nt}{popup}\PYG{p}{:}\PYG{+w}{ }\PYG{l+lScalar+lScalarPlain}{name}
\PYG{+w}{      }\PYG{n+nt}{desc\PYGZus{}popup}\PYG{p}{:}\PYG{+w}{ }\PYG{l+s}{\PYGZsq{}}\PYG{l+s}{Nome}\PYG{l+s}{\PYGZsq{}}
\PYG{+w}{ }\PYG{p+pIndicator}{\PYGZhy{}}\PYG{+w}{ }\PYG{n+nt}{topico}\PYG{p}{:}
\PYG{+w}{      }\PYG{n+nt}{titulo}\PYG{p}{:}\PYG{+w}{ }\PYG{l+s}{\PYGZsq{}}\PYG{l+s}{Restauro}\PYG{n+nv}{ }\PYG{l+s}{para}\PYG{n+nv}{ }\PYG{l+s}{Habitação}\PYG{l+s}{\PYGZsq{}}
\PYG{+w}{      }\PYG{n+nt}{descricao}\PYG{p}{:}\PYG{+w}{ }\PYG{l+s}{\PYGZsq{}}\PYG{l+s}{\PYGZsq{}}
\PYG{+w}{      }\PYG{n+nt}{mapa1}\PYG{p}{:}
\PYG{+w}{           }\PYG{n+nt}{titulo}\PYG{p}{:}\PYG{+w}{ }\PYG{l+s}{\PYGZsq{}}\PYG{l+s}{**Distribuição}\PYG{n+nv}{ }\PYG{l+s}{de}\PYG{n+nv}{ }\PYG{l+s}{Mulheres}\PYG{n+nv}{ }\PYG{l+s}{por}\PYG{n+nv}{ }\PYG{l+s}{Setor}\PYG{n+nv}{ }\PYG{l+s}{censitario**}\PYG{l+s}{\PYGZsq{}}
\PYG{+w}{           }\PYG{n+nt}{tabela}\PYG{p}{:}\PYG{+w}{ }\PYG{l+lScalar+lScalarPlain}{pdc\PYGZus{}bid\PYGZus{}ibge}
\PYG{+w}{           }\PYG{n+nt}{camada}\PYG{p}{:}\PYG{+w}{ }\PYG{l+lScalar+lScalarPlain}{setor\PYGZus{}censitario}
\PYG{+w}{           }\PYG{n+nt}{variavel}\PYG{p}{:}\PYG{+w}{ }\PYG{l+s}{\PYGZsq{}}\PYG{l+s}{mulheres\PYGZus{}14\PYGZus{}anos\PYGZus{}ou\PYGZus{}mais\PYGZsh{}mulheres\PYGZus{}resp\PYGZus{}pelo\PYGZus{}dom}\PYG{l+s}{\PYGZsq{}}
\PYG{+w}{           }\PYG{n+nt}{alias}\PYG{p}{:}\PYG{+w}{ }\PYG{l+s}{\PYGZsq{}}\PYG{l+s}{Mulheres}\PYG{n+nv}{ }\PYG{l+s}{(14}\PYG{n+nv}{ }\PYG{l+s}{anos}\PYG{n+nv}{ }\PYG{l+s}{ou}\PYG{n+nv}{ }\PYG{l+s}{mais):\PYGZsh{}Mulheres}\PYG{n+nv}{ }\PYG{l+s}{responsáveis}\PYG{n+nv}{ }\PYG{l+s}{pelo}\PYG{n+nv}{ }\PYG{l+s}{domicílio}\PYG{l+s}{\PYGZsq{}}
\PYG{+w}{           }\PYG{n+nt}{descricao}\PYG{p}{:}\PYG{+w}{ }\PYG{l+s}{\PYGZsq{}}\PYG{l+s}{\PYGZsq{}}
\PYG{+w}{           }\PYG{n+nt}{camada\PYGZus{}extra}\PYG{p}{:}\PYG{+w}{ }\PYG{l+s}{\PYGZsq{}}\PYG{l+s}{lotes\PYGZus{}por\PYGZus{}ponto}\PYG{l+s}{\PYGZsq{}}
\PYG{+w}{           }\PYG{n+nt}{camada\PYGZus{}interna}\PYG{p}{:}\PYG{+w}{ }\PYG{l+lScalar+lScalarPlain}{tombamento}
\PYG{+w}{           }\PYG{n+nt}{camada\PYGZus{}base}\PYG{p}{:}\PYG{+w}{ }\PYG{l+lScalar+lScalarPlain}{anel\PYGZus{}viario}
\PYG{+w}{      }\PYG{n+nt}{grafico1}\PYG{p}{:}
\PYG{+w}{           }\PYG{n+nt}{titulo}\PYG{p}{:}\PYG{+w}{ }\PYG{l+s}{\PYGZsq{}}\PYG{l+s}{**Distribuição}\PYG{n+nv}{ }\PYG{l+s}{da}\PYG{n+nv}{ }\PYG{l+s}{população**}\PYG{l+s}{\PYGZsq{}}
\PYG{+w}{           }\PYG{n+nt}{tabela}\PYG{p}{:}\PYG{+w}{ }\PYG{l+lScalar+lScalarPlain}{pdc\PYGZus{}bid\PYGZus{}ibge}
\PYG{+w}{           }\PYG{n+nt}{x}\PYG{p}{:}
\PYG{+w}{                }\PYG{p+pIndicator}{\PYGZhy{}}\PYG{+w}{ }\PYG{l+lScalar+lScalarPlain}{mulheres\PYGZus{}14\PYGZus{}anos\PYGZus{}ou\PYGZus{}mais}
\PYG{+w}{                }\PYG{p+pIndicator}{\PYGZhy{}}\PYG{+w}{ }\PYG{l+lScalar+lScalarPlain}{mulher\PYGZus{}responsavel}
\PYG{+w}{                }\PYG{p+pIndicator}{\PYGZhy{}}\PYG{+w}{ }\PYG{l+lScalar+lScalarPlain}{idosos\PYGZus{}65\PYGZus{}anos\PYGZus{}ou\PYGZus{}mais}
\PYG{+w}{                }\PYG{p+pIndicator}{\PYGZhy{}}\PYG{+w}{ }\PYG{l+lScalar+lScalarPlain}{crianca\PYGZus{}13\PYGZus{}ou\PYGZus{}menos}
\PYG{+w}{           }\PYG{n+nt}{x\PYGZus{}alias}\PYG{p}{:}
\PYG{+w}{                }\PYG{p+pIndicator}{\PYGZhy{}}\PYG{+w}{ }\PYG{l+lScalar+lScalarPlain}{Mulheres}\PYG{l+lScalar+lScalarPlain}{ }\PYG{l+lScalar+lScalarPlain}{de}\PYG{l+lScalar+lScalarPlain}{ }\PYG{l+lScalar+lScalarPlain}{14}\PYG{l+lScalar+lScalarPlain}{ }\PYG{l+lScalar+lScalarPlain}{anos}\PYG{l+lScalar+lScalarPlain}{ }\PYG{l+lScalar+lScalarPlain}{ou}\PYG{l+lScalar+lScalarPlain}{ }\PYG{l+lScalar+lScalarPlain}{mais}
\PYG{+w}{                }\PYG{p+pIndicator}{\PYGZhy{}}\PYG{+w}{ }\PYG{l+lScalar+lScalarPlain}{Mulher}\PYG{l+lScalar+lScalarPlain}{ }\PYG{l+lScalar+lScalarPlain}{Responsável}
\PYG{+w}{                }\PYG{p+pIndicator}{\PYGZhy{}}\PYG{+w}{ }\PYG{l+lScalar+lScalarPlain}{Idosos}\PYG{l+lScalar+lScalarPlain}{ }\PYG{l+lScalar+lScalarPlain}{de}\PYG{l+lScalar+lScalarPlain}{ }\PYG{l+lScalar+lScalarPlain}{65}\PYG{l+lScalar+lScalarPlain}{ }\PYG{l+lScalar+lScalarPlain}{anos}\PYG{l+lScalar+lScalarPlain}{ }\PYG{l+lScalar+lScalarPlain}{ou}\PYG{l+lScalar+lScalarPlain}{ }\PYG{l+lScalar+lScalarPlain}{mais}
\PYG{+w}{                }\PYG{p+pIndicator}{\PYGZhy{}}\PYG{+w}{ }\PYG{l+lScalar+lScalarPlain}{Crianças}\PYG{l+lScalar+lScalarPlain}{ }\PYG{l+lScalar+lScalarPlain}{de}\PYG{l+lScalar+lScalarPlain}{ }\PYG{l+lScalar+lScalarPlain}{13}\PYG{l+lScalar+lScalarPlain}{ }\PYG{l+lScalar+lScalarPlain}{anos}\PYG{l+lScalar+lScalarPlain}{ }\PYG{l+lScalar+lScalarPlain}{ou}\PYG{l+lScalar+lScalarPlain}{ }\PYG{l+lScalar+lScalarPlain}{menos}
\PYG{+w}{           }\PYG{n+nt}{y}\PYG{p}{:}
\PYG{+w}{                }\PYG{p+pIndicator}{\PYGZhy{}}\PYG{+w}{ }\PYG{l+lScalar+lScalarPlain}{Quantidade}
\PYG{+w}{           }\PYG{n+nt}{tipo}\PYG{p}{:}\PYG{+w}{ }\PYG{l+lScalar+lScalarPlain}{barra\PYGZus{}vertical}
\PYG{+w}{           }\PYG{n+nt}{descricao}\PYG{p}{:}\PYG{+w}{ }\PYG{l+s}{\PYGZsq{}}\PYG{l+s}{\PYGZsq{}}
\PYG{+w}{      }\PYG{n+nt}{mapa2}\PYG{p}{:}
\PYG{+w}{           }\PYG{n+nt}{titulo}\PYG{p}{:}\PYG{+w}{ }\PYG{l+s}{\PYGZsq{}}\PYG{l+s}{**Lotes**}\PYG{l+s}{\PYGZsq{}}
\PYG{+w}{           }\PYG{n+nt}{tabela}\PYG{p}{:}\PYG{+w}{ }\PYG{l+lScalar+lScalarPlain}{pdc\PYGZus{}bid\PYGZus{}lotes}
\PYG{+w}{           }\PYG{n+nt}{camada}\PYG{p}{:}\PYG{+w}{ }\PYG{l+lScalar+lScalarPlain}{lotes}
\PYG{+w}{           }\PYG{n+nt}{variavel}\PYG{p}{:}\PYG{+w}{ }\PYG{l+s}{\PYGZsq{}}\PYG{l+s}{tipologia\PYGZus{}finais\PYGZsh{}conservacao\PYGZus{}finais\PYGZsh{}uso\PYGZus{}finais\PYGZsh{}prioridade\PYGZus{}finais\PYGZsh{}divida}\PYG{l+s}{\PYGZsq{}}
\PYG{+w}{           }\PYG{n+nt}{alias}\PYG{p}{:}\PYG{+w}{ }\PYG{l+s}{\PYGZsq{}}\PYG{l+s}{Tipologia\PYGZsh{}Estado}\PYG{n+nv}{ }\PYG{l+s}{de}\PYG{n+nv}{ }\PYG{l+s}{Conservação\PYGZsh{}Uso\PYGZsh{}Prioridade\PYGZsh{}Divida}\PYG{l+s}{\PYGZsq{}}
\PYG{+w}{           }\PYG{n+nt}{descricao}\PYG{p}{:}\PYG{+w}{ }\PYG{l+s}{\PYGZsq{}}\PYG{l+s}{\PYGZsq{}}
\PYG{+w}{           }\PYG{n+nt}{camada\PYGZus{}extra}\PYG{p}{:}\PYG{+w}{ }\PYG{l+s}{\PYGZsq{}}\PYG{l+s}{equipamento\PYGZus{}publico}\PYG{l+s}{\PYGZsq{}}
\PYG{+w}{           }\PYG{n+nt}{camada\PYGZus{}interna}\PYG{p}{:}\PYG{+w}{ }\PYG{l+s}{\PYGZsq{}}\PYG{l+s}{\PYGZsq{}}
\PYG{+w}{           }\PYG{n+nt}{camada\PYGZus{}base}\PYG{p}{:}\PYG{+w}{ }\PYG{l+s}{\PYGZsq{}}\PYG{l+s}{anel\PYGZus{}viario}\PYG{l+s}{\PYGZsq{}}
\PYG{+w}{ }\PYG{p+pIndicator}{\PYGZhy{}}\PYG{+w}{ }\PYG{n+nt}{topico}\PYG{p}{:}
\PYG{+w}{      }\PYG{n+nt}{titulo}\PYG{p}{:}\PYG{+w}{ }\PYG{l+s}{\PYGZsq{}}\PYG{l+s}{Habitações}\PYG{n+nv}{ }\PYG{l+s}{Precárias}\PYG{l+s}{\PYGZsq{}}
\PYG{+w}{      }\PYG{n+nt}{descricao}\PYG{p}{:}\PYG{+w}{ }\PYG{l+s}{\PYGZsq{}}\PYG{l+s}{\PYGZsq{}}
\PYG{+w}{      }\PYG{n+nt}{mapa1}\PYG{p}{:}
\PYG{+w}{           }\PYG{n+nt}{titulo}\PYG{p}{:}\PYG{+w}{ }\PYG{l+s}{\PYGZsq{}}\PYG{l+s}{**Distribruição}\PYG{n+nv}{ }\PYG{l+s}{de}\PYG{n+nv}{ }\PYG{l+s}{banheiros}\PYG{n+nv}{ }\PYG{l+s}{nas}\PYG{n+nv}{ }\PYG{l+s}{habitações**}\PYG{l+s}{\PYGZsq{}}
\PYG{+w}{           }\PYG{n+nt}{tabela}\PYG{p}{:}\PYG{+w}{ }\PYG{l+lScalar+lScalarPlain}{pdc\PYGZus{}bid\PYGZus{}ibge}
\PYG{+w}{           }\PYG{n+nt}{camada}\PYG{p}{:}\PYG{+w}{ }\PYG{l+lScalar+lScalarPlain}{setor\PYGZus{}censitario}
\PYG{+w}{           }\PYG{n+nt}{variavel}\PYG{p}{:}\PYG{+w}{ }\PYG{l+s}{\PYGZsq{}}\PYG{l+s}{dom\PYGZus{}sem\PYGZus{}banheiro\PYGZus{}nem\PYGZus{}sanitario\PYGZsh{}dom\PYGZus{}sem\PYGZus{}banheiro\PYGZsh{}dom\PYGZus{}com\PYGZus{}1\PYGZus{}banheiro\PYGZsh{}dom\PYGZus{}com\PYGZus{}2\PYGZus{}banheiro\PYGZsh{}dom\PYGZus{}com\PYGZus{}3\PYGZus{}banheiro}\PYG{l+s}{\PYGZsq{}}
\PYG{+w}{           }\PYG{n+nt}{alias}\PYG{p}{:}\PYG{+w}{ }\PYG{l+s}{\PYGZsq{}}\PYG{l+s}{Dom.}\PYG{n+nv}{ }\PYG{l+s}{sem}\PYG{n+nv}{ }\PYG{l+s}{banheiro}\PYG{n+nv}{ }\PYG{l+s}{nem}\PYG{n+nv}{ }\PYG{l+s}{sanitário\PYGZsh{}Dom.}\PYG{n+nv}{ }\PYG{l+s}{sem}\PYG{n+nv}{ }\PYG{l+s}{banheiro\PYGZsh{}Dom.}\PYG{n+nv}{ }\PYG{l+s}{com}\PYG{n+nv}{ }\PYG{l+s}{até}\PYG{n+nv}{ }\PYG{l+s}{1}\PYG{n+nv}{ }\PYG{l+s}{banheiro\PYGZsh{}Dom.}\PYG{n+nv}{ }\PYG{l+s}{com}\PYG{n+nv}{ }\PYG{l+s}{até}\PYG{n+nv}{ }\PYG{l+s}{2}\PYG{n+nv}{ }\PYG{l+s}{banheiros\PYGZsh{}Dom.}\PYG{n+nv}{ }\PYG{l+s}{com}\PYG{n+nv}{ }\PYG{l+s}{até}\PYG{n+nv}{ }\PYG{l+s}{3}\PYG{n+nv}{ }\PYG{l+s}{banheiros}\PYG{l+s}{\PYGZsq{}}
\PYG{+w}{           }\PYG{n+nt}{descricao}\PYG{p}{:}\PYG{+w}{ }\PYG{l+s}{\PYGZsq{}}\PYG{l+s}{\PYGZsq{}}
\PYG{+w}{           }\PYG{n+nt}{camada\PYGZus{}extra}\PYG{p}{:}\PYG{+w}{ }\PYG{l+s}{\PYGZsq{}}\PYG{l+s}{equipamento\PYGZus{}publico}\PYG{l+s}{\PYGZsq{}}
\PYG{+w}{           }\PYG{n+nt}{camada\PYGZus{}interna}\PYG{p}{:}\PYG{+w}{ }\PYG{l+s}{\PYGZsq{}}\PYG{l+s}{\PYGZsq{}}
\PYG{+w}{           }\PYG{n+nt}{camada\PYGZus{}base}\PYG{p}{:}\PYG{+w}{ }\PYG{l+s}{\PYGZsq{}}\PYG{l+s}{anel\PYGZus{}viario}\PYG{l+s}{\PYGZsq{}}
\PYG{+w}{      }\PYG{n+nt}{grafico1}\PYG{p}{:}
\PYG{+w}{           }\PYG{n+nt}{titulo}\PYG{p}{:}\PYG{+w}{ }\PYG{l+s}{\PYGZsq{}}\PYG{l+s}{**Distribuição}\PYG{n+nv}{ }\PYG{l+s}{da}\PYG{n+nv}{ }\PYG{l+s}{população**}\PYG{l+s}{\PYGZsq{}}
\PYG{+w}{           }\PYG{n+nt}{tabela}\PYG{p}{:}\PYG{+w}{ }\PYG{l+lScalar+lScalarPlain}{pdc\PYGZus{}bid\PYGZus{}ibge}
\PYG{+w}{           }\PYG{n+nt}{x}\PYG{p}{:}
\PYG{+w}{                }\PYG{p+pIndicator}{\PYGZhy{}}\PYG{+w}{ }\PYG{l+lScalar+lScalarPlain}{mulheres\PYGZus{}14\PYGZus{}anos\PYGZus{}ou\PYGZus{}mais}
\PYG{+w}{                }\PYG{p+pIndicator}{\PYGZhy{}}\PYG{+w}{ }\PYG{l+lScalar+lScalarPlain}{mulher\PYGZus{}responsavel}
\PYG{+w}{                }\PYG{p+pIndicator}{\PYGZhy{}}\PYG{+w}{ }\PYG{l+lScalar+lScalarPlain}{idosos\PYGZus{}65\PYGZus{}anos\PYGZus{}ou\PYGZus{}mais}
\PYG{+w}{                }\PYG{p+pIndicator}{\PYGZhy{}}\PYG{+w}{ }\PYG{l+lScalar+lScalarPlain}{crianca\PYGZus{}13\PYGZus{}ou\PYGZus{}menos}
\PYG{+w}{           }\PYG{n+nt}{x\PYGZus{}alias}\PYG{p}{:}
\PYG{+w}{                }\PYG{p+pIndicator}{\PYGZhy{}}\PYG{+w}{ }\PYG{l+lScalar+lScalarPlain}{Mulheres}\PYG{l+lScalar+lScalarPlain}{ }\PYG{l+lScalar+lScalarPlain}{de}\PYG{l+lScalar+lScalarPlain}{ }\PYG{l+lScalar+lScalarPlain}{14}\PYG{l+lScalar+lScalarPlain}{ }\PYG{l+lScalar+lScalarPlain}{anos}\PYG{l+lScalar+lScalarPlain}{ }\PYG{l+lScalar+lScalarPlain}{ou}\PYG{l+lScalar+lScalarPlain}{ }\PYG{l+lScalar+lScalarPlain}{mais}
\PYG{+w}{                }\PYG{p+pIndicator}{\PYGZhy{}}\PYG{+w}{ }\PYG{l+lScalar+lScalarPlain}{Mulher}\PYG{l+lScalar+lScalarPlain}{ }\PYG{l+lScalar+lScalarPlain}{Responsável}
\PYG{+w}{                }\PYG{p+pIndicator}{\PYGZhy{}}\PYG{+w}{ }\PYG{l+lScalar+lScalarPlain}{Idosos}\PYG{l+lScalar+lScalarPlain}{ }\PYG{l+lScalar+lScalarPlain}{de}\PYG{l+lScalar+lScalarPlain}{ }\PYG{l+lScalar+lScalarPlain}{65}\PYG{l+lScalar+lScalarPlain}{ }\PYG{l+lScalar+lScalarPlain}{anos}\PYG{l+lScalar+lScalarPlain}{ }\PYG{l+lScalar+lScalarPlain}{ou}\PYG{l+lScalar+lScalarPlain}{ }\PYG{l+lScalar+lScalarPlain}{mais}
\PYG{+w}{                }\PYG{p+pIndicator}{\PYGZhy{}}\PYG{+w}{ }\PYG{l+lScalar+lScalarPlain}{Crianças}\PYG{l+lScalar+lScalarPlain}{ }\PYG{l+lScalar+lScalarPlain}{de}\PYG{l+lScalar+lScalarPlain}{ }\PYG{l+lScalar+lScalarPlain}{13}\PYG{l+lScalar+lScalarPlain}{ }\PYG{l+lScalar+lScalarPlain}{anos}\PYG{l+lScalar+lScalarPlain}{ }\PYG{l+lScalar+lScalarPlain}{ou}\PYG{l+lScalar+lScalarPlain}{ }\PYG{l+lScalar+lScalarPlain}{menos}
\PYG{+w}{           }\PYG{n+nt}{y}\PYG{p}{:}
\PYG{+w}{                }\PYG{p+pIndicator}{\PYGZhy{}}\PYG{+w}{ }\PYG{l+lScalar+lScalarPlain}{Quantidade}
\PYG{+w}{           }\PYG{n+nt}{tipo}\PYG{p}{:}\PYG{+w}{ }\PYG{l+lScalar+lScalarPlain}{barra\PYGZus{}vertical}
\PYG{+w}{           }\PYG{n+nt}{descricao}\PYG{p}{:}\PYG{+w}{ }\PYG{l+s}{\PYGZsq{}}\PYG{l+s}{\PYGZsq{}}
\PYG{+w}{      }\PYG{n+nt}{grafico2}\PYG{p}{:}
\PYG{+w}{           }\PYG{n+nt}{titulo}\PYG{p}{:}\PYG{+w}{ }\PYG{l+s}{\PYGZsq{}}\PYG{l+s}{**Distribuição}\PYG{n+nv}{ }\PYG{l+s}{de}\PYG{n+nv}{ }\PYG{l+s}{Esgoto**}\PYG{l+s}{\PYGZsq{}}
\PYG{+w}{           }\PYG{n+nt}{tabela}\PYG{p}{:}\PYG{+w}{ }\PYG{l+lScalar+lScalarPlain}{pdc\PYGZus{}bid\PYGZus{}ibge}
\PYG{+w}{           }\PYG{n+nt}{x}\PYG{p}{:}
\PYG{+w}{                }\PYG{p+pIndicator}{\PYGZhy{}}\PYG{+w}{ }\PYG{l+lScalar+lScalarPlain}{dom\PYGZus{}banh\PYGZus{}excl\PYGZus{}rede\PYGZus{}geral}
\PYG{+w}{                }\PYG{p+pIndicator}{\PYGZhy{}}\PYG{+w}{ }\PYG{l+lScalar+lScalarPlain}{dom\PYGZus{}banh\PYGZus{}excl\PYGZus{}fossa\PYGZus{}septica}
\PYG{+w}{                }\PYG{p+pIndicator}{\PYGZhy{}}\PYG{+w}{ }\PYG{l+lScalar+lScalarPlain}{dom\PYGZus{}banh\PYGZus{}excl\PYGZus{}fossa\PYGZus{}rud}
\PYG{+w}{                }\PYG{p+pIndicator}{\PYGZhy{}}\PYG{+w}{ }\PYG{l+lScalar+lScalarPlain}{dom\PYGZus{}banh\PYGZus{}excl\PYGZus{}vala}
\PYG{+w}{                }\PYG{p+pIndicator}{\PYGZhy{}}\PYG{+w}{ }\PYG{l+lScalar+lScalarPlain}{dom\PYGZus{}banh\PYGZus{}excl\PYGZus{}rio\PYGZus{}lago\PYGZus{}mar}
\PYG{+w}{                }\PYG{p+pIndicator}{\PYGZhy{}}\PYG{+w}{ }\PYG{l+lScalar+lScalarPlain}{dom\PYGZus{}banh\PYGZus{}excl\PYGZus{}outro}
\PYG{+w}{           }\PYG{n+nt}{x\PYGZus{}alias}\PYG{p}{:}
\PYG{+w}{                }\PYG{p+pIndicator}{\PYGZhy{}}\PYG{+w}{ }\PYG{l+lScalar+lScalarPlain}{Rede}\PYG{l+lScalar+lScalarPlain}{ }\PYG{l+lScalar+lScalarPlain}{Geral}
\PYG{+w}{                }\PYG{p+pIndicator}{\PYGZhy{}}\PYG{+w}{ }\PYG{l+lScalar+lScalarPlain}{Fossa}\PYG{l+lScalar+lScalarPlain}{ }\PYG{l+lScalar+lScalarPlain}{Séptica}
\PYG{+w}{                }\PYG{p+pIndicator}{\PYGZhy{}}\PYG{+w}{ }\PYG{l+lScalar+lScalarPlain}{Fossa}\PYG{l+lScalar+lScalarPlain}{ }\PYG{l+lScalar+lScalarPlain}{Rudimentar}
\PYG{+w}{                }\PYG{p+pIndicator}{\PYGZhy{}}\PYG{+w}{ }\PYG{l+lScalar+lScalarPlain}{Vala}
\PYG{+w}{                }\PYG{p+pIndicator}{\PYGZhy{}}\PYG{+w}{ }\PYG{l+lScalar+lScalarPlain}{Rio,}\PYG{l+lScalar+lScalarPlain}{ }\PYG{l+lScalar+lScalarPlain}{Lago}\PYG{l+lScalar+lScalarPlain}{ }\PYG{l+lScalar+lScalarPlain}{ou}\PYG{l+lScalar+lScalarPlain}{ }\PYG{l+lScalar+lScalarPlain}{Mar}
\PYG{+w}{                }\PYG{p+pIndicator}{\PYGZhy{}}\PYG{+w}{ }\PYG{l+lScalar+lScalarPlain}{Outro}
\PYG{+w}{           }\PYG{n+nt}{y}\PYG{p}{:}
\PYG{+w}{                }\PYG{p+pIndicator}{\PYGZhy{}}\PYG{+w}{ }\PYG{l+lScalar+lScalarPlain}{Quantidade}
\PYG{+w}{           }\PYG{n+nt}{tipo}\PYG{p}{:}\PYG{+w}{ }\PYG{l+lScalar+lScalarPlain}{pizza}
\PYG{+w}{           }\PYG{n+nt}{descricao}\PYG{p}{:}\PYG{+w}{ }\PYG{l+s}{\PYGZsq{}}\PYG{l+s}{\PYGZsq{}}
\PYG{+w}{      }\PYG{n+nt}{grafico3}\PYG{p}{:}
\PYG{+w}{           }\PYG{n+nt}{titulo}\PYG{p}{:}\PYG{+w}{ }\PYG{l+s}{\PYGZsq{}}\PYG{l+s}{**Distribuição}\PYG{n+nv}{ }\PYG{l+s}{de}\PYG{n+nv}{ }\PYG{l+s}{banheiros**}\PYG{l+s}{\PYGZsq{}}
\PYG{+w}{           }\PYG{n+nt}{tabela}\PYG{p}{:}\PYG{+w}{ }\PYG{l+lScalar+lScalarPlain}{pdc\PYGZus{}bid\PYGZus{}ibge}
\PYG{+w}{           }\PYG{n+nt}{x}\PYG{p}{:}
\PYG{+w}{                }\PYG{p+pIndicator}{\PYGZhy{}}\PYG{+w}{ }\PYG{l+lScalar+lScalarPlain}{dom\PYGZus{}sem\PYGZus{}banheiro\PYGZus{}nem\PYGZus{}sanitario}
\PYG{+w}{                }\PYG{p+pIndicator}{\PYGZhy{}}\PYG{+w}{ }\PYG{l+lScalar+lScalarPlain}{dom\PYGZus{}com\PYGZus{}1\PYGZus{}banheiro}
\PYG{+w}{                }\PYG{p+pIndicator}{\PYGZhy{}}\PYG{+w}{ }\PYG{l+lScalar+lScalarPlain}{dom\PYGZus{}com\PYGZus{}2\PYGZus{}banheiro}
\PYG{+w}{                }\PYG{p+pIndicator}{\PYGZhy{}}\PYG{+w}{ }\PYG{l+lScalar+lScalarPlain}{dom\PYGZus{}com\PYGZus{}3\PYGZus{}banheiro}
\PYG{+w}{           }\PYG{n+nt}{x\PYGZus{}alias}\PYG{p}{:}
\PYG{+w}{                }\PYG{p+pIndicator}{\PYGZhy{}}\PYG{+w}{ }\PYG{l+lScalar+lScalarPlain}{Dom.}\PYG{l+lScalar+lScalarPlain}{ }\PYG{l+lScalar+lScalarPlain}{sem}\PYG{l+lScalar+lScalarPlain}{ }\PYG{l+lScalar+lScalarPlain}{banheiro}\PYG{l+lScalar+lScalarPlain}{ }\PYG{l+lScalar+lScalarPlain}{nem}\PYG{l+lScalar+lScalarPlain}{ }\PYG{l+lScalar+lScalarPlain}{sanitário}
\PYG{+w}{                }\PYG{p+pIndicator}{\PYGZhy{}}\PYG{+w}{ }\PYG{l+lScalar+lScalarPlain}{Dom.}\PYG{l+lScalar+lScalarPlain}{ }\PYG{l+lScalar+lScalarPlain}{com}\PYG{l+lScalar+lScalarPlain}{ }\PYG{l+lScalar+lScalarPlain}{1}\PYG{l+lScalar+lScalarPlain}{ }\PYG{l+lScalar+lScalarPlain}{banheiro}
\PYG{+w}{                }\PYG{p+pIndicator}{\PYGZhy{}}\PYG{+w}{ }\PYG{l+lScalar+lScalarPlain}{Dom.}\PYG{l+lScalar+lScalarPlain}{ }\PYG{l+lScalar+lScalarPlain}{com}\PYG{l+lScalar+lScalarPlain}{ }\PYG{l+lScalar+lScalarPlain}{2}\PYG{l+lScalar+lScalarPlain}{ }\PYG{l+lScalar+lScalarPlain}{banheiros}
\PYG{+w}{                }\PYG{p+pIndicator}{\PYGZhy{}}\PYG{+w}{ }\PYG{l+lScalar+lScalarPlain}{Dom.}\PYG{l+lScalar+lScalarPlain}{ }\PYG{l+lScalar+lScalarPlain}{com}\PYG{l+lScalar+lScalarPlain}{ }\PYG{l+lScalar+lScalarPlain}{3}\PYG{l+lScalar+lScalarPlain}{ }\PYG{l+lScalar+lScalarPlain}{banheiros}
\PYG{+w}{           }\PYG{n+nt}{y}\PYG{p}{:}
\PYG{+w}{                }\PYG{p+pIndicator}{\PYGZhy{}}\PYG{+w}{ }\PYG{l+lScalar+lScalarPlain}{Quantidade}
\PYG{+w}{           }\PYG{n+nt}{tipo}\PYG{p}{:}\PYG{+w}{ }\PYG{l+lScalar+lScalarPlain}{barra\PYGZus{}vertical}
\PYG{+w}{           }\PYG{n+nt}{descricao}\PYG{p}{:}\PYG{+w}{ }\PYG{l+s}{\PYGZsq{}}\PYG{l+s}{\PYGZsq{}}
\PYG{+w}{      }\PYG{n+nt}{grafico4}\PYG{p}{:}
\PYG{+w}{           }\PYG{n+nt}{titulo}\PYG{p}{:}\PYG{+w}{ }\PYG{l+s}{\PYGZsq{}}\PYG{l+s}{**Distribuição}\PYG{n+nv}{ }\PYG{l+s}{de}\PYG{n+nv}{ }\PYG{l+s}{Água**}\PYG{l+s}{\PYGZsq{}}
\PYG{+w}{           }\PYG{n+nt}{tabela}\PYG{p}{:}\PYG{+w}{ }\PYG{l+lScalar+lScalarPlain}{pdc\PYGZus{}bid\PYGZus{}ibge}
\PYG{+w}{           }\PYG{n+nt}{x}\PYG{p}{:}
\PYG{+w}{                }\PYG{p+pIndicator}{\PYGZhy{}}\PYG{+w}{ }\PYG{l+lScalar+lScalarPlain}{dom\PYGZus{}abastecimento\PYGZus{}agua\PYGZus{}rede\PYGZus{}geral}
\PYG{+w}{                }\PYG{p+pIndicator}{\PYGZhy{}}\PYG{+w}{ }\PYG{l+lScalar+lScalarPlain}{dom\PYGZus{}abastecimento\PYGZus{}agua\PYGZus{}poco\PYGZus{}nascente}
\PYG{+w}{                }\PYG{p+pIndicator}{\PYGZhy{}}\PYG{+w}{ }\PYG{l+lScalar+lScalarPlain}{dom\PYGZus{}abastecimento\PYGZus{}agua\PYGZus{}chuva\PYGZus{}cisterna}
\PYG{+w}{                }\PYG{p+pIndicator}{\PYGZhy{}}\PYG{+w}{ }\PYG{l+lScalar+lScalarPlain}{dom\PYGZus{}abastecimento\PYGZus{}agua\PYGZus{}outra}
\PYG{+w}{           }\PYG{n+nt}{x\PYGZus{}alias}\PYG{p}{:}
\PYG{+w}{                }\PYG{p+pIndicator}{\PYGZhy{}}\PYG{+w}{ }\PYG{l+lScalar+lScalarPlain}{Rede}\PYG{l+lScalar+lScalarPlain}{ }\PYG{l+lScalar+lScalarPlain}{Geral}
\PYG{+w}{                }\PYG{p+pIndicator}{\PYGZhy{}}\PYG{+w}{ }\PYG{l+lScalar+lScalarPlain}{Poço}\PYG{l+lScalar+lScalarPlain}{ }\PYG{l+lScalar+lScalarPlain}{ou}\PYG{l+lScalar+lScalarPlain}{ }\PYG{l+lScalar+lScalarPlain}{Nascente}
\PYG{+w}{                }\PYG{p+pIndicator}{\PYGZhy{}}\PYG{+w}{ }\PYG{l+lScalar+lScalarPlain}{Chuva}\PYG{l+lScalar+lScalarPlain}{ }\PYG{l+lScalar+lScalarPlain}{ou}\PYG{l+lScalar+lScalarPlain}{ }\PYG{l+lScalar+lScalarPlain}{Cisterna}
\PYG{+w}{                }\PYG{p+pIndicator}{\PYGZhy{}}\PYG{+w}{ }\PYG{l+lScalar+lScalarPlain}{Outro}
\PYG{+w}{           }\PYG{n+nt}{y}\PYG{p}{:}
\PYG{+w}{                }\PYG{p+pIndicator}{\PYGZhy{}}\PYG{+w}{ }\PYG{l+lScalar+lScalarPlain}{Quantidade}
\PYG{+w}{           }\PYG{n+nt}{tipo}\PYG{p}{:}\PYG{+w}{ }\PYG{l+lScalar+lScalarPlain}{barra\PYGZus{}horizontal}
\PYG{+w}{           }\PYG{n+nt}{descricao}\PYG{p}{:}\PYG{+w}{ }\PYG{l+s}{\PYGZsq{}}\PYG{l+s}{\PYGZsq{}}
\PYG{+w}{      }\PYG{n+nt}{grafico5}\PYG{p}{:}
\PYG{+w}{           }\PYG{n+nt}{titulo}\PYG{p}{:}\PYG{+w}{ }\PYG{l+s}{\PYGZsq{}}\PYG{l+s}{**Distribuição}\PYG{n+nv}{ }\PYG{l+s}{de}\PYG{n+nv}{ }\PYG{l+s}{Lixo**}\PYG{l+s}{\PYGZsq{}}
\PYG{+w}{           }\PYG{n+nt}{tabela}\PYG{p}{:}\PYG{+w}{ }\PYG{l+lScalar+lScalarPlain}{pdc\PYGZus{}bid\PYGZus{}ibge}
\PYG{+w}{           }\PYG{n+nt}{x}\PYG{p}{:}
\PYG{+w}{                }\PYG{p+pIndicator}{\PYGZhy{}}\PYG{+w}{ }\PYG{l+lScalar+lScalarPlain}{dom\PYGZus{}part\PYGZus{}lixo\PYGZus{}coletado\PYGZus{}servico}
\PYG{+w}{                }\PYG{p+pIndicator}{\PYGZhy{}}\PYG{+w}{ }\PYG{l+lScalar+lScalarPlain}{dom\PYGZus{}part\PYGZus{}lixo\PYGZus{}coletado\PYGZus{}cacamba\PYGZus{}servico}
\PYG{+w}{                }\PYG{p+pIndicator}{\PYGZhy{}}\PYG{+w}{ }\PYG{l+lScalar+lScalarPlain}{dom\PYGZus{}part\PYGZus{}lixo\PYGZus{}queimado\PYGZus{}propriedade}
\PYG{+w}{                }\PYG{p+pIndicator}{\PYGZhy{}}\PYG{+w}{ }\PYG{l+lScalar+lScalarPlain}{dom\PYGZus{}part\PYGZus{}lixo\PYGZus{}enterrado\PYGZus{}propriedade}
\PYG{+w}{                }\PYG{p+pIndicator}{\PYGZhy{}}\PYG{+w}{ }\PYG{l+lScalar+lScalarPlain}{dom\PYGZus{}part\PYGZus{}lixo\PYGZus{}terreno\PYGZus{}baldio}
\PYG{+w}{                }\PYG{p+pIndicator}{\PYGZhy{}}\PYG{+w}{ }\PYG{l+lScalar+lScalarPlain}{dom\PYGZus{}part\PYGZus{}lixo\PYGZus{}rio\PYGZus{}lago\PYGZus{}mar}
\PYG{+w}{                }\PYG{p+pIndicator}{\PYGZhy{}}\PYG{+w}{ }\PYG{l+lScalar+lScalarPlain}{dom\PYGZus{}part\PYGZus{}lixo\PYGZus{}outro}
\PYG{+w}{           }\PYG{n+nt}{x\PYGZus{}alias}\PYG{p}{:}
\PYG{+w}{                }\PYG{p+pIndicator}{\PYGZhy{}}\PYG{+w}{ }\PYG{l+lScalar+lScalarPlain}{Serviço}\PYG{l+lScalar+lScalarPlain}{ }\PYG{l+lScalar+lScalarPlain}{de}\PYG{l+lScalar+lScalarPlain}{ }\PYG{l+lScalar+lScalarPlain}{Coleta}
\PYG{+w}{                }\PYG{p+pIndicator}{\PYGZhy{}}\PYG{+w}{ }\PYG{l+lScalar+lScalarPlain}{Caçamba}
\PYG{+w}{                }\PYG{p+pIndicator}{\PYGZhy{}}\PYG{+w}{ }\PYG{l+lScalar+lScalarPlain}{Lixo}\PYG{l+lScalar+lScalarPlain}{ }\PYG{l+lScalar+lScalarPlain}{Queimado}
\PYG{+w}{                }\PYG{p+pIndicator}{\PYGZhy{}}\PYG{+w}{ }\PYG{l+lScalar+lScalarPlain}{Enterrado}\PYG{l+lScalar+lScalarPlain}{ }\PYG{l+lScalar+lScalarPlain}{na}\PYG{l+lScalar+lScalarPlain}{ }\PYG{l+lScalar+lScalarPlain}{Propriedade}
\PYG{+w}{                }\PYG{p+pIndicator}{\PYGZhy{}}\PYG{+w}{ }\PYG{l+lScalar+lScalarPlain}{Terreno}\PYG{l+lScalar+lScalarPlain}{ }\PYG{l+lScalar+lScalarPlain}{Baldio}
\PYG{+w}{                }\PYG{p+pIndicator}{\PYGZhy{}}\PYG{+w}{ }\PYG{l+lScalar+lScalarPlain}{Rio,}\PYG{l+lScalar+lScalarPlain}{ }\PYG{l+lScalar+lScalarPlain}{Lago}\PYG{l+lScalar+lScalarPlain}{ }\PYG{l+lScalar+lScalarPlain}{ou}\PYG{l+lScalar+lScalarPlain}{ }\PYG{l+lScalar+lScalarPlain}{Mar}
\PYG{+w}{                }\PYG{p+pIndicator}{\PYGZhy{}}\PYG{+w}{ }\PYG{l+lScalar+lScalarPlain}{Outro}
\PYG{+w}{           }\PYG{n+nt}{y}\PYG{p}{:}
\PYG{+w}{                }\PYG{p+pIndicator}{\PYGZhy{}}\PYG{+w}{ }\PYG{l+lScalar+lScalarPlain}{Quantidade}
\PYG{+w}{           }\PYG{n+nt}{tipo}\PYG{p}{:}\PYG{+w}{ }\PYG{l+lScalar+lScalarPlain}{pizza}
\PYG{+w}{           }\PYG{n+nt}{descricao}\PYG{p}{:}\PYG{+w}{ }\PYG{l+s}{\PYGZsq{}}\PYG{l+s}{\PYGZsq{}}
\PYG{+w}{ }\PYG{p+pIndicator}{\PYGZhy{}}\PYG{+w}{ }\PYG{n+nt}{topico}\PYG{p}{:}
\PYG{+w}{      }\PYG{n+nt}{titulo}\PYG{p}{:}\PYG{+w}{ }\PYG{l+s}{\PYGZsq{}}\PYG{l+s}{Equipamentos}\PYG{n+nv}{ }\PYG{l+s}{Públicos}\PYG{l+s}{\PYGZsq{}}
\PYG{+w}{      }\PYG{n+nt}{descricao}\PYG{p}{:}\PYG{+w}{ }\PYG{l+s}{\PYGZsq{}}\PYG{l+s}{\PYGZsq{}}
\PYG{+w}{ }\PYG{p+pIndicator}{\PYGZhy{}}\PYG{+w}{ }\PYG{n+nt}{topico}\PYG{p}{:}
\PYG{+w}{      }\PYG{n+nt}{titulo}\PYG{p}{:}\PYG{+w}{ }\PYG{l+s}{\PYGZsq{}}\PYG{l+s}{Imóveis}\PYG{n+nv}{ }\PYG{l+s}{Ociosos}\PYG{l+s}{\PYGZsq{}}
\PYG{+w}{      }\PYG{n+nt}{descricao}\PYG{p}{:}\PYG{+w}{ }\PYG{l+s}{\PYGZsq{}}\PYG{l+s}{\PYGZsq{}}
\PYG{+w}{ }\PYG{p+pIndicator}{\PYGZhy{}}\PYG{+w}{ }\PYG{n+nt}{topico}\PYG{p}{:}
\PYG{+w}{      }\PYG{n+nt}{titulo}\PYG{p}{:}\PYG{+w}{ }\PYG{l+s}{\PYGZsq{}}\PYG{l+s}{Regularização}\PYG{n+nv}{ }\PYG{l+s}{Fundiária}\PYG{l+s}{\PYGZsq{}}
\PYG{+w}{      }\PYG{n+nt}{descricao}\PYG{p}{:}\PYG{+w}{ }\PYG{l+s}{\PYGZsq{}}\PYG{l+s}{\PYGZsq{}}
\end{sphinxVerbatim}


\section{Implementação}
\label{\detokenize{pdcvis:module-app}}\label{\detokenize{pdcvis:implementacao}}\index{module@\spxentry{module}!app@\spxentry{app}}\index{app@\spxentry{app}!module@\spxentry{module}}

\begin{savenotes}\sphinxattablestart
\sphinxthistablewithglobalstyle
\sphinxthistablewithnovlinesstyle
\centering
\begin{tabulary}{\linewidth}[t]{\X{1}{2}\X{1}{2}}
\sphinxtoprule
\sphinxtableatstartofbodyhook
\sphinxAtStartPar
{\hyperref[\detokenize{pdcvis:app.loadData}]{\sphinxcrossref{\sphinxcode{\sphinxupquote{app.loadData}}}}}(conf)
&
\sphinxAtStartPar
Load de todos os dados informados no YAML
\\
\sphinxhline
\sphinxAtStartPar
{\hyperref[\detokenize{pdcvis:app.addMapMarcador}]{\sphinxcrossref{\sphinxcode{\sphinxupquote{app.addMapMarcador}}}}}(m, camada, camada\_sel)
&
\sphinxAtStartPar
Adiciona marcados (pontos) num mapa
\\
\sphinxhline
\sphinxAtStartPar
{\hyperref[\detokenize{pdcvis:app.addMapHeat}]{\sphinxcrossref{\sphinxcode{\sphinxupquote{app.addMapHeat}}}}}(m, camada, camada\_sel)
&
\sphinxAtStartPar
A partir dos pontos selecionados, cria um mapa de calor
\\
\sphinxhline
\sphinxAtStartPar
{\hyperref[\detokenize{pdcvis:app.addMapVoronoi}]{\sphinxcrossref{\sphinxcode{\sphinxupquote{app.addMapVoronoi}}}}}(m, camada, area\_base, ...)
&
\sphinxAtStartPar
A partir dos pontos selecionados, cria um mapa de Voronoi
\\
\sphinxhline
\sphinxAtStartPar
{\hyperref[\detokenize{pdcvis:app.addMap}]{\sphinxcrossref{\sphinxcode{\sphinxupquote{app.addMap}}}}}(geo, variavel, alias{[}, ...{]})
&
\sphinxAtStartPar
Função básica para inclusão de mapas
\\
\sphinxhline
\sphinxAtStartPar
{\hyperref[\detokenize{pdcvis:app.addGrafico}]{\sphinxcrossref{\sphinxcode{\sphinxupquote{app.addGrafico}}}}}(data, variaveis, ...{[}, agg{]})
&
\sphinxAtStartPar
Gera gráficos de acordo com a configuração informada no YAML
\\
\sphinxbottomrule
\end{tabulary}
\sphinxtableafterendhook\par
\sphinxattableend\end{savenotes}
\index{addGrafico() (in module app)@\spxentry{addGrafico()}\spxextra{in module app}}

\begin{fulllineitems}
\phantomsection\label{\detokenize{pdcvis:app.addGrafico}}
\pysigstartsignatures
\pysiglinewithargsret{\sphinxcode{\sphinxupquote{app.}}\sphinxbfcode{\sphinxupquote{addGrafico}}}{\emph{\DUrole{n}{data}}, \emph{\DUrole{n}{variaveis}}, \emph{\DUrole{n}{variaveis\_alias}}, \emph{\DUrole{n}{alias}}, \emph{\DUrole{n}{tipo}}, \emph{\DUrole{n}{agg}\DUrole{o}{=}\DUrole{default_value}{\textquotesingle{}sum\textquotesingle{}}}}{}
\pysigstopsignatures
\sphinxAtStartPar
Gera gráficos de acordo com a configuração informada no YAML
\begin{quote}\begin{description}
\sphinxlineitem{Parameters}\begin{itemize}
\item {} 
\sphinxAtStartPar
\sphinxstyleliteralstrong{\sphinxupquote{data}} \textendash{} dados para construção do gráfico

\item {} 
\sphinxAtStartPar
\sphinxstyleliteralstrong{\sphinxupquote{variaveis}} \textendash{} colunas a serem coletadas de data

\item {} 
\sphinxAtStartPar
\sphinxstyleliteralstrong{\sphinxupquote{variaveis\_alias}} \textendash{} nome formatado das variáveis

\item {} 
\sphinxAtStartPar
\sphinxstyleliteralstrong{\sphinxupquote{alias}} \textendash{} filtro selecionado na interface gráfica

\item {} 
\sphinxAtStartPar
\sphinxstyleliteralstrong{\sphinxupquote{tipo}} \textendash{} tipo do mapa, configurado no YAML

\item {} 
\sphinxAtStartPar
\sphinxstyleliteralstrong{\sphinxupquote{agg}} \textendash{} 

\end{itemize}

\end{description}\end{quote}

\end{fulllineitems}

\index{addMap() (in module app)@\spxentry{addMap()}\spxextra{in module app}}

\begin{fulllineitems}
\phantomsection\label{\detokenize{pdcvis:app.addMap}}
\pysigstartsignatures
\pysiglinewithargsret{\sphinxcode{\sphinxupquote{app.}}\sphinxbfcode{\sphinxupquote{addMap}}}{\emph{\DUrole{n}{geo}}, \emph{\DUrole{n}{variavel}}, \emph{\DUrole{n}{alias}}, \emph{\DUrole{n}{camadas\_extra}\DUrole{o}{=}\DUrole{default_value}{None}}, \emph{\DUrole{n}{camada\_interna}\DUrole{o}{=}\DUrole{default_value}{None}}, \emph{\DUrole{n}{camada\_base}\DUrole{o}{=}\DUrole{default_value}{None}}, \emph{\DUrole{n}{ponto\_tipo}\DUrole{o}{=}\DUrole{default_value}{\textquotesingle{}Marcador\textquotesingle{}}}}{}
\pysigstopsignatures
\sphinxAtStartPar
Função básica para inclusão de mapas
\begin{quote}\begin{description}
\sphinxlineitem{Parameters}\begin{itemize}
\item {} 
\sphinxAtStartPar
\sphinxstyleliteralstrong{\sphinxupquote{geo}} \textendash{} camada geográfica e dados

\item {} 
\sphinxAtStartPar
\sphinxstyleliteralstrong{\sphinxupquote{variavel}} \textendash{} informação que deve ser exibida no mapa, numa escala de cor

\item {} 
\sphinxAtStartPar
\sphinxstyleliteralstrong{\sphinxupquote{alias}} \textendash{} nome formatado da variável

\item {} 
\sphinxAtStartPar
\sphinxstyleliteralstrong{\sphinxupquote{camadas\_extra}} \textendash{} informação de camada adicional a ser inserida por seleção

\item {} 
\sphinxAtStartPar
\sphinxstyleliteralstrong{\sphinxupquote{camadas\_interna}} \textendash{} informação de camada adicional a ser inserida por padrão

\item {} 
\sphinxAtStartPar
\sphinxstyleliteralstrong{\sphinxupquote{camadas\_base}} \textendash{} limites da área de interesse, usado pelo voronoi

\item {} 
\sphinxAtStartPar
\sphinxstyleliteralstrong{\sphinxupquote{ponto\_tipo}} \textendash{} Marcador \sphinxhyphen{} Calor ou Voronoi

\end{itemize}

\end{description}\end{quote}

\end{fulllineitems}

\index{addMapHeat() (in module app)@\spxentry{addMapHeat()}\spxextra{in module app}}

\begin{fulllineitems}
\phantomsection\label{\detokenize{pdcvis:app.addMapHeat}}
\pysigstartsignatures
\pysiglinewithargsret{\sphinxcode{\sphinxupquote{app.}}\sphinxbfcode{\sphinxupquote{addMapHeat}}}{\emph{\DUrole{n}{m}}, \emph{\DUrole{n}{camada}}, \emph{\DUrole{n}{camada\_sel}}}{}
\pysigstopsignatures
\sphinxAtStartPar
A partir dos pontos selecionados, cria um mapa de calor
\begin{quote}\begin{description}
\sphinxlineitem{Parameters}\begin{itemize}
\item {} 
\sphinxAtStartPar
\sphinxstyleliteralstrong{\sphinxupquote{camada}} \textendash{} camada previamente informada com os pontos

\item {} 
\sphinxAtStartPar
\sphinxstyleliteralstrong{\sphinxupquote{camada\_sel}} \textendash{} filtro selecionado na interface gráfica

\end{itemize}

\end{description}\end{quote}

\end{fulllineitems}

\index{addMapMarcador() (in module app)@\spxentry{addMapMarcador()}\spxextra{in module app}}

\begin{fulllineitems}
\phantomsection\label{\detokenize{pdcvis:app.addMapMarcador}}
\pysigstartsignatures
\pysiglinewithargsret{\sphinxcode{\sphinxupquote{app.}}\sphinxbfcode{\sphinxupquote{addMapMarcador}}}{\emph{\DUrole{n}{m}}, \emph{\DUrole{n}{camada}}, \emph{\DUrole{n}{camada\_sel}}}{}
\pysigstopsignatures
\sphinxAtStartPar
Adiciona marcados (pontos) num mapa
\begin{quote}\begin{description}
\sphinxlineitem{Parameters}\begin{itemize}
\item {} 
\sphinxAtStartPar
\sphinxstyleliteralstrong{\sphinxupquote{camada}} \textendash{} camada previamente informada com os pontos

\item {} 
\sphinxAtStartPar
\sphinxstyleliteralstrong{\sphinxupquote{camada\_sel}} \textendash{} filtro selecionado na interface gráfica

\end{itemize}

\end{description}\end{quote}

\end{fulllineitems}

\index{addMapVoronoi() (in module app)@\spxentry{addMapVoronoi()}\spxextra{in module app}}

\begin{fulllineitems}
\phantomsection\label{\detokenize{pdcvis:app.addMapVoronoi}}
\pysigstartsignatures
\pysiglinewithargsret{\sphinxcode{\sphinxupquote{app.}}\sphinxbfcode{\sphinxupquote{addMapVoronoi}}}{\emph{\DUrole{n}{m}}, \emph{\DUrole{n}{camada}}, \emph{\DUrole{n}{area\_base}}, \emph{\DUrole{n}{camada\_sel}}}{}
\pysigstopsignatures
\sphinxAtStartPar
A partir dos pontos selecionados, cria um mapa de Voronoi
\begin{quote}\begin{description}
\sphinxlineitem{Parameters}\begin{itemize}
\item {} 
\sphinxAtStartPar
\sphinxstyleliteralstrong{\sphinxupquote{camada}} \textendash{} camada previamente informada com os pontos

\item {} 
\sphinxAtStartPar
\sphinxstyleliteralstrong{\sphinxupquote{area\_base}} \textendash{} geometria base para filtrar os limites do voronoi

\item {} 
\sphinxAtStartPar
\sphinxstyleliteralstrong{\sphinxupquote{camada\_sel}} \textendash{} filtro selecionado na interface gráfica

\end{itemize}

\end{description}\end{quote}

\end{fulllineitems}

\index{loadData() (in module app)@\spxentry{loadData()}\spxextra{in module app}}

\begin{fulllineitems}
\phantomsection\label{\detokenize{pdcvis:app.loadData}}
\pysigstartsignatures
\pysiglinewithargsret{\sphinxcode{\sphinxupquote{app.}}\sphinxbfcode{\sphinxupquote{loadData}}}{\emph{\DUrole{n}{conf}}}{}
\pysigstopsignatures
\sphinxAtStartPar
Load de todos os dados informados no YAML
\begin{quote}\begin{description}
\sphinxlineitem{Parameters}
\sphinxAtStartPar
\sphinxstyleliteralstrong{\sphinxupquote{conf}} \textendash{} dicionário vindo da leitura do YAML

\sphinxlineitem{Return data, georet}
\sphinxAtStartPar
dados e camadas espaciais

\end{description}\end{quote}

\end{fulllineitems}


\sphinxstepscope


\chapter{Exemplos}
\label{\detokenize{exemplos:exemplos}}\label{\detokenize{exemplos::doc}}
\sphinxAtStartPar
Esta seção apresenta alguns exemplos da utilização da API em Notebooks.

\sphinxstepscope


\section{Como criar uma camada}
\label{\detokenize{exemplos/criando_camada:Como-criar-uma-camada}}\label{\detokenize{exemplos/criando_camada::doc}}\begin{itemize}
\item {} 
\sphinxAtStartPar
Exemplo de inserção da camada IVS Ipea São Luís, a partir de um arquivo CSV

\end{itemize}

\begin{sphinxuseclass}{nbinput}
\begin{sphinxuseclass}{nblast}
{
\sphinxsetup{VerbatimColor={named}{nbsphinx-code-bg}}
\sphinxsetup{VerbatimBorderColor={named}{nbsphinx-code-border}}
\begin{sphinxVerbatim}[commandchars=\\\{\}]
\llap{\color{nbsphinxin}[1]:\,\hspace{\fboxrule}\hspace{\fboxsep}}\PYG{k+kn}{import} \PYG{n+nn}{os}
\PYG{k+kn}{import} \PYG{n+nn}{sys}

\PYG{c+c1}{\PYGZsh{}path para a biblioteca do apiModulo. Ajuste de acordo com a necessidade}
\PYG{n}{sys}\PYG{o}{.}\PYG{n}{path}\PYG{o}{.}\PYG{n}{insert}\PYG{p}{(}\PYG{l+m+mi}{0}\PYG{p}{,} \PYG{n}{os}\PYG{o}{.}\PYG{n}{path}\PYG{o}{.}\PYG{n}{abspath}\PYG{p}{(}\PYG{l+s+s1}{\PYGZsq{}}\PYG{l+s+s1}{../../..}\PYG{l+s+s1}{\PYGZsq{}}\PYG{p}{)}\PYG{p}{)}
\PYG{k+kn}{from} \PYG{n+nn}{apiModulo}\PYG{n+nn}{.}\PYG{n+nn}{api} \PYG{k+kn}{import} \PYG{o}{*}
\end{sphinxVerbatim}
}

\end{sphinxuseclass}
\end{sphinxuseclass}
\sphinxAtStartPar
A função inserirCamada deve ser chamada.

\begin{sphinxuseclass}{nbinput}
\begin{sphinxuseclass}{nblast}
{
\sphinxsetup{VerbatimColor={named}{nbsphinx-code-bg}}
\sphinxsetup{VerbatimBorderColor={named}{nbsphinx-code-border}}
\begin{sphinxVerbatim}[commandchars=\\\{\}]
\llap{\color{nbsphinxin}[2]:\,\hspace{\fboxrule}\hspace{\fboxsep}}\PYG{n}{inserirCamada}\PYG{p}{(}
    \PYG{n}{dado} \PYG{o}{=} \PYG{l+s+s1}{\PYGZsq{}}\PYG{l+s+s1}{RM\PYGZus{}Grande\PYGZus{}Sao\PYGZus{}Luis/shape/RM\PYGZus{}SaoLuis\PYGZus{}UDH.shp}\PYG{l+s+s1}{\PYGZsq{}}\PYG{p}{,}
    \PYG{n}{tabela}\PYG{o}{=}\PYG{l+s+s1}{\PYGZsq{}}\PYG{l+s+s1}{ivs\PYGZus{}ipea\PYGZus{}sao\PYGZus{}luis}\PYG{l+s+s1}{\PYGZsq{}}\PYG{p}{,}
    \PYG{n}{campo\PYGZus{}chave}\PYG{o}{=}\PYG{l+s+s1}{\PYGZsq{}}\PYG{l+s+s1}{UDH\PYGZus{}ATLAS}\PYG{l+s+s1}{\PYGZsq{}}\PYG{p}{,}
    \PYG{n}{nome}\PYG{o}{=}\PYG{l+s+s1}{\PYGZsq{}}\PYG{l+s+s1}{udh\PYGZus{}atlas}\PYG{l+s+s1}{\PYGZsq{}}\PYG{p}{,}
    \PYG{n}{descricao}\PYG{o}{=}\PYG{l+s+s1}{\PYGZsq{}}\PYG{l+s+s1}{Shape IPEA Grande São Luís \PYGZhy{} IDH Atlas}\PYG{l+s+s1}{\PYGZsq{}}
\PYG{p}{)}
\end{sphinxVerbatim}
}

\end{sphinxuseclass}
\end{sphinxuseclass}\begin{itemize}
\item {} 
\sphinxAtStartPar
Para checar se a camada foi inserida, você pode usar o comando de \sphinxcode{\sphinxupquote{lstCamadas}}

\end{itemize}

\begin{sphinxuseclass}{nbinput}
{
\sphinxsetup{VerbatimColor={named}{nbsphinx-code-bg}}
\sphinxsetup{VerbatimBorderColor={named}{nbsphinx-code-border}}
\begin{sphinxVerbatim}[commandchars=\\\{\}]
\llap{\color{nbsphinxin}[3]:\,\hspace{\fboxrule}\hspace{\fboxsep}}\PYG{n}{df} \PYG{o}{=} \PYG{n}{lstCamadas}\PYG{p}{(}\PYG{p}{)}
\PYG{n+nb}{print} \PYG{p}{(}\PYG{n}{df}\PYG{o}{.}\PYG{n}{iloc}\PYG{p}{[}\PYG{p}{[}\PYG{l+m+mi}{0}\PYG{p}{,} \PYG{o}{\PYGZhy{}}\PYG{l+m+mi}{1}\PYG{p}{]}\PYG{p}{]}\PYG{p}{)}
\end{sphinxVerbatim}
}

\end{sphinxuseclass}
\begin{sphinxuseclass}{nboutput}
\begin{sphinxuseclass}{nblast}
{

\kern-\sphinxverbatimsmallskipamount\kern-\baselineskip
\kern+\FrameHeightAdjust\kern-\fboxrule
\vspace{\nbsphinxcodecellspacing}

\sphinxsetup{VerbatimColor={named}{white}}
\sphinxsetup{VerbatimBorderColor={named}{nbsphinx-code-border}}
\begin{sphinxuseclass}{output_area}
\begin{sphinxuseclass}{}


\begin{sphinxVerbatim}[commandchars=\\\{\}]
   id\_camada                      nome             tabela
0          1  Setores Censitários IBGE   setor\_censitario
5         42                 udh\_atlas  ivs\_ipea\_sao\_luis
\end{sphinxVerbatim}



\end{sphinxuseclass}
\end{sphinxuseclass}
}

\end{sphinxuseclass}
\end{sphinxuseclass}\begin{itemize}
\item {} 
\sphinxAtStartPar
Tamém pode ser realizado a recuperação dos dados do banco para visualização

\end{itemize}

\begin{sphinxuseclass}{nbinput}
{
\sphinxsetup{VerbatimColor={named}{nbsphinx-code-bg}}
\sphinxsetup{VerbatimBorderColor={named}{nbsphinx-code-border}}
\begin{sphinxVerbatim}[commandchars=\\\{\}]
\llap{\color{nbsphinxin}[4]:\,\hspace{\fboxrule}\hspace{\fboxsep}}\PYG{n}{df} \PYG{o}{=} \PYG{n}{obterCamada}\PYG{p}{(}\PYG{l+s+s1}{\PYGZsq{}}\PYG{l+s+s1}{ivs\PYGZus{}ipea\PYGZus{}sao\PYGZus{}luis}\PYG{l+s+s1}{\PYGZsq{}}\PYG{p}{)}
\PYG{n}{df}\PYG{o}{.}\PYG{n}{plot}\PYG{p}{(}\PYG{p}{)} \PYG{c+c1}{\PYGZsh{}função básica do geopandas}
\end{sphinxVerbatim}
}

\end{sphinxuseclass}
\begin{sphinxuseclass}{nboutput}
{

\kern-\sphinxverbatimsmallskipamount\kern-\baselineskip
\kern+\FrameHeightAdjust\kern-\fboxrule
\vspace{\nbsphinxcodecellspacing}

\sphinxsetup{VerbatimColor={named}{white}}
\sphinxsetup{VerbatimBorderColor={named}{nbsphinx-code-border}}
\begin{sphinxuseclass}{output_area}
\begin{sphinxuseclass}{}


\begin{sphinxVerbatim}[commandchars=\\\{\}]
\llap{\color{nbsphinxout}[4]:\,\hspace{\fboxrule}\hspace{\fboxsep}}<AxesSubplot:>
\end{sphinxVerbatim}



\end{sphinxuseclass}
\end{sphinxuseclass}
}

\end{sphinxuseclass}
\begin{sphinxuseclass}{nboutput}
\begin{sphinxuseclass}{nblast}
\hrule height -\fboxrule\relax
\vspace{\nbsphinxcodecellspacing}

\makeatletter\setbox\nbsphinxpromptbox\box\voidb@x\makeatother

\begin{nbsphinxfancyoutput}

\begin{sphinxuseclass}{output_area}
\begin{sphinxuseclass}{}
\noindent\sphinxincludegraphics[width=462\sphinxpxdimen,height=413\sphinxpxdimen]{{exemplos_criando_camada_7_1}.png}

\end{sphinxuseclass}
\end{sphinxuseclass}
\end{nbsphinxfancyoutput}

\end{sphinxuseclass}
\end{sphinxuseclass}
\sphinxstepscope


\section{Como inserir indicadores}
\label{\detokenize{exemplos/criando_indicadores:Como-inserir-indicadores}}\label{\detokenize{exemplos/criando_indicadores::doc}}\begin{itemize}
\item {} 
\sphinxAtStartPar
Pré\sphinxhyphen{}requisitos
\begin{itemize}
\item {} 
\sphinxAtStartPar
camada {[}se houver referencia espacial para os dados{]} deve estar criada

\item {} 
\sphinxAtStartPar
os dados devem estar em CSV. Uma das colunas deve ser uma referência ao dado espacial

\item {} 
\sphinxAtStartPar
os indicadores devem estar em CSV ou serem inseridos 1 a 1.

\end{itemize}

\item {} 
\sphinxAtStartPar
O CSV dos indicadores deve obedecer o formato deve ser \{tema;assunto;tabela;definicao;descricao;fonte;ano\}

\end{itemize}

\begin{sphinxuseclass}{nbinput}
\begin{sphinxuseclass}{nblast}
{
\sphinxsetup{VerbatimColor={named}{nbsphinx-code-bg}}
\sphinxsetup{VerbatimBorderColor={named}{nbsphinx-code-border}}
\begin{sphinxVerbatim}[commandchars=\\\{\}]
\llap{\color{nbsphinxin}[1]:\,\hspace{\fboxrule}\hspace{\fboxsep}}\PYG{k+kn}{import} \PYG{n+nn}{os}
\PYG{k+kn}{import} \PYG{n+nn}{sys}

\PYG{c+c1}{\PYGZsh{}path para a biblioteca do apiModulo. Ajuste de acordo com a necessidade}
\PYG{n}{sys}\PYG{o}{.}\PYG{n}{path}\PYG{o}{.}\PYG{n}{insert}\PYG{p}{(}\PYG{l+m+mi}{0}\PYG{p}{,} \PYG{n}{os}\PYG{o}{.}\PYG{n}{path}\PYG{o}{.}\PYG{n}{abspath}\PYG{p}{(}\PYG{l+s+s1}{\PYGZsq{}}\PYG{l+s+s1}{../../..}\PYG{l+s+s1}{\PYGZsq{}}\PYG{p}{)}\PYG{p}{)}
\PYG{k+kn}{from} \PYG{n+nn}{apiModulo}\PYG{n+nn}{.}\PYG{n+nn}{api} \PYG{k+kn}{import} \PYG{o}{*}
\end{sphinxVerbatim}
}

\end{sphinxuseclass}
\end{sphinxuseclass}
\begin{sphinxuseclass}{nbinput}
{
\sphinxsetup{VerbatimColor={named}{nbsphinx-code-bg}}
\sphinxsetup{VerbatimBorderColor={named}{nbsphinx-code-border}}
\begin{sphinxVerbatim}[commandchars=\\\{\}]
\llap{\color{nbsphinxin}[2]:\,\hspace{\fboxrule}\hspace{\fboxsep}}\PYG{n}{inserirDados}\PYG{p}{(}\PYG{l+s+s1}{\PYGZsq{}}\PYG{l+s+s1}{RM\PYGZus{}Grande\PYGZus{}Sao\PYGZus{}Luis/atlasivs\PYGZus{}dadosbrutos\PYGZus{}Grande\PYGZus{}Sao\PYGZus{}Luis.csv}\PYG{l+s+s1}{\PYGZsq{}}\PYG{p}{,}
                \PYG{n}{nome\PYGZus{}tabela}\PYG{o}{=}\PYG{l+s+s1}{\PYGZsq{}}\PYG{l+s+s1}{atlas\PYGZus{}udh}\PYG{l+s+s1}{\PYGZsq{}}\PYG{p}{,}
                \PYG{n}{camada}\PYG{o}{=}\PYG{l+s+s1}{\PYGZsq{}}\PYG{l+s+s1}{ivs\PYGZus{}ipea\PYGZus{}sao\PYGZus{}luis}\PYG{l+s+s1}{\PYGZsq{}}\PYG{p}{,}
                \PYG{n}{campo\PYGZus{}camada}\PYG{o}{=}\PYG{l+s+s1}{\PYGZsq{}}\PYG{l+s+s1}{udh}\PYG{l+s+s1}{\PYGZsq{}}\PYG{p}{)}
\end{sphinxVerbatim}
}

\end{sphinxuseclass}
\begin{sphinxuseclass}{nboutput}
\begin{sphinxuseclass}{nblast}
{

\kern-\sphinxverbatimsmallskipamount\kern-\baselineskip
\kern+\FrameHeightAdjust\kern-\fboxrule
\vspace{\nbsphinxcodecellspacing}

\sphinxsetup{VerbatimColor={named}{white}}
\sphinxsetup{VerbatimBorderColor={named}{nbsphinx-code-border}}
\begin{sphinxuseclass}{output_area}
\begin{sphinxuseclass}{}


\begin{sphinxVerbatim}[commandchars=\\\{\}]
AVISO: Linha vazia na coluna 'prosp\_soc': nan
AVISO: acesso de tipos divergentes
AVISO: acesso de tipos divergentes
AVISO: acesso de tipos divergentes
AVISO: acesso de tipos divergentes
AVISO: acesso de tipos divergentes
AVISO: acesso de tipos divergentes
AVISO: acesso de tipos divergentes
AVISO: acesso de tipos divergentes
AVISO: acesso de tipos divergentes
AVISO: acesso de tipos divergentes
AVISO: acesso de tipos divergentes
AVISO: acesso de tipos divergentes
AVISO: acesso de tipos divergentes
AVISO: acesso de tipos divergentes
AVISO: acesso de tipos divergentes
AVISO: acesso de tipos divergentes
AVISO: acesso de tipos divergentes
AVISO: acesso de tipos divergentes
AVISO: acesso de tipos divergentes
AVISO: acesso de tipos divergentes
AVISO: Linha vazia na coluna 'vulner15a24': nan
AVISO: Linha vazia na coluna 'mchefe\_fmenor': nan
AVISO: Linha vazia na coluna 'vulner\_dia': nan
AVISO: Linha vazia na coluna 'dom\_vulner\_idoso': nan
atlas\_udh: Foram importadas 100 colunas e 252 linhas
\end{sphinxVerbatim}



\end{sphinxuseclass}
\end{sphinxuseclass}
}

\end{sphinxuseclass}
\end{sphinxuseclass}
\sphinxAtStartPar
Importando indicadores:

\begin{sphinxuseclass}{nbinput}
{
\sphinxsetup{VerbatimColor={named}{nbsphinx-code-bg}}
\sphinxsetup{VerbatimBorderColor={named}{nbsphinx-code-border}}
\begin{sphinxVerbatim}[commandchars=\\\{\}]
\llap{\color{nbsphinxin}[3]:\,\hspace{\fboxrule}\hspace{\fboxsep}}\PYG{n}{indicadores} \PYG{o}{=} \PYG{n}{pd}\PYG{o}{.}\PYG{n}{read\PYGZus{}csv}\PYG{p}{(}\PYG{l+s+s1}{\PYGZsq{}}\PYG{l+s+s1}{RM\PYGZus{}Grande\PYGZus{}Sao\PYGZus{}Luis/indicadores.csv}\PYG{l+s+s1}{\PYGZsq{}}\PYG{p}{,} \PYG{n}{delimiter}\PYG{o}{=}\PYG{l+s+s1}{\PYGZsq{}}\PYG{l+s+s1}{;}\PYG{l+s+s1}{\PYGZsq{}}\PYG{p}{)}
\PYG{n}{indicadores}\PYG{o}{.}\PYG{n}{head}\PYG{p}{(}\PYG{p}{)}
\end{sphinxVerbatim}
}

\end{sphinxuseclass}
\begin{sphinxuseclass}{nboutput}
\begin{sphinxuseclass}{nblast}
{

\kern-\sphinxverbatimsmallskipamount\kern-\baselineskip
\kern+\FrameHeightAdjust\kern-\fboxrule
\vspace{\nbsphinxcodecellspacing}

\sphinxsetup{VerbatimColor={named}{white}}
\sphinxsetup{VerbatimBorderColor={named}{nbsphinx-code-border}}
\begin{sphinxuseclass}{output_area}
\begin{sphinxuseclass}{}


\begin{sphinxVerbatim}[commandchars=\\\{\}]
\llap{\color{nbsphinxout}[3]:\,\hspace{\fboxrule}\hspace{\fboxsep}}  tema assunto         tabela                  definicao  \textbackslash{}
0  UDH   Atlas  atlas\_vis\_udh                        ivs
1  UDH   Atlas  atlas\_vis\_udh  ivs\_infraestrutura\_urbana
2  UDH   Atlas  atlas\_vis\_udh         ivs\_capital\_humano
3  UDH   Atlas  atlas\_vis\_udh       ivs\_renda\_e\_trabalho
4  UDH   Atlas  atlas\_vis\_udh          t\_sem\_agua\_esgoto

                                           descricao     fonte   ano
0  Índice de Vulnerabilidade Social. Média aritmé{\ldots}  IVS\_IPEA  2010
1  Índice da dimensão Infraestrutura Urbana, é um{\ldots}  IVS\_IPEA  2010
2  Índice da dimensão Capital Humano, é um dos 3 {\ldots}  IVS\_IPEA  2010
3  Índice da dimensão Renda e Trabalho, é um dos {\ldots}  IVS\_IPEA  2010
4  Razão entre o número de pessoas que vivem em d{\ldots}  IVS\_IPEA  2010
\end{sphinxVerbatim}



\end{sphinxuseclass}
\end{sphinxuseclass}
}

\end{sphinxuseclass}
\end{sphinxuseclass}
\sphinxAtStartPar
Obtenha a referência correta da camada:

\begin{sphinxuseclass}{nbinput}
{
\sphinxsetup{VerbatimColor={named}{nbsphinx-code-bg}}
\sphinxsetup{VerbatimBorderColor={named}{nbsphinx-code-border}}
\begin{sphinxVerbatim}[commandchars=\\\{\}]
\llap{\color{nbsphinxin}[5]:\,\hspace{\fboxrule}\hspace{\fboxsep}}\PYG{n}{df} \PYG{o}{=} \PYG{n}{lstCamadas}\PYG{p}{(}\PYG{p}{)}
\PYG{n+nb}{print} \PYG{p}{(}\PYG{n}{df}\PYG{p}{)}
\end{sphinxVerbatim}
}

\end{sphinxuseclass}
\begin{sphinxuseclass}{nboutput}
\begin{sphinxuseclass}{nblast}
{

\kern-\sphinxverbatimsmallskipamount\kern-\baselineskip
\kern+\FrameHeightAdjust\kern-\fboxrule
\vspace{\nbsphinxcodecellspacing}

\sphinxsetup{VerbatimColor={named}{white}}
\sphinxsetup{VerbatimBorderColor={named}{nbsphinx-code-border}}
\begin{sphinxuseclass}{output_area}
\begin{sphinxuseclass}{}


\begin{sphinxVerbatim}[commandchars=\\\{\}]
   id\_camada                      nome             tabela
0          1  Setores Censitários IBGE   setor\_censitario
1          2            Bairros SEMFAZ     bairros\_semfaz
2          3               Anel Viário        anel\_viario
3         23                                        setor
4         38               lotes\_uniao        lotes\_uniao
5         43                 udh\_atlas  ivs\_ipea\_sao\_luis
\end{sphinxVerbatim}



\end{sphinxuseclass}
\end{sphinxuseclass}
}

\end{sphinxuseclass}
\end{sphinxuseclass}
\sphinxAtStartPar
Loop de inserção de todos os indicadores que estavam no CSV

\begin{sphinxuseclass}{nbinput}
\begin{sphinxuseclass}{nblast}
{
\sphinxsetup{VerbatimColor={named}{nbsphinx-code-bg}}
\sphinxsetup{VerbatimBorderColor={named}{nbsphinx-code-border}}
\begin{sphinxVerbatim}[commandchars=\\\{\}]
\llap{\color{nbsphinxin}[6]:\,\hspace{\fboxrule}\hspace{\fboxsep}}\PYG{k}{for} \PYG{n}{index}\PYG{p}{,} \PYG{n}{row} \PYG{o+ow}{in} \PYG{n}{indicadores}\PYG{o}{.}\PYG{n}{iterrows}\PYG{p}{(}\PYG{p}{)}\PYG{p}{:}
    \PYG{n}{inserirIndicador}\PYG{p}{(}\PYG{n}{tema}\PYG{o}{=}\PYG{n}{row}\PYG{p}{[}\PYG{l+s+s2}{\PYGZdq{}}\PYG{l+s+s2}{tema}\PYG{l+s+s2}{\PYGZdq{}}\PYG{p}{]}\PYG{p}{,}
                            \PYG{n}{assunto}\PYG{o}{=}\PYG{n}{row}\PYG{p}{[}\PYG{l+s+s2}{\PYGZdq{}}\PYG{l+s+s2}{assunto}\PYG{l+s+s2}{\PYGZdq{}}\PYG{p}{]}\PYG{p}{,}
                            \PYG{n}{tabela}\PYG{o}{=}\PYG{n}{row}\PYG{p}{[}\PYG{l+s+s2}{\PYGZdq{}}\PYG{l+s+s2}{tabela}\PYG{l+s+s2}{\PYGZdq{}}\PYG{p}{]}\PYG{p}{,}
                            \PYG{n}{definicao}\PYG{o}{=}\PYG{n}{row}\PYG{p}{[}\PYG{l+s+s2}{\PYGZdq{}}\PYG{l+s+s2}{definicao}\PYG{l+s+s2}{\PYGZdq{}}\PYG{p}{]}\PYG{p}{,}
                            \PYG{n}{descricao}\PYG{o}{=}\PYG{n}{row}\PYG{p}{[}\PYG{l+s+s2}{\PYGZdq{}}\PYG{l+s+s2}{descricao}\PYG{l+s+s2}{\PYGZdq{}}\PYG{p}{]}\PYG{p}{,}
                            \PYG{n}{fonte}\PYG{o}{=}\PYG{n}{row}\PYG{p}{[}\PYG{l+s+s2}{\PYGZdq{}}\PYG{l+s+s2}{fonte}\PYG{l+s+s2}{\PYGZdq{}}\PYG{p}{]}\PYG{p}{,}
                            \PYG{n}{ano}\PYG{o}{=}\PYG{n}{row}\PYG{p}{[}\PYG{l+s+s2}{\PYGZdq{}}\PYG{l+s+s2}{ano}\PYG{l+s+s2}{\PYGZdq{}}\PYG{p}{]}\PYG{p}{,}
                            \PYG{n}{camada} \PYG{o}{=} \PYG{l+m+mi}{43}\PYG{p}{)} \PYG{c+c1}{\PYGZsh{}referencia a camada}
\end{sphinxVerbatim}
}

\end{sphinxuseclass}
\end{sphinxuseclass}
\sphinxAtStartPar
Conferindo os indicadores inseridos:

\begin{sphinxuseclass}{nbinput}
{
\sphinxsetup{VerbatimColor={named}{nbsphinx-code-bg}}
\sphinxsetup{VerbatimBorderColor={named}{nbsphinx-code-border}}
\begin{sphinxVerbatim}[commandchars=\\\{\}]
\llap{\color{nbsphinxin}[7]:\,\hspace{\fboxrule}\hspace{\fboxsep}}\PYG{n}{lst} \PYG{o}{=} \PYG{n}{lstIndicador}\PYG{p}{(}\PYG{n}{filtro}\PYG{o}{=}\PYG{l+s+s1}{\PYGZsq{}}\PYG{l+s+s1}{UDH}\PYG{l+s+s1}{\PYGZsq{}}\PYG{p}{)}
\PYG{n}{lst}\PYG{o}{.}\PYG{n}{head}\PYG{p}{(}\PYG{p}{)}
\end{sphinxVerbatim}
}

\end{sphinxuseclass}
\begin{sphinxuseclass}{nboutput}
\begin{sphinxuseclass}{nblast}
{

\kern-\sphinxverbatimsmallskipamount\kern-\baselineskip
\kern+\FrameHeightAdjust\kern-\fboxrule
\vspace{\nbsphinxcodecellspacing}

\sphinxsetup{VerbatimColor={named}{white}}
\sphinxsetup{VerbatimBorderColor={named}{nbsphinx-code-border}}
\begin{sphinxuseclass}{output_area}
\begin{sphinxuseclass}{}


\begin{sphinxVerbatim}[commandchars=\\\{\}]
\llap{\color{nbsphinxout}[7]:\,\hspace{\fboxrule}\hspace{\fboxsep}}       index tema assunto                                          descricao  \textbackslash{}
index
2668    2668  UDH   Atlas  Índice de Vulnerabilidade Social. Média aritmé{\ldots}
2669    2669  UDH   Atlas  Índice da dimensão Infraestrutura Urbana, é um{\ldots}
2670    2670  UDH   Atlas  Índice da dimensão Capital Humano, é um dos 3 {\ldots}
2671    2671  UDH   Atlas  Índice da dimensão Renda e Trabalho, é um dos {\ldots}
2672    2672  UDH   Atlas  Razão entre o número de pessoas que vivem em d{\ldots}

          fonte   ano         tabela                  definicao  \textbackslash{}
index
2668   IVS\_IPEA  2010  atlas\_vis\_udh                        ivs
2669   IVS\_IPEA  2010  atlas\_vis\_udh  ivs\_infraestrutura\_urbana
2670   IVS\_IPEA  2010  atlas\_vis\_udh         ivs\_capital\_humano
2671   IVS\_IPEA  2010  atlas\_vis\_udh       ivs\_renda\_e\_trabalho
2672   IVS\_IPEA  2010  atlas\_vis\_udh          t\_sem\_agua\_esgoto

                  tabela
index
2668   ivs\_ipea\_sao\_luis
2669   ivs\_ipea\_sao\_luis
2670   ivs\_ipea\_sao\_luis
2671   ivs\_ipea\_sao\_luis
2672   ivs\_ipea\_sao\_luis
\end{sphinxVerbatim}



\end{sphinxuseclass}
\end{sphinxuseclass}
}

\end{sphinxuseclass}
\end{sphinxuseclass}
\sphinxAtStartPar
Obtendo valores de do indicador: Índice de Vulnerabilidade Social. Média aritmética dos índices das dimensões: IVS Infraestrutura Urbana, IVS Capital Humano e IVS Renda e Trabalho.

\begin{sphinxVerbatim}[commandchars=\\\{\}]
\PYGZsh{}df, metadados = obterTema(indexes=\PYGZsq{}2668\PYGZsq{})
visMapaIndicador(\PYGZsq{}2668\PYGZsq{}, MAPA\PYGZus{}ZOOM=11)
\end{sphinxVerbatim}

\noindent\sphinxincludegraphics[width=1012\sphinxpxdimen,height=609\sphinxpxdimen]{{saida}.png}

\sphinxstepscope


\section{Como visualizar indicadores}
\label{\detokenize{exemplos/visualizando:Como-visualizar-indicadores}}\label{\detokenize{exemplos/visualizando::doc}}\begin{itemize}
\item {} 
\sphinxAtStartPar
Exemplos de visualização com dados do IBGE

\end{itemize}

\begin{sphinxuseclass}{nbinput}
\begin{sphinxuseclass}{nblast}
{
\sphinxsetup{VerbatimColor={named}{nbsphinx-code-bg}}
\sphinxsetup{VerbatimBorderColor={named}{nbsphinx-code-border}}
\begin{sphinxVerbatim}[commandchars=\\\{\}]
\llap{\color{nbsphinxin}[1]:\,\hspace{\fboxrule}\hspace{\fboxsep}}\PYG{k+kn}{import} \PYG{n+nn}{os}
\PYG{k+kn}{import} \PYG{n+nn}{sys}

\PYG{c+c1}{\PYGZsh{}path para a biblioteca do apiModulo. Ajuste de acordo com a necessidade}
\PYG{n}{sys}\PYG{o}{.}\PYG{n}{path}\PYG{o}{.}\PYG{n}{insert}\PYG{p}{(}\PYG{l+m+mi}{0}\PYG{p}{,} \PYG{n}{os}\PYG{o}{.}\PYG{n}{path}\PYG{o}{.}\PYG{n}{abspath}\PYG{p}{(}\PYG{l+s+s1}{\PYGZsq{}}\PYG{l+s+s1}{../../..}\PYG{l+s+s1}{\PYGZsq{}}\PYG{p}{)}\PYG{p}{)}
\PYG{k+kn}{from} \PYG{n+nn}{apiModulo}\PYG{n+nn}{.}\PYG{n+nn}{api} \PYG{k+kn}{import} \PYG{o}{*}
\end{sphinxVerbatim}
}

\end{sphinxuseclass}
\end{sphinxuseclass}
\sphinxAtStartPar
O primeiro passo é verificar na base quais indicadores estão disponíveis:

\begin{sphinxuseclass}{nbinput}
{
\sphinxsetup{VerbatimColor={named}{nbsphinx-code-bg}}
\sphinxsetup{VerbatimBorderColor={named}{nbsphinx-code-border}}
\begin{sphinxVerbatim}[commandchars=\\\{\}]
\llap{\color{nbsphinxin}[3]:\,\hspace{\fboxrule}\hspace{\fboxsep}}\PYG{n}{indicadores} \PYG{o}{=} \PYG{n}{lstIndicador}\PYG{p}{(}\PYG{n}{filtro}\PYG{o}{=}\PYG{l+s+s1}{\PYGZsq{}}\PYG{l+s+s1}{Mulheres responsáveis}\PYG{l+s+s1}{\PYGZsq{}}\PYG{p}{)}
\PYG{n}{indicadores}
\end{sphinxVerbatim}
}

\end{sphinxuseclass}
\begin{sphinxuseclass}{nboutput}
\begin{sphinxuseclass}{nblast}
{

\kern-\sphinxverbatimsmallskipamount\kern-\baselineskip
\kern+\FrameHeightAdjust\kern-\fboxrule
\vspace{\nbsphinxcodecellspacing}

\sphinxsetup{VerbatimColor={named}{white}}
\sphinxsetup{VerbatimBorderColor={named}{nbsphinx-code-border}}
\begin{sphinxuseclass}{output_area}
\begin{sphinxuseclass}{}


\begin{sphinxVerbatim}[commandchars=\\\{\}]
\llap{\color{nbsphinxout}[3]:\,\hspace{\fboxrule}\hspace{\fboxsep}}       index    tema                           assunto  \textbackslash{}
index
928      928  Pessoa  Alfabetização, Homens e Mulheres
2536    2536  Pessoa                   Idade, mulheres

                                               descricao  \textbackslash{}
index
928    Mulheres responsáveis alfabetizadas com 10 ou {\ldots}
2536     Mulheres responsáveis pelo domicílio particular

                             fonte   ano         tabela definicao  \textbackslash{}
index
928    IBGE Censo Demográfico 2010  2010  ibge\_pessoa02      v154
2536   IBGE Censo Demográfico 2010  2010  ibge\_pessoa13      v003

                 tabela
index
928    setor\_censitario
2536   setor\_censitario
\end{sphinxVerbatim}



\end{sphinxuseclass}
\end{sphinxuseclass}
}

\end{sphinxuseclass}
\end{sphinxuseclass}
\sphinxAtStartPar
Com o inidicador desejado, pegar o valor do index para usar na função seguinte: \sphinxhyphen{} Mulheres responsáveis pelo domicílio particular \sphinxhyphen{} index 2536

\begin{sphinxuseclass}{nbinput}
{
\sphinxsetup{VerbatimColor={named}{nbsphinx-code-bg}}
\sphinxsetup{VerbatimBorderColor={named}{nbsphinx-code-border}}
\begin{sphinxVerbatim}[commandchars=\\\{\}]
\llap{\color{nbsphinxin}[2]:\,\hspace{\fboxrule}\hspace{\fboxsep}}\PYG{n}{visMapaIndicador}\PYG{p}{(}\PYG{l+s+s1}{\PYGZsq{}}\PYG{l+s+s1}{928}\PYG{l+s+s1}{\PYGZsq{}}\PYG{p}{,} \PYG{l+s+s1}{\PYGZsq{}}\PYG{l+s+s1}{800}\PYG{l+s+s1}{\PYGZsq{}}\PYG{p}{,} \PYG{l+s+s1}{\PYGZsq{}}\PYG{l+s+s1}{90}\PYG{l+s+s1}{\PYGZpc{}}\PYG{l+s+s1}{\PYGZsq{}}\PYG{p}{)}
\end{sphinxVerbatim}
}

\end{sphinxuseclass}
\begin{sphinxuseclass}{nboutput}
\begin{sphinxuseclass}{nblast}
{

\kern-\sphinxverbatimsmallskipamount\kern-\baselineskip
\kern+\FrameHeightAdjust\kern-\fboxrule
\vspace{\nbsphinxcodecellspacing}

\sphinxsetup{VerbatimColor={named}{white}}
\sphinxsetup{VerbatimBorderColor={named}{nbsphinx-code-border}}
\begin{sphinxuseclass}{output_area}
\begin{sphinxuseclass}{}


\begin{sphinxVerbatim}[commandchars=\\\{\}]
\llap{\color{nbsphinxout}[2]:\,\hspace{\fboxrule}\hspace{\fboxsep}}<IPython.core.display.HTML object>
\end{sphinxVerbatim}



\end{sphinxuseclass}
\end{sphinxuseclass}
}

\end{sphinxuseclass}
\end{sphinxuseclass}\begin{itemize}
\item {} 
\sphinxAtStartPar
Infraestrutura urbana \sphinxhyphen{} abastecimento de água
\begin{itemize}
\item {} 
\sphinxAtStartPar
13 Domicílios particulares permanentes com abastecimento de água da chuva armazenada em cisterna

\item {} 
\sphinxAtStartPar
11 Domicílios particulares permanentes com abastecimento de água da rede geral

\item {} 
\sphinxAtStartPar
12 Domicílios particulares permanentes com abastecimento de água de poço ou nascente na propriedade

\item {} 
\sphinxAtStartPar
14 Domicílios particulares permanentes com outra forma de abastecimento de água

\end{itemize}

\end{itemize}

\begin{sphinxuseclass}{nbinput}
{
\sphinxsetup{VerbatimColor={named}{nbsphinx-code-bg}}
\sphinxsetup{VerbatimBorderColor={named}{nbsphinx-code-border}}
\begin{sphinxVerbatim}[commandchars=\\\{\}]
\llap{\color{nbsphinxin}[4]:\,\hspace{\fboxrule}\hspace{\fboxsep}}\PYG{n}{visMapaIndicador}\PYG{p}{(}\PYG{l+s+s1}{\PYGZsq{}}\PYG{l+s+s1}{11,12,13,14}\PYG{l+s+s1}{\PYGZsq{}}\PYG{p}{,} \PYG{l+s+s1}{\PYGZsq{}}\PYG{l+s+s1}{600}\PYG{l+s+s1}{\PYGZsq{}}\PYG{p}{,} \PYG{l+s+s1}{\PYGZsq{}}\PYG{l+s+s1}{40}\PYG{l+s+s1}{\PYGZpc{}}\PYG{l+s+s1}{\PYGZsq{}}\PYG{p}{)}

\end{sphinxVerbatim}
}

\end{sphinxuseclass}
\begin{sphinxuseclass}{nboutput}
\begin{sphinxuseclass}{nblast}
{

\kern-\sphinxverbatimsmallskipamount\kern-\baselineskip
\kern+\FrameHeightAdjust\kern-\fboxrule
\vspace{\nbsphinxcodecellspacing}

\sphinxsetup{VerbatimColor={named}{white}}
\sphinxsetup{VerbatimBorderColor={named}{nbsphinx-code-border}}
\begin{sphinxuseclass}{output_area}
\begin{sphinxuseclass}{}


\begin{sphinxVerbatim}[commandchars=\\\{\}]
\llap{\color{nbsphinxout}[4]:\,\hspace{\fboxrule}\hspace{\fboxsep}}<IPython.core.display.HTML object>
\end{sphinxVerbatim}



\end{sphinxuseclass}
\end{sphinxuseclass}
}

\end{sphinxuseclass}
\end{sphinxuseclass}\begin{itemize}
\item {} 
\sphinxAtStartPar
Mulheres (14 anos ou mais) =
\begin{itemize}
\item {} 
\sphinxAtStartPar
Soma(2667,2581,2582,2583,2584,2585,2586,2587,2588,2589,2590,2591,2592,2593,2594,2595,2596,2597,2598, 2599,2600,2601,2602,2603,2604,2605,2606,2607,2608,2609,2610,2611,2612,2613,2614,2615,2616,2617,2618,2619, 2620,2621,2622,2623,2624,2625,2626,2627,2628,2629,2630,2631,2632,2633,2634,2635,2636,2637,2638,2639,2640, 2641,2642,2643,2644,2645,2646,2647,2648,2649,2650,2651,2652,2653,2654,2655,2656,2657,2658,2659,2660,2661, 2662,2663,2664,2665,2666)

\end{itemize}

\end{itemize}

\begin{sphinxuseclass}{nbinput}
{
\sphinxsetup{VerbatimColor={named}{nbsphinx-code-bg}}
\sphinxsetup{VerbatimBorderColor={named}{nbsphinx-code-border}}
\begin{sphinxVerbatim}[commandchars=\\\{\}]
\llap{\color{nbsphinxin}[2]:\,\hspace{\fboxrule}\hspace{\fboxsep}}\PYG{n}{mulheres\PYGZus{}14\PYGZus{}ou\PYGZus{}mais}\PYG{p}{,} \PYG{n}{metadados} \PYG{o}{=} \PYG{n}{obterTema}\PYG{p}{(}\PYG{n}{indexes}\PYG{o}{=}\PYG{l+s+s1}{\PYGZsq{}}\PYG{l+s+s1}{2667,2581,2582,2583,2584,2585,2586,2587,2588,2589,2590,2591,2592,2593,2594,2595,2596,2597,2598,2599,2600,2601,2602,2603,2604,2605,2606,2607,2608,2609,2610,2611,2612,2613,2614,2615,2616,2617,2618,2619,2620,2621,2622,2623,2624,2625,2626,2627,2628,2629,2630,2631,2632,2633,2634,2635,2636,2637,2638,2639,2640,2641,2642,2643,2644,2645,2646,2647,2648,2649,2650,2651,2652,2653,2654,2655,2656,2657,2658,2659,2660,2661,2662,2663,2664,2665,2666}\PYG{l+s+s1}{\PYGZsq{}}\PYG{p}{)}
\PYG{n}{mulheres\PYGZus{}14\PYGZus{}ou\PYGZus{}mais} \PYG{o}{=} \PYG{n}{limparDados}\PYG{p}{(}\PYG{n}{mulheres\PYGZus{}14\PYGZus{}ou\PYGZus{}mais}\PYG{p}{)}

\PYG{n}{dados} \PYG{o}{=} \PYG{n}{somarColunas}\PYG{p}{(}\PYG{n}{mulheres\PYGZus{}14\PYGZus{}ou\PYGZus{}mais}\PYG{p}{)}
\PYG{n}{visMapaGJson}\PYG{p}{(}\PYG{n}{dados}\PYG{p}{,} \PYG{l+s+s1}{\PYGZsq{}}\PYG{l+s+s1}{soma}\PYG{l+s+s1}{\PYGZsq{}}\PYG{p}{,} \PYG{l+s+s1}{\PYGZsq{}}\PYG{l+s+s1}{Mulheres (14 anos ou mais)}\PYG{l+s+s1}{\PYGZsq{}}\PYG{p}{,} \PYG{l+s+s1}{\PYGZsq{}}\PYG{l+s+s1}{600}\PYG{l+s+s1}{\PYGZsq{}}\PYG{p}{,} \PYG{l+s+s1}{\PYGZsq{}}\PYG{l+s+s1}{90}\PYG{l+s+s1}{\PYGZpc{}}\PYG{l+s+s1}{\PYGZsq{}}\PYG{p}{)}
\end{sphinxVerbatim}
}

\end{sphinxuseclass}
\begin{sphinxuseclass}{nboutput}
\begin{sphinxuseclass}{nblast}
{

\kern-\sphinxverbatimsmallskipamount\kern-\baselineskip
\kern+\FrameHeightAdjust\kern-\fboxrule
\vspace{\nbsphinxcodecellspacing}

\sphinxsetup{VerbatimColor={named}{white}}
\sphinxsetup{VerbatimBorderColor={named}{nbsphinx-code-border}}
\begin{sphinxuseclass}{output_area}
\begin{sphinxuseclass}{}


\begin{sphinxVerbatim}[commandchars=\\\{\}]
\llap{\color{nbsphinxout}[2]:\,\hspace{\fboxrule}\hspace{\fboxsep}}<IPython.core.display.HTML object>
\end{sphinxVerbatim}



\end{sphinxuseclass}
\end{sphinxuseclass}
}

\end{sphinxuseclass}
\end{sphinxuseclass}
\sphinxstepscope


\section{Como fazer uma junção espacial}
\label{\detokenize{exemplos/cruzamento_lotes:Como-fazer-uma-jun_xe7_xe3o-espacial}}\label{\detokenize{exemplos/cruzamento_lotes::doc}}\begin{itemize}
\item {} 
\sphinxAtStartPar
Usando a API para fazer a junção de dois arquivos de Lotes que não tem correspondência numérica

\item {} 
\sphinxAtStartPar
Inicialmente vamos ler os arquivos .SHP

\end{itemize}

\begin{sphinxuseclass}{nbinput}
{
\sphinxsetup{VerbatimColor={named}{nbsphinx-code-bg}}
\sphinxsetup{VerbatimBorderColor={named}{nbsphinx-code-border}}
\begin{sphinxVerbatim}[commandchars=\\\{\}]
\llap{\color{nbsphinxin}[1]:\,\hspace{\fboxrule}\hspace{\fboxsep}}\PYG{k+kn}{import} \PYG{n+nn}{os}
\PYG{k+kn}{import} \PYG{n+nn}{sys}

\PYG{c+c1}{\PYGZsh{}path para a biblioteca do apiModulo. Ajuste de acordo com a necessidade}
\PYG{n}{sys}\PYG{o}{.}\PYG{n}{path}\PYG{o}{.}\PYG{n}{insert}\PYG{p}{(}\PYG{l+m+mi}{0}\PYG{p}{,} \PYG{n}{os}\PYG{o}{.}\PYG{n}{path}\PYG{o}{.}\PYG{n}{abspath}\PYG{p}{(}\PYG{l+s+s1}{\PYGZsq{}}\PYG{l+s+s1}{../../..}\PYG{l+s+s1}{\PYGZsq{}}\PYG{p}{)}\PYG{p}{)}
\PYG{k+kn}{from} \PYG{n+nn}{apiModulo}\PYG{n+nn}{.}\PYG{n+nn}{api} \PYG{k+kn}{import} \PYG{o}{*}

\PYG{n}{semfaz} \PYG{o}{=} \PYG{n}{lerArquivo}\PYG{p}{(}\PYG{l+s+s1}{\PYGZsq{}}\PYG{l+s+s1}{LOTES/LOTES\PYGZus{}FINAIS\PYGZus{}2.shp}\PYG{l+s+s1}{\PYGZsq{}}\PYG{p}{)}
\PYG{n}{htl} \PYG{o}{=} \PYG{n}{lerArquivo}\PYG{p}{(}\PYG{l+s+s1}{\PYGZsq{}}\PYG{l+s+s1}{LOTES/SLM\PYGZus{}LOTES\PYGZus{}USO\PYGZus{}DO\PYGZus{}SOLO\PYGZus{}2020\PYGZus{}2021.shp}\PYG{l+s+s1}{\PYGZsq{}}\PYG{p}{)}

\PYG{n+nb}{print}\PYG{p}{(}\PYG{l+s+sa}{f}\PYG{l+s+s2}{\PYGZdq{}}\PYG{l+s+s2}{Feições em SEMFAZ: }\PYG{l+s+si}{\PYGZob{}}\PYG{n}{semfaz}\PYG{o}{.}\PYG{n}{shape}\PYG{p}{[}\PYG{l+m+mi}{0}\PYG{p}{]}\PYG{l+s+si}{\PYGZcb{}}\PYG{l+s+s2}{\PYGZdq{}}\PYG{p}{)}
\PYG{n+nb}{print}\PYG{p}{(}\PYG{l+s+sa}{f}\PYG{l+s+s2}{\PYGZdq{}}\PYG{l+s+s2}{Feições em HTL: }\PYG{l+s+si}{\PYGZob{}}\PYG{n}{htl}\PYG{o}{.}\PYG{n}{shape}\PYG{p}{[}\PYG{l+m+mi}{0}\PYG{p}{]}\PYG{l+s+si}{\PYGZcb{}}\PYG{l+s+s2}{\PYGZdq{}}\PYG{p}{)}
\end{sphinxVerbatim}
}

\end{sphinxuseclass}
\begin{sphinxuseclass}{nboutput}
\begin{sphinxuseclass}{nblast}
{

\kern-\sphinxverbatimsmallskipamount\kern-\baselineskip
\kern+\FrameHeightAdjust\kern-\fboxrule
\vspace{\nbsphinxcodecellspacing}

\sphinxsetup{VerbatimColor={named}{white}}
\sphinxsetup{VerbatimBorderColor={named}{nbsphinx-code-border}}
\begin{sphinxuseclass}{output_area}
\begin{sphinxuseclass}{}


\begin{sphinxVerbatim}[commandchars=\\\{\}]
Feições em SEMFAZ: 12871
Feições em HTL: 12290
\end{sphinxVerbatim}



\end{sphinxuseclass}
\end{sphinxuseclass}
}

\end{sphinxuseclass}
\end{sphinxuseclass}\begin{itemize}
\item {} 
\sphinxAtStartPar
Como não existe correspondência entre os dois arquivos, vamos calcular o centroide da geometria do arquivo HTL
\begin{itemize}
\item {} 
\sphinxAtStartPar
Caso queira visualizar o resultado da conversão dos polígono em ponto, você pode usar: visMapaSemIndicador(dados=htl, coluna\_geom=‘geometry’)

\end{itemize}

\end{itemize}

\begin{sphinxuseclass}{nbinput}
\begin{sphinxuseclass}{nblast}
{
\sphinxsetup{VerbatimColor={named}{nbsphinx-code-bg}}
\sphinxsetup{VerbatimBorderColor={named}{nbsphinx-code-border}}
\begin{sphinxVerbatim}[commandchars=\\\{\}]
\llap{\color{nbsphinxin}[2]:\,\hspace{\fboxrule}\hspace{\fboxsep}}\PYG{n}{htl}\PYG{p}{[}\PYG{l+s+s1}{\PYGZsq{}}\PYG{l+s+s1}{geometry}\PYG{l+s+s1}{\PYGZsq{}}\PYG{p}{]} \PYG{o}{=} \PYG{n}{htl}\PYG{p}{[}\PYG{l+s+s1}{\PYGZsq{}}\PYG{l+s+s1}{geometry}\PYG{l+s+s1}{\PYGZsq{}}\PYG{p}{]}\PYG{o}{.}\PYG{n}{centroid}
\end{sphinxVerbatim}
}

\end{sphinxuseclass}
\end{sphinxuseclass}\begin{itemize}
\item {} 
\sphinxAtStartPar
Com os centroides para uma coleção de geometria calculado, podemos obter a junção espacial entre eles

\item {} 
\sphinxAtStartPar
Para isso vamos usar o predicado = contains e a forma = inner

\end{itemize}

\begin{sphinxuseclass}{nbinput}
\begin{sphinxuseclass}{nblast}
{
\sphinxsetup{VerbatimColor={named}{nbsphinx-code-bg}}
\sphinxsetup{VerbatimBorderColor={named}{nbsphinx-code-border}}
\begin{sphinxVerbatim}[commandchars=\\\{\}]
\llap{\color{nbsphinxin}[4]:\,\hspace{\fboxrule}\hspace{\fboxsep}}\PYG{n}{novos\PYGZus{}lotes} \PYG{o}{=} \PYG{n}{juncaoEspacial}\PYG{p}{(}\PYG{n}{semfaz}\PYG{p}{,} \PYG{n}{htl}\PYG{p}{,} \PYG{n}{tipo}\PYG{o}{=}\PYG{l+s+s2}{\PYGZdq{}}\PYG{l+s+s2}{inner}\PYG{l+s+s2}{\PYGZdq{}}\PYG{p}{,} \PYG{n}{predicado}\PYG{o}{=}\PYG{l+s+s1}{\PYGZsq{}}\PYG{l+s+s1}{contains}\PYG{l+s+s1}{\PYGZsq{}}\PYG{p}{)}
\end{sphinxVerbatim}
}

\end{sphinxuseclass}
\end{sphinxuseclass}\begin{itemize}
\item {} 
\sphinxAtStartPar
Para checar o resultado podemos:
\begin{itemize}
\item {} 
\sphinxAtStartPar
imprimir as colunas e verificar que o resultado possui colunas dos dois arquivos originais

\end{itemize}

\end{itemize}

\begin{sphinxuseclass}{nbinput}
{
\sphinxsetup{VerbatimColor={named}{nbsphinx-code-bg}}
\sphinxsetup{VerbatimBorderColor={named}{nbsphinx-code-border}}
\begin{sphinxVerbatim}[commandchars=\\\{\}]
\llap{\color{nbsphinxin}[5]:\,\hspace{\fboxrule}\hspace{\fboxsep}}\PYG{n}{novos\PYGZus{}lotes}\PYG{o}{.}\PYG{n}{columns}
\end{sphinxVerbatim}
}

\end{sphinxuseclass}
\begin{sphinxuseclass}{nboutput}
\begin{sphinxuseclass}{nblast}
{

\kern-\sphinxverbatimsmallskipamount\kern-\baselineskip
\kern+\FrameHeightAdjust\kern-\fboxrule
\vspace{\nbsphinxcodecellspacing}

\sphinxsetup{VerbatimColor={named}{white}}
\sphinxsetup{VerbatimBorderColor={named}{nbsphinx-code-border}}
\begin{sphinxuseclass}{output_area}
\begin{sphinxuseclass}{}


\begin{sphinxVerbatim}[commandchars=\\\{\}]
\llap{\color{nbsphinxout}[5]:\,\hspace{\fboxrule}\hspace{\fboxsep}}Index(['OBJECTID\_1', 'SETOR', 'QUADRA', 'DISTRITO', 'SETOR\_NOV', 'QDR\_NPV',
       'GEOCODE', 'INS\_ANT', 'AREA', 'Shape\_Leng', 'de\_observa', 'de\_geocodi',
       'DE\_USO', 'DE\_TIPOLOG', 'GEOCODE\_SE', 'Shape\_STAr', 'Shape\_STLe',
       'PRIOR', 'NR', 'N\_IMOVEL', 'FOTO', 'TIPO', 'TIPOLOGIA', 'INSCRICAO\_',
       'LOGRADOURO', 'NUMERO\_left', 'BAIRRO', 'TOMBAMENTO', 'ZONA\_DE\_IN',
       'NUMERO\_CAR', 'DOMINIO', 'PROPRIEDAD', 'CONDENACAO', 'NUMERO\_DO\_',
       'DIVIDA\_DE\_', 'NUMERO\_DO1', 'SITUACAO', 'STATUS\_OCU', 'STATUS\_SIT',
       'USO\_left', 'ESTADO\_DE\_', 'OBSERVACAO', 'OBSERVAC\_1', 'PROJETO',
       'ETAPA', 'OBRA', 'USO1', 'ENTE', 'OBSERVAC\_2', 'PRIORIDADE',
       'GEOCODE\_1', 'DIVIDA', 'NR\_1', 'VIS\_1', 'CONJ', 'CONJ\_PREF',
       'USO\_POTENC', 'USO\_PROPOS', 'OBSERV', 'geometry', 'index\_right',
       'ID\_JOIN3', 'COD\_LOT', 'GABARITO', 'USO\_right', 'TIPOLOGIAS', 'CONSERV',
       'OCIOSIDA', 'VOCACAO', 'NUMERO\_right', 'TP\_COLETA', 'ESTILO',
       'GOOGLE\_ANO', 'AREA\_M2', 'ID\_PA', 'PAISAGEM\_C', 'UN\_PAISAGE',
       'check\_in', 'HA\_LOT'],
      dtype='object')
\end{sphinxVerbatim}



\end{sphinxuseclass}
\end{sphinxuseclass}
}

\end{sphinxuseclass}
\end{sphinxuseclass}\begin{itemize}
\item {} 
\sphinxAtStartPar
Podemos ainda escolher uma dessas colunas para visualizar.
\begin{itemize}
\item {} 
\sphinxAtStartPar
nesse exemplo, escolhemos CONSERV

\end{itemize}

\end{itemize}

\begin{sphinxuseclass}{nbinput}
{
\sphinxsetup{VerbatimColor={named}{nbsphinx-code-bg}}
\sphinxsetup{VerbatimBorderColor={named}{nbsphinx-code-border}}
\begin{sphinxVerbatim}[commandchars=\\\{\}]
\llap{\color{nbsphinxin}[6]:\,\hspace{\fboxrule}\hspace{\fboxsep}}\PYG{n}{visMapaGJson}\PYG{p}{(}\PYG{n}{novos\PYGZus{}lotes}\PYG{p}{,}
                    \PYG{n}{variavel}\PYG{o}{=}\PYG{l+s+s1}{\PYGZsq{}}\PYG{l+s+s1}{CONSERV}\PYG{l+s+s1}{\PYGZsq{}}\PYG{p}{,}
                    \PYG{n}{descricao}\PYG{o}{=}\PYG{l+s+s1}{\PYGZsq{}}\PYG{l+s+s1}{Uso}\PYG{l+s+s1}{\PYGZsq{}}\PYG{p}{,}
                    \PYG{n}{height}\PYG{o}{=}\PYG{l+s+s1}{\PYGZsq{}}\PYG{l+s+s1}{800}\PYG{l+s+s1}{\PYGZsq{}}\PYG{p}{,}
                    \PYG{n}{width}\PYG{o}{=}\PYG{l+s+s1}{\PYGZsq{}}\PYG{l+s+s1}{90}\PYG{l+s+s1}{\PYGZpc{}}\PYG{l+s+s1}{\PYGZsq{}}\PYG{p}{)}
\end{sphinxVerbatim}
}

\end{sphinxuseclass}
\begin{sphinxuseclass}{nboutput}
\begin{sphinxuseclass}{nblast}
{

\kern-\sphinxverbatimsmallskipamount\kern-\baselineskip
\kern+\FrameHeightAdjust\kern-\fboxrule
\vspace{\nbsphinxcodecellspacing}

\sphinxsetup{VerbatimColor={named}{white}}
\sphinxsetup{VerbatimBorderColor={named}{nbsphinx-code-border}}
\begin{sphinxuseclass}{output_area}
\begin{sphinxuseclass}{}


\begin{sphinxVerbatim}[commandchars=\\\{\}]
\llap{\color{nbsphinxout}[6]:\,\hspace{\fboxrule}\hspace{\fboxsep}}<IPython.core.display.HTML object>
\end{sphinxVerbatim}



\end{sphinxuseclass}
\end{sphinxuseclass}
}

\end{sphinxuseclass}
\end{sphinxuseclass}\begin{itemize}
\item {} 
\sphinxAtStartPar
Caso queira salvar o resultado, escolha:
\begin{itemize}
\item {} 
\sphinxAtStartPar
salvar em arquivo: salvar(novos\_lotes, ‘novos\_lotes.shp’)

\item {} 
\sphinxAtStartPar
salvar na base como uma nova análise:

\end{itemize}

\end{itemize}

\begin{sphinxVerbatim}[commandchars=\\\{\}]
inserirCamada(
    dado = novos\PYGZus{}lotes,
    tabela= novos\PYGZus{}lotes,
    \PYGZsh{}campo\PYGZus{}chave=\PYGZsq{}OBJECTID\PYGZus{}1\PYGZsq{},
    nome=\PYGZsq{}novos\PYGZus{}lotes\PYGZsq{},
    descricao=\PYGZsq{}Camada de análise de lotes\PYGZsq{}
)
\end{sphinxVerbatim}

\sphinxstepscope


\section{Análise de Infraestrutura de Domicílios}
\label{\detokenize{exemplos/analise_domicilios:An_xe1lise-de-Infraestrutura-de-Domic_xedlios}}\label{\detokenize{exemplos/analise_domicilios::doc}}
\sphinxAtStartPar
Vamos começar obtendo os dados de interesse (Fonte IBGE 2010):
\begin{itemize}
\item {} 
\sphinxAtStartPar
0 Domicílios particulares e domicílios coletivos

\item {} 
\sphinxAtStartPar
11 Domicílios particulares permanentes com abastecimento de água da rede geral

\item {} 
\sphinxAtStartPar
15 Domicílios particulares permanentes com banheiro de uso exclusivo dos moradores ou sanitário

\item {} 
\sphinxAtStartPar
16 Domicílios particulares permanentes com banheiro de uso exclusivo dos moradores ou sanitário e esgotamento sanitário via rede geral de esgoto ou pluvial

\item {} 
\sphinxAtStartPar
22 Domicílios particulares permanentes sem banheiro de uso exclusivo dos moradores e nem sanitário

\item {} 
\sphinxAtStartPar
23 Domicílios particulares permanentes com banheiro de uso exclusivo dos moradores

\item {} 
\sphinxAtStartPar
33 Domicílios particulares permanentes sem banheiro de uso exclusivo dos moradores

\item {} 
\sphinxAtStartPar
35 Domicílios particulares permanentes com lixo coletado por serviço de limpeza

\item {} 
\sphinxAtStartPar
43 Domicílios particulares permanentes com energia elétrica de companhia distribuidora

\end{itemize}

\begin{sphinxuseclass}{nbinput}
\begin{sphinxuseclass}{nblast}
{
\sphinxsetup{VerbatimColor={named}{nbsphinx-code-bg}}
\sphinxsetup{VerbatimBorderColor={named}{nbsphinx-code-border}}
\begin{sphinxVerbatim}[commandchars=\\\{\}]
\llap{\color{nbsphinxin}[1]:\,\hspace{\fboxrule}\hspace{\fboxsep}}\PYG{k+kn}{import} \PYG{n+nn}{os}
\PYG{k+kn}{import} \PYG{n+nn}{sys}

\PYG{c+c1}{\PYGZsh{}path para a biblioteca do apiModulo. Ajuste de acordo com a necessidade}
\PYG{n}{sys}\PYG{o}{.}\PYG{n}{path}\PYG{o}{.}\PYG{n}{insert}\PYG{p}{(}\PYG{l+m+mi}{0}\PYG{p}{,} \PYG{n}{os}\PYG{o}{.}\PYG{n}{path}\PYG{o}{.}\PYG{n}{abspath}\PYG{p}{(}\PYG{l+s+s1}{\PYGZsq{}}\PYG{l+s+s1}{../../..}\PYG{l+s+s1}{\PYGZsq{}}\PYG{p}{)}\PYG{p}{)}
\PYG{k+kn}{from} \PYG{n+nn}{apiModulo}\PYG{n+nn}{.}\PYG{n+nn}{api} \PYG{k+kn}{import} \PYG{o}{*}
\end{sphinxVerbatim}
}

\end{sphinxuseclass}
\end{sphinxuseclass}

\subsection{Pré\sphinxhyphen{}requisitos:}
\label{\detokenize{exemplos/analise_domicilios:Pr_xe9-requisitos:}}\begin{itemize}
\item {} 
\sphinxAtStartPar
Usaremos o campo “Domicílios particulares e domicílios coletivos” para normalizar todos os demais valores e analisar de acordo com a proporção de domicílio.

\item {} 
\sphinxAtStartPar
Vamos filtrar as geometrias para a área de estudo do Anél Viário

\end{itemize}

\begin{sphinxuseclass}{nbinput}
\begin{sphinxuseclass}{nblast}
{
\sphinxsetup{VerbatimColor={named}{nbsphinx-code-bg}}
\sphinxsetup{VerbatimBorderColor={named}{nbsphinx-code-border}}
\begin{sphinxVerbatim}[commandchars=\\\{\}]
\llap{\color{nbsphinxin}[2]:\,\hspace{\fboxrule}\hspace{\fboxsep}}\PYG{c+c1}{\PYGZsh{}obtém a região de análise do centro de São Luís}
\PYG{n}{centro} \PYG{o}{=} \PYG{n}{obterCamada}\PYG{p}{(}\PYG{l+s+s1}{\PYGZsq{}}\PYG{l+s+s1}{anel\PYGZus{}viario}\PYG{l+s+s1}{\PYGZsq{}}\PYG{p}{,} \PYG{n}{simples}\PYG{o}{=}\PYG{k+kc}{True}\PYG{p}{)}

\PYG{n}{total}\PYG{p}{,} \PYG{n}{m} \PYG{o}{=} \PYG{n}{obterIndicador}\PYG{p}{(}\PYG{l+m+mi}{0}\PYG{p}{)}
\PYG{n}{total} \PYG{o}{=} \PYG{n}{limparDados}\PYG{p}{(}\PYG{n}{total}\PYG{p}{)} \PYG{c+c1}{\PYGZsh{}limpeza de valores não definidos}
\PYG{n}{total} \PYG{o}{=} \PYG{n}{total}\PYG{o}{.}\PYG{n}{rename}\PYG{p}{(}\PYG{n}{columns}\PYG{o}{=}\PYG{p}{\PYGZob{}}\PYG{l+s+s2}{\PYGZdq{}}\PYG{l+s+s2}{ibge\PYGZus{}domicilio01\PYGZus{}v001}\PYG{l+s+s2}{\PYGZdq{}}\PYG{p}{:}\PYG{l+s+s2}{\PYGZdq{}}\PYG{l+s+s2}{total}\PYG{l+s+s2}{\PYGZdq{}}\PYG{p}{\PYGZcb{}}\PYG{p}{)}
\PYG{n}{total} \PYG{o}{=} \PYG{n}{juncaoEspacial}\PYG{p}{(}\PYG{n}{total}\PYG{p}{,} \PYG{n}{centro}\PYG{p}{,} \PYG{n}{tipo}\PYG{o}{=}\PYG{l+s+s1}{\PYGZsq{}}\PYG{l+s+s1}{inner}\PYG{l+s+s1}{\PYGZsq{}}\PYG{p}{,} \PYG{n}{predicado}\PYG{o}{=}\PYG{l+s+s1}{\PYGZsq{}}\PYG{l+s+s1}{intersects}\PYG{l+s+s1}{\PYGZsq{}}\PYG{p}{)}
\PYG{n}{total} \PYG{o}{=} \PYG{n}{total}\PYG{o}{.}\PYG{n}{drop}\PYG{p}{(}\PYG{n}{columns}\PYG{o}{=}\PYG{p}{[}\PYG{l+s+s1}{\PYGZsq{}}\PYG{l+s+s1}{geometria}\PYG{l+s+s1}{\PYGZsq{}}\PYG{p}{]}\PYG{p}{)}
\end{sphinxVerbatim}
}

\end{sphinxuseclass}
\end{sphinxuseclass}

\subsection{Infraestrutura em termos de abastecimento de água:}
\label{\detokenize{exemplos/analise_domicilios:Infraestrutura-em-termos-de-abastecimento-de-_xe1gua:}}\begin{itemize}
\item {} 
\sphinxAtStartPar
11 Domicílios particulares permanentes com abastecimento de água da rede geral

\end{itemize}

\begin{sphinxuseclass}{nbinput}
{
\sphinxsetup{VerbatimColor={named}{nbsphinx-code-bg}}
\sphinxsetup{VerbatimBorderColor={named}{nbsphinx-code-border}}
\begin{sphinxVerbatim}[commandchars=\\\{\}]
\llap{\color{nbsphinxin}[3]:\,\hspace{\fboxrule}\hspace{\fboxsep}}\PYG{k+kn}{import} \PYG{n+nn}{pprint}

\PYG{n}{agua}\PYG{p}{,} \PYG{n}{m} \PYG{o}{=} \PYG{n}{obterIndicador}\PYG{p}{(}\PYG{l+m+mi}{11}\PYG{p}{)}
\PYG{n}{pprint}\PYG{o}{.}\PYG{n}{pprint}\PYG{p}{(}\PYG{n}{m}\PYG{p}{)} \PYG{c+c1}{\PYGZsh{}imprime as variáveis presentes na informação}

\PYG{c+c1}{\PYGZsh{}obtendo dados apenas para a área de estudo}
\PYG{n}{agua} \PYG{o}{=} \PYG{n}{juncaoEspacial}\PYG{p}{(}\PYG{n}{agua}\PYG{p}{,} \PYG{n}{centro}\PYG{p}{,} \PYG{n}{tipo}\PYG{o}{=}\PYG{l+s+s1}{\PYGZsq{}}\PYG{l+s+s1}{inner}\PYG{l+s+s1}{\PYGZsq{}}\PYG{p}{,} \PYG{n}{predicado}\PYG{o}{=}\PYG{l+s+s1}{\PYGZsq{}}\PYG{l+s+s1}{intersects}\PYG{l+s+s1}{\PYGZsq{}}\PYG{p}{)}
\PYG{n}{agua} \PYG{o}{=} \PYG{n}{limparDados}\PYG{p}{(}\PYG{n}{agua}\PYG{p}{)} \PYG{c+c1}{\PYGZsh{}limpeza de valores não definidos}

\PYG{n}{agua} \PYG{o}{=} \PYG{n}{agua}\PYG{o}{.}\PYG{n}{merge}\PYG{p}{(}\PYG{n}{total}\PYG{p}{,} \PYG{n}{on}\PYG{o}{=}\PYG{l+s+s2}{\PYGZdq{}}\PYG{l+s+s2}{index}\PYG{l+s+s2}{\PYGZdq{}}\PYG{p}{)}
\PYG{c+c1}{\PYGZsh{}normaliza pela quantidade de domicíliios na análise}
\PYG{n}{agua}\PYG{p}{[}\PYG{l+s+s1}{\PYGZsq{}}\PYG{l+s+s1}{ibge\PYGZus{}domicilio01\PYGZus{}v012}\PYG{l+s+s1}{\PYGZsq{}}\PYG{p}{]} \PYG{o}{=} \PYG{p}{(}\PYG{n}{agua}\PYG{p}{[}\PYG{l+s+s1}{\PYGZsq{}}\PYG{l+s+s1}{ibge\PYGZus{}domicilio01\PYGZus{}v012}\PYG{l+s+s1}{\PYGZsq{}}\PYG{p}{]}\PYG{o}{/}\PYG{n}{agua}\PYG{p}{[}\PYG{l+s+s1}{\PYGZsq{}}\PYG{l+s+s1}{total}\PYG{l+s+s1}{\PYGZsq{}}\PYG{p}{]}\PYG{p}{)}\PYG{o}{*}\PYG{l+m+mi}{100}
\end{sphinxVerbatim}
}

\end{sphinxuseclass}
\begin{sphinxuseclass}{nboutput}
\begin{sphinxuseclass}{nblast}
{

\kern-\sphinxverbatimsmallskipamount\kern-\baselineskip
\kern+\FrameHeightAdjust\kern-\fboxrule
\vspace{\nbsphinxcodecellspacing}

\sphinxsetup{VerbatimColor={named}{white}}
\sphinxsetup{VerbatimBorderColor={named}{nbsphinx-code-border}}
\begin{sphinxuseclass}{output_area}
\begin{sphinxuseclass}{}


\begin{sphinxVerbatim}[commandchars=\\\{\}]
[\{'descrição': 'Domicílios particulares permanentes com abastecimento de água '
               'da rede geral',
  'suporte espacial': 'setor\_censitario',
  'tabela': 'ibge\_domicilio01',
  'variável': 'ibge\_domicilio01\_v012'\}]
\end{sphinxVerbatim}



\end{sphinxuseclass}
\end{sphinxuseclass}
}

\end{sphinxuseclass}
\end{sphinxuseclass}
\begin{sphinxuseclass}{nbinput}
{
\sphinxsetup{VerbatimColor={named}{nbsphinx-code-bg}}
\sphinxsetup{VerbatimBorderColor={named}{nbsphinx-code-border}}
\begin{sphinxVerbatim}[commandchars=\\\{\}]
\llap{\color{nbsphinxin}[4]:\,\hspace{\fboxrule}\hspace{\fboxsep}}\PYG{n}{visMapaGJson}\PYG{p}{(}\PYG{n}{agua}\PYG{p}{,}
        \PYG{n}{variavel}\PYG{o}{=}\PYG{l+s+s1}{\PYGZsq{}}\PYG{l+s+s1}{ibge\PYGZus{}domicilio01\PYGZus{}v012}\PYG{l+s+s1}{\PYGZsq{}}\PYG{p}{,}
        \PYG{n}{descricao}\PYG{o}{=}\PYG{l+s+s1}{\PYGZsq{}}\PYG{l+s+s1}{Domicílios particulares permanentes com abastecimento de água da rede geral}\PYG{l+s+s1}{\PYGZsq{}}\PYG{p}{,}
        \PYG{n}{width}\PYG{o}{=}\PYG{l+s+s1}{\PYGZsq{}}\PYG{l+s+s1}{90}\PYG{l+s+s1}{\PYGZpc{}}\PYG{l+s+s1}{\PYGZsq{}}\PYG{p}{)}
\end{sphinxVerbatim}
}

\end{sphinxuseclass}
\begin{sphinxuseclass}{nboutput}
\begin{sphinxuseclass}{nblast}
{

\kern-\sphinxverbatimsmallskipamount\kern-\baselineskip
\kern+\FrameHeightAdjust\kern-\fboxrule
\vspace{\nbsphinxcodecellspacing}

\sphinxsetup{VerbatimColor={named}{white}}
\sphinxsetup{VerbatimBorderColor={named}{nbsphinx-code-border}}
\begin{sphinxuseclass}{output_area}
\begin{sphinxuseclass}{}


\begin{sphinxVerbatim}[commandchars=\\\{\}]
\llap{\color{nbsphinxout}[4]:\,\hspace{\fboxrule}\hspace{\fboxsep}}<IPython.core.display.HTML object>
\end{sphinxVerbatim}



\end{sphinxuseclass}
\end{sphinxuseclass}
}

\end{sphinxuseclass}
\end{sphinxuseclass}
\sphinxAtStartPar
As condições domiciliares, avaliadas a partir de indicadores e variáveis de acesso a serviços básicos como água encanada e instalação sanitária, verificamos que a absoluta maioria dos domicílios da região tem abastecimento por água encanada da rede geral. As localidades com menor percentual obtém taxas por volta de 80\% e merecem atenção.


\subsection{Infraestrutura sanitária:}
\label{\detokenize{exemplos/analise_domicilios:Infraestrutura-sanit_xe1ria:}}\begin{itemize}
\item {} 
\sphinxAtStartPar
15 Domicílios particulares permanentes com banheiro de uso exclusivo dos moradores ou sanitário

\item {} 
\sphinxAtStartPar
16 Domicílios particulares permanentes com banheiro de uso exclusivo dos moradores ou sanitário e esgotamento sanitário via rede geral de esgoto ou pluvial

\item {} 
\sphinxAtStartPar
22 Domicílios particulares permanentes sem banheiro de uso exclusivo dos moradores e nem sanitário

\item {} 
\sphinxAtStartPar
23 Domicílios particulares permanentes com banheiro de uso exclusivo dos moradores

\item {} 
\sphinxAtStartPar
33 Domicílios particulares permanentes sem banheiro de uso exclusivo dos moradores

\end{itemize}

\begin{sphinxuseclass}{nbinput}
{
\sphinxsetup{VerbatimColor={named}{nbsphinx-code-bg}}
\sphinxsetup{VerbatimBorderColor={named}{nbsphinx-code-border}}
\begin{sphinxVerbatim}[commandchars=\\\{\}]
\llap{\color{nbsphinxin}[5]:\,\hspace{\fboxrule}\hspace{\fboxsep}}\PYG{n}{banheiro}\PYG{p}{,} \PYG{n}{m} \PYG{o}{=} \PYG{n}{obterTema}\PYG{p}{(}\PYG{n}{indexes}\PYG{o}{=}\PYG{l+s+s1}{\PYGZsq{}}\PYG{l+s+s1}{15, 16, 22, 23, 33}\PYG{l+s+s1}{\PYGZsq{}}\PYG{p}{)}
\PYG{n}{pprint}\PYG{o}{.}\PYG{n}{pprint}\PYG{p}{(}\PYG{n}{m}\PYG{p}{)} \PYG{c+c1}{\PYGZsh{}imprime as variáveis presentes na informação}

\PYG{c+c1}{\PYGZsh{}obtendo dados apenas para a área de estudo}
\PYG{n}{banheiro} \PYG{o}{=} \PYG{n}{juncaoEspacial}\PYG{p}{(}\PYG{n}{banheiro}\PYG{p}{,} \PYG{n}{centro}\PYG{p}{,} \PYG{n}{tipo}\PYG{o}{=}\PYG{l+s+s1}{\PYGZsq{}}\PYG{l+s+s1}{inner}\PYG{l+s+s1}{\PYGZsq{}}\PYG{p}{,} \PYG{n}{predicado}\PYG{o}{=}\PYG{l+s+s1}{\PYGZsq{}}\PYG{l+s+s1}{intersects}\PYG{l+s+s1}{\PYGZsq{}}\PYG{p}{)}
\PYG{n}{banheiro} \PYG{o}{=} \PYG{n}{limparDados}\PYG{p}{(}\PYG{n}{banheiro}\PYG{p}{)} \PYG{c+c1}{\PYGZsh{}limpeza de valores não definidos}

\PYG{n}{banheiro} \PYG{o}{=} \PYG{n}{banheiro}\PYG{o}{.}\PYG{n}{merge}\PYG{p}{(}\PYG{n}{total}\PYG{p}{,} \PYG{n}{on}\PYG{o}{=}\PYG{l+s+s2}{\PYGZdq{}}\PYG{l+s+s2}{index}\PYG{l+s+s2}{\PYGZdq{}}\PYG{p}{)}

\end{sphinxVerbatim}
}

\end{sphinxuseclass}
\begin{sphinxuseclass}{nboutput}
\begin{sphinxuseclass}{nblast}
{

\kern-\sphinxverbatimsmallskipamount\kern-\baselineskip
\kern+\FrameHeightAdjust\kern-\fboxrule
\vspace{\nbsphinxcodecellspacing}

\sphinxsetup{VerbatimColor={named}{white}}
\sphinxsetup{VerbatimBorderColor={named}{nbsphinx-code-border}}
\begin{sphinxuseclass}{output_area}
\begin{sphinxuseclass}{}


\begin{sphinxVerbatim}[commandchars=\\\{\}]
[\{'descrição': 'Domicílios particulares permanentes com banheiro de uso '
               'exclusivo dos moradores ou sanitário',
  'suporte espacial': 'setor\_censitario',
  'tabela': 'ibge\_domicilio01',
  'variável': 'ibge\_domicilio01\_v016'\},
 \{'descrição': 'Domicílios particulares permanentes com banheiro de uso '
               'exclusivo dos moradores ou sanitário e esgotamento sanitário '
               'via rede geral de esgoto ou pluvial',
  'suporte espacial': 'setor\_censitario',
  'tabela': 'ibge\_domicilio01',
  'variável': 'ibge\_domicilio01\_v017'\},
 \{'descrição': 'Domicílios particulares permanentes sem banheiro de uso '
               'exclusivo dos moradores e nem sanitário',
  'suporte espacial': 'setor\_censitario',
  'tabela': 'ibge\_domicilio01',
  'variável': 'ibge\_domicilio01\_v023'\},
 \{'descrição': 'Domicílios particulares permanentes com banheiro de uso '
               'exclusivo dos moradores',
  'suporte espacial': 'setor\_censitario',
  'tabela': 'ibge\_domicilio01',
  'variável': 'ibge\_domicilio01\_v024'\},
 \{'descrição': 'Domicílios particulares permanentes sem banheiro de uso '
               'exclusivo dos moradores',
  'suporte espacial': 'setor\_censitario',
  'tabela': 'ibge\_domicilio01',
  'variável': 'ibge\_domicilio01\_v034'\}]
\end{sphinxVerbatim}



\end{sphinxuseclass}
\end{sphinxuseclass}
}

\end{sphinxuseclass}
\end{sphinxuseclass}
\begin{sphinxuseclass}{nbinput}
\begin{sphinxuseclass}{nblast}
{
\sphinxsetup{VerbatimColor={named}{nbsphinx-code-bg}}
\sphinxsetup{VerbatimBorderColor={named}{nbsphinx-code-border}}
\begin{sphinxVerbatim}[commandchars=\\\{\}]
\llap{\color{nbsphinxin}[6]:\,\hspace{\fboxrule}\hspace{\fboxsep}}\PYG{c+c1}{\PYGZsh{}normaliza pela quantidade de domicíliios na análise}
\PYG{n}{banheiro}\PYG{p}{[}\PYG{l+s+s1}{\PYGZsq{}}\PYG{l+s+s1}{ibge\PYGZus{}domicilio01\PYGZus{}v016}\PYG{l+s+s1}{\PYGZsq{}}\PYG{p}{]} \PYG{o}{=} \PYG{p}{(}\PYG{n}{banheiro}\PYG{p}{[}\PYG{l+s+s1}{\PYGZsq{}}\PYG{l+s+s1}{ibge\PYGZus{}domicilio01\PYGZus{}v016}\PYG{l+s+s1}{\PYGZsq{}}\PYG{p}{]}\PYG{o}{/}\PYG{n}{banheiro}\PYG{p}{[}\PYG{l+s+s1}{\PYGZsq{}}\PYG{l+s+s1}{total}\PYG{l+s+s1}{\PYGZsq{}}\PYG{p}{]}\PYG{p}{)}\PYG{o}{*}\PYG{l+m+mi}{100}
\PYG{n}{banheiro}\PYG{p}{[}\PYG{l+s+s1}{\PYGZsq{}}\PYG{l+s+s1}{ibge\PYGZus{}domicilio01\PYGZus{}v017}\PYG{l+s+s1}{\PYGZsq{}}\PYG{p}{]} \PYG{o}{=} \PYG{p}{(}\PYG{n}{banheiro}\PYG{p}{[}\PYG{l+s+s1}{\PYGZsq{}}\PYG{l+s+s1}{ibge\PYGZus{}domicilio01\PYGZus{}v017}\PYG{l+s+s1}{\PYGZsq{}}\PYG{p}{]}\PYG{o}{/}\PYG{n}{banheiro}\PYG{p}{[}\PYG{l+s+s1}{\PYGZsq{}}\PYG{l+s+s1}{total}\PYG{l+s+s1}{\PYGZsq{}}\PYG{p}{]}\PYG{p}{)}\PYG{o}{*}\PYG{l+m+mi}{100}
\PYG{n}{banheiro}\PYG{p}{[}\PYG{l+s+s1}{\PYGZsq{}}\PYG{l+s+s1}{ibge\PYGZus{}domicilio01\PYGZus{}v023}\PYG{l+s+s1}{\PYGZsq{}}\PYG{p}{]} \PYG{o}{=} \PYG{p}{(}\PYG{n}{banheiro}\PYG{p}{[}\PYG{l+s+s1}{\PYGZsq{}}\PYG{l+s+s1}{ibge\PYGZus{}domicilio01\PYGZus{}v023}\PYG{l+s+s1}{\PYGZsq{}}\PYG{p}{]}\PYG{o}{/}\PYG{n}{banheiro}\PYG{p}{[}\PYG{l+s+s1}{\PYGZsq{}}\PYG{l+s+s1}{total}\PYG{l+s+s1}{\PYGZsq{}}\PYG{p}{]}\PYG{p}{)}\PYG{o}{*}\PYG{l+m+mi}{100}
\PYG{n}{banheiro}\PYG{p}{[}\PYG{l+s+s1}{\PYGZsq{}}\PYG{l+s+s1}{ibge\PYGZus{}domicilio01\PYGZus{}v024}\PYG{l+s+s1}{\PYGZsq{}}\PYG{p}{]} \PYG{o}{=} \PYG{p}{(}\PYG{n}{banheiro}\PYG{p}{[}\PYG{l+s+s1}{\PYGZsq{}}\PYG{l+s+s1}{ibge\PYGZus{}domicilio01\PYGZus{}v024}\PYG{l+s+s1}{\PYGZsq{}}\PYG{p}{]}\PYG{o}{/}\PYG{n}{banheiro}\PYG{p}{[}\PYG{l+s+s1}{\PYGZsq{}}\PYG{l+s+s1}{total}\PYG{l+s+s1}{\PYGZsq{}}\PYG{p}{]}\PYG{p}{)}\PYG{o}{*}\PYG{l+m+mi}{100}
\PYG{n}{banheiro}\PYG{p}{[}\PYG{l+s+s1}{\PYGZsq{}}\PYG{l+s+s1}{ibge\PYGZus{}domicilio01\PYGZus{}v034}\PYG{l+s+s1}{\PYGZsq{}}\PYG{p}{]} \PYG{o}{=} \PYG{p}{(}\PYG{n}{banheiro}\PYG{p}{[}\PYG{l+s+s1}{\PYGZsq{}}\PYG{l+s+s1}{ibge\PYGZus{}domicilio01\PYGZus{}v034}\PYG{l+s+s1}{\PYGZsq{}}\PYG{p}{]}\PYG{o}{/}\PYG{n}{banheiro}\PYG{p}{[}\PYG{l+s+s1}{\PYGZsq{}}\PYG{l+s+s1}{total}\PYG{l+s+s1}{\PYGZsq{}}\PYG{p}{]}\PYG{p}{)}\PYG{o}{*}\PYG{l+m+mi}{100}
\end{sphinxVerbatim}
}

\end{sphinxuseclass}
\end{sphinxuseclass}
\begin{sphinxuseclass}{nbinput}
{
\sphinxsetup{VerbatimColor={named}{nbsphinx-code-bg}}
\sphinxsetup{VerbatimBorderColor={named}{nbsphinx-code-border}}
\begin{sphinxVerbatim}[commandchars=\\\{\}]
\llap{\color{nbsphinxin}[10]:\,\hspace{\fboxrule}\hspace{\fboxsep}}\PYG{n}{visMapaDados}\PYG{p}{(}\PYG{n}{banheiro}\PYG{p}{,} \PYG{n}{m}\PYG{p}{,} \PYG{n}{width}\PYG{o}{=}\PYG{l+s+s1}{\PYGZsq{}}\PYG{l+s+s1}{45}\PYG{l+s+s1}{\PYGZpc{}}\PYG{l+s+s1}{\PYGZsq{}}\PYG{p}{,} \PYG{n}{MAPA\PYGZus{}ZOOM}\PYG{o}{=}\PYG{l+m+mi}{14}\PYG{p}{)}

\end{sphinxVerbatim}
}

\end{sphinxuseclass}
\begin{sphinxuseclass}{nboutput}
\begin{sphinxuseclass}{nblast}
{

\kern-\sphinxverbatimsmallskipamount\kern-\baselineskip
\kern+\FrameHeightAdjust\kern-\fboxrule
\vspace{\nbsphinxcodecellspacing}

\sphinxsetup{VerbatimColor={named}{white}}
\sphinxsetup{VerbatimBorderColor={named}{nbsphinx-code-border}}
\begin{sphinxuseclass}{output_area}
\begin{sphinxuseclass}{}


\begin{sphinxVerbatim}[commandchars=\\\{\}]
\llap{\color{nbsphinxout}[10]:\,\hspace{\fboxrule}\hspace{\fboxsep}}<IPython.core.display.HTML object>
\end{sphinxVerbatim}



\end{sphinxuseclass}
\end{sphinxuseclass}
}

\end{sphinxuseclass}
\end{sphinxuseclass}
\sphinxAtStartPar
A maior parte dos domicílios possuem banheiro, de uso exclusivo dos moradores. Entretanto, existem alguns setores com uma taxa de até 37\% sem banheiro (em amarelo) e ainda com baixa cobertura de esgotamento sanitário.


\subsection{Infraestrutura de coleta de lixo:}
\label{\detokenize{exemplos/analise_domicilios:Infraestrutura-de-coleta-de-lixo:}}\begin{itemize}
\item {} 
\sphinxAtStartPar
35 Domicílios particulares permanentes com lixo coletado por serviço de limpeza

\end{itemize}

\begin{sphinxuseclass}{nbinput}
{
\sphinxsetup{VerbatimColor={named}{nbsphinx-code-bg}}
\sphinxsetup{VerbatimBorderColor={named}{nbsphinx-code-border}}
\begin{sphinxVerbatim}[commandchars=\\\{\}]
\llap{\color{nbsphinxin}[11]:\,\hspace{\fboxrule}\hspace{\fboxsep}}\PYG{n}{lixo}\PYG{p}{,} \PYG{n}{m} \PYG{o}{=} \PYG{n}{obterTema}\PYG{p}{(}\PYG{n}{indexes}\PYG{o}{=}\PYG{l+s+s1}{\PYGZsq{}}\PYG{l+s+s1}{35}\PYG{l+s+s1}{\PYGZsq{}}\PYG{p}{)}
\PYG{n}{pprint}\PYG{o}{.}\PYG{n}{pprint}\PYG{p}{(}\PYG{n}{m}\PYG{p}{)} \PYG{c+c1}{\PYGZsh{}imprime as variáveis presentes na informação}

\PYG{c+c1}{\PYGZsh{}obtendo dados apenas para a área de estudo}
\PYG{n}{lixo} \PYG{o}{=} \PYG{n}{juncaoEspacial}\PYG{p}{(}\PYG{n}{lixo}\PYG{p}{,} \PYG{n}{centro}\PYG{p}{,} \PYG{n}{tipo}\PYG{o}{=}\PYG{l+s+s1}{\PYGZsq{}}\PYG{l+s+s1}{inner}\PYG{l+s+s1}{\PYGZsq{}}\PYG{p}{,} \PYG{n}{predicado}\PYG{o}{=}\PYG{l+s+s1}{\PYGZsq{}}\PYG{l+s+s1}{intersects}\PYG{l+s+s1}{\PYGZsq{}}\PYG{p}{)}
\PYG{n}{lixo} \PYG{o}{=} \PYG{n}{limparDados}\PYG{p}{(}\PYG{n}{lixo}\PYG{p}{)} \PYG{c+c1}{\PYGZsh{}limpeza de valores não definidos}

\PYG{n}{lixo} \PYG{o}{=} \PYG{n}{lixo}\PYG{o}{.}\PYG{n}{merge}\PYG{p}{(}\PYG{n}{total}\PYG{p}{,} \PYG{n}{on}\PYG{o}{=}\PYG{l+s+s2}{\PYGZdq{}}\PYG{l+s+s2}{index}\PYG{l+s+s2}{\PYGZdq{}}\PYG{p}{)}

\end{sphinxVerbatim}
}

\end{sphinxuseclass}
\begin{sphinxuseclass}{nboutput}
\begin{sphinxuseclass}{nblast}
{

\kern-\sphinxverbatimsmallskipamount\kern-\baselineskip
\kern+\FrameHeightAdjust\kern-\fboxrule
\vspace{\nbsphinxcodecellspacing}

\sphinxsetup{VerbatimColor={named}{white}}
\sphinxsetup{VerbatimBorderColor={named}{nbsphinx-code-border}}
\begin{sphinxuseclass}{output_area}
\begin{sphinxuseclass}{}


\begin{sphinxVerbatim}[commandchars=\\\{\}]
[\{'descrição': 'Domicílios particulares permanentes com lixo coletado por '
               'serviço de limpeza',
  'suporte espacial': 'setor\_censitario',
  'tabela': 'ibge\_domicilio01',
  'variável': 'ibge\_domicilio01\_v036'\}]
\end{sphinxVerbatim}



\end{sphinxuseclass}
\end{sphinxuseclass}
}

\end{sphinxuseclass}
\end{sphinxuseclass}
\begin{sphinxuseclass}{nbinput}
{
\sphinxsetup{VerbatimColor={named}{nbsphinx-code-bg}}
\sphinxsetup{VerbatimBorderColor={named}{nbsphinx-code-border}}
\begin{sphinxVerbatim}[commandchars=\\\{\}]
\llap{\color{nbsphinxin}[14]:\,\hspace{\fboxrule}\hspace{\fboxsep}}\PYG{n}{lixo}\PYG{p}{[}\PYG{l+s+s1}{\PYGZsq{}}\PYG{l+s+s1}{ibge\PYGZus{}domicilio01\PYGZus{}v036}\PYG{l+s+s1}{\PYGZsq{}}\PYG{p}{]} \PYG{o}{=} \PYG{p}{(}\PYG{n}{lixo}\PYG{p}{[}\PYG{l+s+s1}{\PYGZsq{}}\PYG{l+s+s1}{ibge\PYGZus{}domicilio01\PYGZus{}v036}\PYG{l+s+s1}{\PYGZsq{}}\PYG{p}{]}\PYG{o}{/}\PYG{n}{lixo}\PYG{p}{[}\PYG{l+s+s1}{\PYGZsq{}}\PYG{l+s+s1}{total}\PYG{l+s+s1}{\PYGZsq{}}\PYG{p}{]}\PYG{p}{)}\PYG{o}{*}\PYG{l+m+mi}{100}
\PYG{n}{visMapaGJson}\PYG{p}{(}\PYG{n}{lixo}\PYG{p}{,}
        \PYG{n}{variavel}\PYG{o}{=}\PYG{l+s+s1}{\PYGZsq{}}\PYG{l+s+s1}{ibge\PYGZus{}domicilio01\PYGZus{}v036}\PYG{l+s+s1}{\PYGZsq{}}\PYG{p}{,}
        \PYG{n}{descricao}\PYG{o}{=}\PYG{l+s+s1}{\PYGZsq{}}\PYG{l+s+s1}{Domicílios particulares permanentes com lixo coletado por serviço de limpeza}\PYG{l+s+s1}{\PYGZsq{}}\PYG{p}{,}
        \PYG{n}{width}\PYG{o}{=}\PYG{l+s+s1}{\PYGZsq{}}\PYG{l+s+s1}{90}\PYG{l+s+s1}{\PYGZpc{}}\PYG{l+s+s1}{\PYGZsq{}}\PYG{p}{)}

\end{sphinxVerbatim}
}

\end{sphinxuseclass}
\begin{sphinxuseclass}{nboutput}
\begin{sphinxuseclass}{nblast}
{

\kern-\sphinxverbatimsmallskipamount\kern-\baselineskip
\kern+\FrameHeightAdjust\kern-\fboxrule
\vspace{\nbsphinxcodecellspacing}

\sphinxsetup{VerbatimColor={named}{white}}
\sphinxsetup{VerbatimBorderColor={named}{nbsphinx-code-border}}
\begin{sphinxuseclass}{output_area}
\begin{sphinxuseclass}{}


\begin{sphinxVerbatim}[commandchars=\\\{\}]
\llap{\color{nbsphinxout}[14]:\,\hspace{\fboxrule}\hspace{\fboxsep}}<IPython.core.display.HTML object>
\end{sphinxVerbatim}



\end{sphinxuseclass}
\end{sphinxuseclass}
}

\end{sphinxuseclass}
\end{sphinxuseclass}
\sphinxAtStartPar
Observa\sphinxhyphen{}se que as áreas ao redor das avenidas possuem cobertura de lixo coletado entre 88 e 100\%, entretando existem área mais próximas à Liberdade onde esse índice chega a cair para 59\%


\subsection{Infraestrutura de energia:}
\label{\detokenize{exemplos/analise_domicilios:Infraestrutura-de-energia:}}\begin{itemize}
\item {} 
\sphinxAtStartPar
43 Domicílios particulares permanentes com energia elétrica de companhia distribuidora

\end{itemize}

\begin{sphinxuseclass}{nbinput}
{
\sphinxsetup{VerbatimColor={named}{nbsphinx-code-bg}}
\sphinxsetup{VerbatimBorderColor={named}{nbsphinx-code-border}}
\begin{sphinxVerbatim}[commandchars=\\\{\}]
\llap{\color{nbsphinxin}[15]:\,\hspace{\fboxrule}\hspace{\fboxsep}}\PYG{n}{energia}\PYG{p}{,} \PYG{n}{m} \PYG{o}{=} \PYG{n}{obterTema}\PYG{p}{(}\PYG{n}{indexes}\PYG{o}{=}\PYG{l+s+s1}{\PYGZsq{}}\PYG{l+s+s1}{43}\PYG{l+s+s1}{\PYGZsq{}}\PYG{p}{)}
\PYG{n}{pprint}\PYG{o}{.}\PYG{n}{pprint}\PYG{p}{(}\PYG{n}{m}\PYG{p}{)} \PYG{c+c1}{\PYGZsh{}imprime as variáveis presentes na informação}

\PYG{c+c1}{\PYGZsh{}obtendo dados apenas para a área de estudo}
\PYG{n}{energia} \PYG{o}{=} \PYG{n}{juncaoEspacial}\PYG{p}{(}\PYG{n}{energia}\PYG{p}{,} \PYG{n}{centro}\PYG{p}{,} \PYG{n}{tipo}\PYG{o}{=}\PYG{l+s+s1}{\PYGZsq{}}\PYG{l+s+s1}{inner}\PYG{l+s+s1}{\PYGZsq{}}\PYG{p}{,} \PYG{n}{predicado}\PYG{o}{=}\PYG{l+s+s1}{\PYGZsq{}}\PYG{l+s+s1}{intersects}\PYG{l+s+s1}{\PYGZsq{}}\PYG{p}{)}
\PYG{n}{energia} \PYG{o}{=} \PYG{n}{limparDados}\PYG{p}{(}\PYG{n}{energia}\PYG{p}{)} \PYG{c+c1}{\PYGZsh{}limpeza de valores não definidos}

\PYG{n}{energia} \PYG{o}{=} \PYG{n}{energia}\PYG{o}{.}\PYG{n}{merge}\PYG{p}{(}\PYG{n}{total}\PYG{p}{,} \PYG{n}{on}\PYG{o}{=}\PYG{l+s+s2}{\PYGZdq{}}\PYG{l+s+s2}{index}\PYG{l+s+s2}{\PYGZdq{}}\PYG{p}{)}

\end{sphinxVerbatim}
}

\end{sphinxuseclass}
\begin{sphinxuseclass}{nboutput}
\begin{sphinxuseclass}{nblast}
{

\kern-\sphinxverbatimsmallskipamount\kern-\baselineskip
\kern+\FrameHeightAdjust\kern-\fboxrule
\vspace{\nbsphinxcodecellspacing}

\sphinxsetup{VerbatimColor={named}{white}}
\sphinxsetup{VerbatimBorderColor={named}{nbsphinx-code-border}}
\begin{sphinxuseclass}{output_area}
\begin{sphinxuseclass}{}


\begin{sphinxVerbatim}[commandchars=\\\{\}]
[\{'descrição': 'Domicílios particulares permanentes com energia elétrica de '
               'companhia distribuidora',
  'suporte espacial': 'setor\_censitario',
  'tabela': 'ibge\_domicilio01',
  'variável': 'ibge\_domicilio01\_v044'\}]
\end{sphinxVerbatim}



\end{sphinxuseclass}
\end{sphinxuseclass}
}

\end{sphinxuseclass}
\end{sphinxuseclass}
\begin{sphinxuseclass}{nbinput}
{
\sphinxsetup{VerbatimColor={named}{nbsphinx-code-bg}}
\sphinxsetup{VerbatimBorderColor={named}{nbsphinx-code-border}}
\begin{sphinxVerbatim}[commandchars=\\\{\}]
\llap{\color{nbsphinxin}[16]:\,\hspace{\fboxrule}\hspace{\fboxsep}}\PYG{n}{energia}\PYG{p}{[}\PYG{l+s+s1}{\PYGZsq{}}\PYG{l+s+s1}{ibge\PYGZus{}domicilio01\PYGZus{}v044}\PYG{l+s+s1}{\PYGZsq{}}\PYG{p}{]} \PYG{o}{=} \PYG{p}{(}\PYG{n}{energia}\PYG{p}{[}\PYG{l+s+s1}{\PYGZsq{}}\PYG{l+s+s1}{ibge\PYGZus{}domicilio01\PYGZus{}v044}\PYG{l+s+s1}{\PYGZsq{}}\PYG{p}{]}\PYG{o}{/}\PYG{n}{energia}\PYG{p}{[}\PYG{l+s+s1}{\PYGZsq{}}\PYG{l+s+s1}{total}\PYG{l+s+s1}{\PYGZsq{}}\PYG{p}{]}\PYG{p}{)}\PYG{o}{*}\PYG{l+m+mi}{100}
\PYG{n}{visMapaGJson}\PYG{p}{(}\PYG{n}{energia}\PYG{p}{,}
        \PYG{n}{variavel}\PYG{o}{=}\PYG{l+s+s1}{\PYGZsq{}}\PYG{l+s+s1}{ibge\PYGZus{}domicilio01\PYGZus{}v044}\PYG{l+s+s1}{\PYGZsq{}}\PYG{p}{,}
        \PYG{n}{descricao}\PYG{o}{=}\PYG{l+s+s1}{\PYGZsq{}}\PYG{l+s+s1}{Domicílios particulares permanentes com energia elétrica de companhia distribuidora}\PYG{l+s+s1}{\PYGZsq{}}\PYG{p}{,}
        \PYG{n}{width}\PYG{o}{=}\PYG{l+s+s1}{\PYGZsq{}}\PYG{l+s+s1}{90}\PYG{l+s+s1}{\PYGZpc{}}\PYG{l+s+s1}{\PYGZsq{}}\PYG{p}{)}

\end{sphinxVerbatim}
}

\end{sphinxuseclass}
\begin{sphinxuseclass}{nboutput}
\begin{sphinxuseclass}{nblast}
{

\kern-\sphinxverbatimsmallskipamount\kern-\baselineskip
\kern+\FrameHeightAdjust\kern-\fboxrule
\vspace{\nbsphinxcodecellspacing}

\sphinxsetup{VerbatimColor={named}{white}}
\sphinxsetup{VerbatimBorderColor={named}{nbsphinx-code-border}}
\begin{sphinxuseclass}{output_area}
\begin{sphinxuseclass}{}


\begin{sphinxVerbatim}[commandchars=\\\{\}]
\llap{\color{nbsphinxout}[16]:\,\hspace{\fboxrule}\hspace{\fboxsep}}<IPython.core.display.HTML object>
\end{sphinxVerbatim}



\end{sphinxuseclass}
\end{sphinxuseclass}
}

\end{sphinxuseclass}
\end{sphinxuseclass}
\sphinxAtStartPar
Chama a atenção as localidades onde a cobertura de energia elétrica pela distribuidora é inferior à 98\% visto que se trata de uma área de grande urbanização da cidade.


\subsection{Índice de Infraestrutura (IPEA IVS UDH)}
\label{\detokenize{exemplos/analise_domicilios:_xcdndice-de-Infraestrutura-(IPEA-IVS-UDH)}}
\sphinxAtStartPar
As dimensões do Índice de Vulnerabilidade Social (IVS) permitem a identificação de regiões do espaço urbano onde há sobreposição de condições que podem ser associadas à vulnerabilidade social. Deste modo, esses índices se constituem ferramentas úteis para o planejamento de políticas públicas voltadas para lidar melhor com as fragilidades e vulnerabilidades que se apresenta nesses territórios.

\sphinxAtStartPar
A dimensão infraestrutura urbana (IVS\sphinxhyphen{}I) é obtido através da média ponderada como: Percentual da população que vive em domicílios urbanos sem o serviço de coleta de lixo (peso: 0,300); Percentual de pessoas em domicílios com abastecimento de água e esgotamento sanitário inadequados (peso: 0,300); Percentual de pessoas em domicílios vulneráveis à pobreza e que gastam mais de uma hora até o trabalho no total de pessoas ocupadas, vulneráveis e que retornam diariamente do trabalho (peso: 0,400).

\sphinxAtStartPar
Analisando o IVS\sphinxhyphen{}I para a região em estudo verificamos que a mesmo se encontra no intervalo entre 0,35 e 0,42, pelo pode ser constatado que do ponto de vista da infraestrutura urbana a área de atuação do Programa se encontra na faixa de média a alta vulnerabilidade.

\begin{sphinxuseclass}{nbinput}
{
\sphinxsetup{VerbatimColor={named}{nbsphinx-code-bg}}
\sphinxsetup{VerbatimBorderColor={named}{nbsphinx-code-border}}
\begin{sphinxVerbatim}[commandchars=\\\{\}]
\llap{\color{nbsphinxin}[19]:\,\hspace{\fboxrule}\hspace{\fboxsep}}\PYG{c+c1}{\PYGZsh{}obtém apenas as geometrias dos setores, para plotar por traz das geometrias de UDM}
\PYG{n}{setor\PYGZus{}censitario} \PYG{o}{=} \PYG{n}{obterCamada}\PYG{p}{(}\PYG{l+s+s1}{\PYGZsq{}}\PYG{l+s+s1}{setor\PYGZus{}censitario}\PYG{l+s+s1}{\PYGZsq{}}\PYG{p}{,} \PYG{n}{simples}\PYG{o}{=}\PYG{k+kc}{True}\PYG{p}{)}

\PYG{c+c1}{\PYGZsh{}obtém as geometrias e todos os indicadores do udm}
\PYG{n}{udm}\PYG{p}{,} \PYG{n}{meta\PYGZus{}udm} \PYG{o}{=} \PYG{n}{obterIndicador}\PYG{p}{(}\PYG{l+m+mi}{2669}\PYG{p}{)}
\PYG{n}{pprint}\PYG{o}{.}\PYG{n}{pprint}\PYG{p}{(}\PYG{n}{meta\PYGZus{}udm}\PYG{p}{)}

\PYG{c+c1}{\PYGZsh{}restringindo a área de anaálise para o centro da cidade:}
\PYG{n}{setores\PYGZus{}centro} \PYG{o}{=} \PYG{n}{juncaoEspacial}\PYG{p}{(}\PYG{n}{setor\PYGZus{}censitario}\PYG{p}{,} \PYG{n}{centro}\PYG{p}{,} \PYG{n}{tipo}\PYG{o}{=}\PYG{l+s+s1}{\PYGZsq{}}\PYG{l+s+s1}{inner}\PYG{l+s+s1}{\PYGZsq{}}\PYG{p}{,} \PYG{n}{predicado}\PYG{o}{=}\PYG{l+s+s1}{\PYGZsq{}}\PYG{l+s+s1}{intersects}\PYG{l+s+s1}{\PYGZsq{}}\PYG{p}{)}

\PYG{c+c1}{\PYGZsh{}restringindo a área de anaálise para o centro da cidade:}
\PYG{n}{udm\PYGZus{}centro} \PYG{o}{=} \PYG{n}{juncaoEspacial}\PYG{p}{(}\PYG{n}{udm}\PYG{p}{,} \PYG{n}{centro}\PYG{p}{,} \PYG{n}{tipo}\PYG{o}{=}\PYG{l+s+s1}{\PYGZsq{}}\PYG{l+s+s1}{inner}\PYG{l+s+s1}{\PYGZsq{}}\PYG{p}{,} \PYG{n}{predicado}\PYG{o}{=}\PYG{l+s+s1}{\PYGZsq{}}\PYG{l+s+s1}{intersects}\PYG{l+s+s1}{\PYGZsq{}}\PYG{p}{)}

\end{sphinxVerbatim}
}

\end{sphinxuseclass}
\begin{sphinxuseclass}{nboutput}
\begin{sphinxuseclass}{nblast}
{

\kern-\sphinxverbatimsmallskipamount\kern-\baselineskip
\kern+\FrameHeightAdjust\kern-\fboxrule
\vspace{\nbsphinxcodecellspacing}

\sphinxsetup{VerbatimColor={named}{white}}
\sphinxsetup{VerbatimBorderColor={named}{nbsphinx-code-border}}
\begin{sphinxuseclass}{output_area}
\begin{sphinxuseclass}{}


\begin{sphinxVerbatim}[commandchars=\\\{\}]
[\{'descrição': 'Índice da dimensão Infraestrutura Urbana, é um dos 3 índices '
               'que compõem o IVS. É obtido através da média ponderada de '
               'índices normalizados construídos a partir dos indicadores que '
               'compõem esta dimensão, a saber: 1) Percentual da população que '
               'vive em domicílios urbanos sem o serviço de coleta de lixo '
               '(peso: 0,300); 2) Percentual de pessoas em domicílios com '
               'abastecimento de água e esgotamento sanitário inadequados '
               '(peso: 0,300); 3) Percentual de pessoas em domicílios '
               'vulneráveis à pobreza e que gastam mais de uma hora até o '
               'trabalho no total de pessoas ocupadas, vulneráveis e que '
               'retornam diariamente do trabalho (peso: 0,400).',
  'suporte espacial': 'ivs\_ipea\_sao\_luis',
  'tabela': 'atlas\_vis\_udh',
  'variável': 'atlas\_vis\_udh\_ivs\_infraestrutura\_urbana'\}]
\end{sphinxVerbatim}



\end{sphinxuseclass}
\end{sphinxuseclass}
}

\end{sphinxuseclass}
\end{sphinxuseclass}
\begin{sphinxuseclass}{nbinput}
{
\sphinxsetup{VerbatimColor={named}{nbsphinx-code-bg}}
\sphinxsetup{VerbatimBorderColor={named}{nbsphinx-code-border}}
\begin{sphinxVerbatim}[commandchars=\\\{\}]
\llap{\color{nbsphinxin}[20]:\,\hspace{\fboxrule}\hspace{\fboxsep}}\PYG{n}{m} \PYG{o}{=} \PYG{n}{visMultiMapa}\PYG{p}{(}\PYG{n}{width}\PYG{o}{=}\PYG{l+s+s1}{\PYGZsq{}}\PYG{l+s+s1}{80}\PYG{l+s+s1}{\PYGZpc{}}\PYG{l+s+s1}{\PYGZsq{}}\PYG{p}{,} \PYG{n}{MAPA\PYGZus{}ZOOM}\PYG{o}{=}\PYG{l+m+mi}{14}\PYG{p}{)}
\PYG{n}{m} \PYG{o}{=} \PYG{n}{visMultiMapa}\PYG{p}{(}\PYG{n}{m}\PYG{p}{,} \PYG{n}{tipo}\PYG{o}{=}\PYG{l+s+s1}{\PYGZsq{}}\PYG{l+s+s1}{layer}\PYG{l+s+s1}{\PYGZsq{}}\PYG{p}{,} \PYG{n}{dado}\PYG{o}{=}\PYG{n}{setores\PYGZus{}centro}\PYG{p}{)}
\PYG{n}{m} \PYG{o}{=} \PYG{n}{visMultiMapa}\PYG{p}{(}\PYG{n}{m}\PYG{p}{,} \PYG{n}{tipo}\PYG{o}{=}\PYG{l+s+s1}{\PYGZsq{}}\PYG{l+s+s1}{choro}\PYG{l+s+s1}{\PYGZsq{}}\PYG{p}{,}
                    \PYG{n}{dado}\PYG{o}{=}\PYG{n}{udm\PYGZus{}centro}\PYG{p}{,}
                    \PYG{n}{variavel}\PYG{o}{=}\PYG{l+s+s1}{\PYGZsq{}}\PYG{l+s+s1}{atlas\PYGZus{}vis\PYGZus{}udh\PYGZus{}ivs\PYGZus{}infraestrutura\PYGZus{}urbana}\PYG{l+s+s1}{\PYGZsq{}}\PYG{p}{,}
                    \PYG{n}{alias}\PYG{o}{=}\PYG{l+s+s1}{\PYGZsq{}}\PYG{l+s+s1}{Infraestrutura Urbana}\PYG{l+s+s1}{\PYGZsq{}}\PYG{p}{)}
\PYG{n}{m}
\end{sphinxVerbatim}
}

\end{sphinxuseclass}
\begin{sphinxuseclass}{nboutput}
\begin{sphinxuseclass}{nblast}
{

\kern-\sphinxverbatimsmallskipamount\kern-\baselineskip
\kern+\FrameHeightAdjust\kern-\fboxrule
\vspace{\nbsphinxcodecellspacing}

\sphinxsetup{VerbatimColor={named}{white}}
\sphinxsetup{VerbatimBorderColor={named}{nbsphinx-code-border}}
\begin{sphinxuseclass}{output_area}
\begin{sphinxuseclass}{}


\begin{sphinxVerbatim}[commandchars=\\\{\}]
\llap{\color{nbsphinxout}[20]:\,\hspace{\fboxrule}\hspace{\fboxsep}}<folium.folium.Map at 0x7fba90e09160>
\end{sphinxVerbatim}



\end{sphinxuseclass}
\end{sphinxuseclass}
}

\end{sphinxuseclass}
\end{sphinxuseclass}

\chapter{Índices}
\label{\detokenize{index:indices}}\begin{itemize}
\item {} 
\sphinxAtStartPar
\DUrole{xref,std,std-ref}{genindex}

\item {} 
\sphinxAtStartPar
\DUrole{xref,std,std-ref}{search}

\end{itemize}


\renewcommand{\indexname}{Python Module Index}
\begin{sphinxtheindex}
\let\bigletter\sphinxstyleindexlettergroup
\bigletter{a}
\item\relax\sphinxstyleindexentry{app}\sphinxstyleindexpageref{pdcvis:\detokenize{module-app}}
\end{sphinxtheindex}

\renewcommand{\indexname}{Index}
\printindex
\end{document}